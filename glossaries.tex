\chapter*{术语和符号表}

\section*{术语的翻译}

如下是一些尚未广为流传的中文术语 (其中一部分是本人的翻译) 与外文的对应.

\begin{center}
	\begin{tabular}
		{ll}
		中文&外文\\\hline
		\coherent{}逻辑 & coherent logic \\
		\cohesive{}\topos{} & cohesive topos \\
		教条 & doctrine \\
		\fm{} & frame \\
		联合满射族 & jointly epimorphic/surjective family \\
		位象 & locale \\
		\regular{}逻辑 & regular logic \\
		景 & site \\
		提纲 & sketch \\
		清晰空间 & sober space \\
		\topos{} & topos \\
		旋子 & torsor
	\end{tabular}
\end{center}

\section*{符号}

如下是一些可能不通用, 或不广为人知的数学符号.

\begin{center}
	\begin{tabular}
		{ll}
		符号&含义\\ \hline
		$\internalprop{-}$&借用自然语言表达的内语言中的命题\\
		$\interpretation{-}$&逻辑公式或类型论陈述在范畴语义中的解释\\
		$\square$&Lawvere--Tierney 拓扑 (定义 \ref{Lawvere--Tierney-topology-internal-definition})\\
		$\yo$&米田嵌入 (定义 \ref{definition-yoneda-embedding})\\
		$\yo_{(\mathcal C,J)}$& 景到层范畴的米田嵌入 (定义 \ref{sheafified-yoneda})
	\end{tabular}
\end{center}