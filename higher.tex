\chapter{高阶\topos{}}

\philoquote{Quite contrary to superficial perception, higher topos theory provides just the mathematical context that physicists are often intuitively but informally assuming anyway.}{Urs Schreiber, \cite{HTTP}}

% https://q.uiver.app/#q=WzAsNCxbMCwwLCJcXHRleHR76ZuG5ZCIfSJdLFswLDEsIlxcdG9wb3N7fSJdLFsxLDAsIlxcdGV4dHvnqbrpl7R9Il0sWzEsMSwiXFxpbmZ0eSBcXHRleHR7LX0gXFx0b3Bvc3t9Il0sWzAsMSwiIiwwLHsic3R5bGUiOnsiYm9keSI6eyJuYW1lIjoic3F1aWdnbHkifX19XSxbMiwzLCIiLDAseyJzdHlsZSI6eyJib2R5Ijp7Im5hbWUiOiJzcXVpZ2dseSJ9fX1dXQ==
\[\begin{tikzcd}[ampersand replacement=\&]
	{\text{集合}} \& {\text{\topos{}}} \\
	{\text{空间}} \& {\infty \text{-\topos{}}}
	\arrow[rightsquigarrow, from=1-1, to=2-1]
	\arrow[rightsquigarrow, from=1-2, to=2-2]
\end{tikzcd}\]

\minitoc

\section{$\infty$-范畴: 单纯集模型}

粗略地说, 一个 $\infty$-范畴含有如下成分: 对象, 对象之间的态射, 态射之间的 $2$-态射, $\cdots$, $k$-态射之间的 $(k+1)$-态射, 以至于无穷. 我们使用的 $\infty$-范畴又称 $(\infty ,1)$-范畴, 意为对所有 $k>1$, $k$-态射都可逆.

在实践中, $\infty$-范畴有许多不同而可以互相转化的\emph{模型}, 就像一个算法由许多不同的编程语言实现. 单纯集 (定义 \ref{Simplicial-Sets}) 就是一种实用的 ``编程语言''. 我们先介绍用单纯集表达的一种 $\infty$-范畴的模型, 它是 Lurie \cite{HTT} 使用的模型, 也是最简单的模型; 但后面我们陈述的关于无穷范畴的任何结论都可叙述为模型无关的形式.

\begin{definition}
	[label={simplicial-set-horn}]
	{(角形)}
	回忆单纯集 $\Delta^n$ 为 $[n]\in\Delta$ 在米田嵌入下的像 $\yo([n])$. 对于单射 $[m]\to [n]$, 设其像为 $J$, 定义单纯集 $\Delta^J$ 为对应的态射 $\Delta^m\to\Delta^n$ 的像 (作为 $\Delta^n$ 的子对象). 对于 $0\leq k \leq n$, 定义
	$$
	\Lambda_k^n : = \bigcup_{k\in J\neq [n]}\Delta^J,
	$$
	称为\emph{角形} (horn). 其中对应 $0<k<n$ 的角形称为\emph{内角形} (inner horn).
\end{definition}

角形是用来描述单纯集模型 $\infty$-范畴中一些结构的图形. 如下是角形 $\Lambda_k^2$ ($k=0,1,2$) 的示意图. 可以看到它们是互不同构的单纯集, 其中只有内角形 $\Lambda_1^2$ 表示 ``两个首尾相接的箭头''.
% https://q.uiver.app/#q=WzAsMTIsWzAsMiwiMCJdLFsxLDAsIjEiXSxbMiwyLCIyIl0sWzEsMywiXFxMYW1iZGFfMF4yIl0sWzMsMiwiMCJdLFs0LDAsIjEiXSxbNSwyLCIyIl0sWzQsMywiXFxMYW1iZGFfMV4yIl0sWzcsMywiXFxMYW1iZGFfMl4yIl0sWzYsMiwiMCJdLFs4LDIsIjIiXSxbNywwLCIxIl0sWzAsMV0sWzAsMl0sWzQsNV0sWzUsNl0sWzksMTBdLFsxMSwxMF1d
\[\begin{tikzcd}[ampersand replacement=\&,column sep=0, row sep=0.8em]
	\& 1 \&\&\& 1 \&\&\& 1 \\
	\\
	0 \&\& 2 \& 0 \&\& 2 \& 0 \&\& 2 \\
	\& {\Lambda_0^2} \&\&\& {\Lambda_1^2} \&\&\& {\Lambda_2^2}
	\arrow[from=3-1, to=1-2]
	\arrow[from=3-1, to=3-3]
	\arrow[from=3-4, to=1-5]
	\arrow[from=1-5, to=3-6]
	\arrow[from=3-7, to=3-9]
	\arrow[from=1-8, to=3-9]
\end{tikzcd}\]

\begin{definition}
	[label={infinity-category-definition}]
	{($\infty$-范畴)}
	\emph{$\infty$-范畴} (又称\emph{拟范畴}) 是满足如下条件的单纯集 $\mathcal X$: 对所有整数 $0<k<n$,
	$$
	\operatorname{Hom}(\Delta^n,\mathcal X) \to \operatorname{Hom}(\Lambda_k^n,\mathcal X)
	$$
	是满射; 换言之, 如下提升总存在 (但\emph{不要求唯一}):
	\[\begin{tikzcd}[ampersand replacement=\&]
		{\Lambda_k^n} \& {X.} \\
		{\Delta^n}
		\arrow[from=1-1, to=2-1]
		\arrow[from=1-1, to=1-2]
		\arrow["\exists"', dashed, from=2-1, to=1-2]
	\end{tikzcd}\]
	称之为内角形的\emph{填充} (filler).
	
	设单纯集 $\mathcal X$ 是 $\infty$-范畴. 定义
	\begin{itemize}
		\item $\mathcal X$ 中的\emph{对象}为 $X_0$ 的元素, 即单纯集映射 $\Delta^0\to X$;
		\item $\mathcal X$ 中的\emph{态射} (\emph{箭头}) 为 $X_1$ 的元素, 即单纯集映射 $\Delta^1\to X$, 对象 $x$ 上的恒等态射 $\operatorname{id}_x$ 为映射 $\Delta^1\to \Delta^0\overset{x}{\to}X$;
		\item $\mathcal X$ 中的\emph{实心三角形}为 $X_2$ 的元素, 即单纯集映射 $\Delta^2\to X$, 对于实心三角形 $\begin{tikzcd}[ampersand replacement=\&,column sep=0,row sep=0.8em]
			\& y \\
			x \&\& z,
			\arrow["f", from=2-1, to=1-2]
			\arrow["g", from=1-2, to=2-3]
			\arrow["h"', from=2-1, to=2-3]
		\end{tikzcd}$
		称 $h$ 为 $f$ 与 $g$ 的一个\emph{复合}. (很明显, 复合不是唯一的.) 若 $\operatorname{id}_x$ 为 $f,g$ 的一个复合, 则称 $g$ 为 $f$ 的一个\emph{逆}. (当然, 逆也不是唯一的.)
	\end{itemize}
\end{definition}

\begin{definition}
	{(单纯集的对偶)}
	考虑函子 $(-)^{\op} \colon \Delta \to \Delta$.
	对于单纯集 $\mathcal X$, 定义其\emph{对偶} $\mathcal X^{\op}$ 为 $\mathcal X^{\op}:= \mathcal X\circ (-)^{\op}\colon \Delta \to\mathsf {Set}$.
\end{definition}

\begin{example}
	{(普通范畴的脉)}
	回忆一个普通范畴 $\mathcal C$ 的脉 $\text{N}(\mathcal C)$ (例 \ref{sset-geometric-realization}) 定义如下,
	$$
	\text{N}\mathcal{C}_n = \mathsf {Fun}(0\to 1\to \cdots \to n,\mathcal C),
	$$
	即 $\text{N}(\mathcal C)_n$ 的元素是 $\mathcal C$ 中连续的 $n$ 个箭头. 由于对任意 $0<k<n$, $\Lambda_k^n$ 都包含一条折线 $0\to 1\to\cdots\to n$, 故映射 $\Lambda_k^n\to \text{N}(\mathcal C)$ 总能提升为 $\Delta^n\to \text{N}(\mathcal C)$, $\text{N}(\mathcal C)$ 是一个 $\infty$-范畴.
\end{example}


\begin{remark}
	{($\infty$-范畴单纯集模型的注意事项)}
	定义 \ref{infinity-category-definition} 中有两点需要注意; 如果忽视这两点, 就会得到另外两种东西.
	\begin{itemize}
		\item 只有内角形可以填充. 若所有角形都可以填充, 则可证明 $\mathcal X$ 的所有态射都可逆, 我们称之为 \emph{$\infty$-群胚}.
		\item 内角形填充不要求唯一. 若内角形填充存在且唯一, 则 $\mathcal X$ 实际上来自一个普通范畴的脉. 直观上, $\infty$-范畴是一种 ``弱化'' 的范畴, 其中的复合是在同伦意义下谈论的. 可以证明\footnotemark{}, 对 $\infty$-范畴中的两个态射 $f\colon x\to y, g\colon y\to z$, 其所有可能的复合构成一个\emph{可缩 Kan 复形} (定义 \ref{equivalence-contractible}).
	\end{itemize}
	因此, $\infty$-范畴可视为普通范畴与 $\infty$-群胚的共同推广.
\end{remark}
\footnotetext{\url{https://kerodon.net/tag/0078}}

%\begin{prop}
%	{}
%	对于单纯集 $X$, 以下条件等价:
%	\begin{itemize}
%		\item
%		对所有整数 $0<k<n$,
%		$
%		\operatorname{Hom}(\Delta^n,X) \to \operatorname{Hom}(\Lambda_k^n,X)
%		$
%		是双射;
%		\item
%		对所有整数 $0<n$,
%		$
%		\operatorname{Hom}(\Delta^n,X) \to \operatorname{Hom}(\operatorname{N}(0\to 1 \to \cdots\to n),X)
%		$
%		是双射;
%		\item
%		$X$ 同构于某个范畴的脉.
%	\end{itemize}
%\end{prop}
%
%第二个条件表明单纯集 $X$ 完全由 $X_0,X_1$ 即 ``$0$ 维和 $1$ 维的信息'' 决定.

\begin{propdef}
	{($\infty$-群胚, Kan 复形模型)}
	定义 \emph{$\infty$-群胚} (又称 \emph{Kan 复形}) 是满足如下等价条件之一的 $\infty$-范畴 $\mathcal X$:
	\begin{itemize}
		\item $\mathcal X$ 中所有态射都可逆;
		\item $\mathcal X$ 中所有角形都可填充, 即对所有整数 $0\leq k\leq n$,
		$$
		\operatorname{Hom}(\Delta^n,X) \to \operatorname{Hom}(\Lambda_k^n,X)
		$$
		是满射.
	\end{itemize}
\end{propdef}

\begin{proof}
	\todo{}
\end{proof}

\begin{example}
	{(基本 $\infty$-群胚)}
	拓扑空间 $X$ 的奇异单纯集 $\operatorname{Sing}X$ 是 $\infty$-群胚, 称为\emph{基本 $\infty$-群胚} $\pi_{\infty}(X)$.
\end{example}

\begin{propdef}
	{(函子, 函子范畴)}
	定义 $\infty$-范畴之间的\emph{函子}为单纯集的映射.
	对于 $\infty$-范畴 $\mathcal X$ 与任意单纯集 $A$, 单纯集的指数对象 ${\mathcal X}^A$ 都是 $\infty$-范畴. 特别地, 对于无穷范畴 $\mathcal X,\mathcal Y$ 定义\emph{函子范畴} $\mathsf {Fun}(\mathcal X,\mathcal Y) := {\mathcal Y}^{\mathcal X}$. 函子范畴中的态射称为\emph{自然变换} (换言之, 两个函子 $\mathcal X\to \mathcal Y$ 之间的自然变换是单纯集映射 $\Delta^1\times \mathcal X \to \mathcal Y$).
\end{propdef}

\begin{definition}
	[label={equivalence-contractible}]
	{(范畴等价, 可缩)}
	对于 $\infty$-范畴 $\mathcal X,\mathcal Y$ 之间的函子 $u\colon \mathcal X\to \mathcal Y$, 称 $u$ 为一个\emph{等价}是指存在函子 $v\colon \mathcal Y\to \mathcal X$, 以及两个可逆的自然变换 $uv\to \operatorname{id}_{\mathcal Y}$, $\operatorname{id}_{\mathcal X}\to vu$.
	称等价于 $\Delta^0$ 的 $\infty$-范畴是\emph{可缩}的.
\end{definition}

\begin{definition}
	{(态射集)}
	对于 $\infty$-范畴 $\mathcal X$ 以及其中的对象 $x,y$, 定义单纯集 $\operatorname{Hom}_{\mathcal X}(x,y)$ 为如下的拉回.
	\[\begin{tikzcd}[ampersand replacement=\&,column sep=1em,row sep=2em]
		{\hspace{-1em}\operatorname{Hom}_{\mathcal X}(x,y)} \& {\mathsf {Fun}(\Delta^1,\mathcal X)\hspace{-1em}} \\
		{\Delta^0} \& {\mathcal X\times \mathcal X}
		\arrow["{(s,t)}", from=1-2, to=2-2]
		\arrow["{(x,y)}"', from=2-1, to=2-2]
		\arrow[from=1-1, to=2-1]
		\arrow[from=1-1, to=1-2]
	\end{tikzcd}\]
	其中 $s,t\colon \mathsf {Fun}(\Delta^1,\mathcal X)\to \mathcal X$ 将 $\mathcal X$ 的态射对应到其起点与终点.
	
	更一般地, 对 $\mathcal X$ 中 $n+1$ 个对象 $x_0,x_1,\cdots,x_n$, 定义 $\operatorname{Hom}_{\mathcal X}(x_0,x_1,\cdots,x_n)$ 为如下的拉回.
	\[\begin{tikzcd}[ampersand replacement=\&,column sep=1em,row sep=2em]
		{\hspace{-1em}\operatorname{Hom}_{\mathcal X}(x_0,\cdots,x_n)} \& {\mathsf {Fun}(\Delta^n,\mathcal X)\hspace{-1em}} \\
		{\Delta^0} \& {\mathcal X^{n+1}}
		\arrow["{}", from=1-2, to=2-2]
		\arrow["{(x_0,\cdots,x_n)}"', from=2-1, to=2-2]
		\arrow[from=1-1, to=2-1]
		\arrow[from=1-1, to=1-2]
	\end{tikzcd}\]
\end{definition}

如果 $\infty$-群胚是空间的模型, 那么 $\operatorname{Hom}_{\mathcal X}(x,y)$ 就是两点 $x,y$ 之间的道路的空间.

\begin{example}
	{}
	设 $\mathcal X$ 为 $\infty$-群胚, $x$ 为其中的对象, 那么单纯集 $\operatorname{Hom}_{\mathcal X}(x,x)$ 在同伦论上又叫\emph{环路空间} $\Omega(\mathcal X,x)$, 其连通分支的集合给出\emph{基本群} $\pi_1({\mathcal X},x)$.
\end{example}

如下命题表示我们考虑的 $\infty$-范畴中 ``$k$-态射都可逆'' ($k>1$).

\begin{prop}
	{}
	对 $\infty$-范畴 $\mathcal X$ 中的任意两个对象 $x,y$, $\operatorname{Hom}_{\mathcal X}(x,y)$ 是 $\infty$-群胚.
\end{prop}

\subsection{同伦}

\newcommand{\Ho}{\text{Ho}}

\begin{propdef}
	{(态射的同伦, 同伦范畴)}
	对于两个态射 $f,g\colon x\to y$, 称 $f$ \emph{同伦}于 $g$ 是指 $g$ 为 $f$ 与 $\operatorname{id}_y$ 的一个复合; 记 $f\sim g$. 态射的同伦为等价关系.
	
	对于 $\infty$-范畴 $\mathcal X$, 定义其\emph{同伦范畴} $\Ho{}(\mathcal X)$ 为如下的范畴: $\Ho(\mathcal X)$ 的对象即为 $\mathcal X$ 的对象, 态射为 $\mathcal X$ 中态射的同伦类, 态射的复合是良定义的.
\end{propdef}

% HTT 1.2.3.1

\begin{proof}

我们证明同伦是一个等价关系: 由下面的示意图以及 $\infty$-范畴的定义, 可知若 $f\sim g$ 则 $g\sim f$.
\[\begin{tikzcd}[ampersand replacement=\&,sep=1.6em]
	x \&\& y \\
	\\
	y \&\& y
	\arrow["f"', from=1-1, to=3-1]
	\arrow["{\operatorname{id}_y}"', from=3-1, to=3-3]
	\arrow["g\!"'{pos=0.7}, from=1-1, to=3-3]
	\arrow["{\operatorname{id}_y\!\!\!\!}"{pos=0.7}, from=3-1, to=1-3]
	\arrow["{\operatorname{id}_y}"', from=3-3, to=1-3]
	\arrow["f", from=1-1, to=1-3]
\end{tikzcd}\]
\vspace{-2em}
\end{proof}

有趣的是, 同伦范畴正是单纯集在 $\mathsf {Cat}$ 中的几何实现 (例 \ref{sset-geometric-realization}).

\begin{prop}
	{}
	设 $\mathcal X$ 为 $\infty$-范畴, 考虑函子 $\mathcal X\to \text{N}(\Ho{}(\mathcal X))$ 将 $\mathcal X$ 的点映射到 $\Ho{}(\mathcal X)$ 的对象, $n$-单形映射到 $\Ho{}(X)$ 的连续 $n$ 个态射. 那么这个函子给出了范畴的同构
	$$
	|\mathcal X|\simeq\Ho{}(\mathcal X),
	$$
	其中 $|{-}|$ 是例 \ref{sset-geometric-realization} 提到的脉函子的左伴随.
\end{prop}

同伦范畴还可通过单纯范畴定义: 由定义 \ref{coherent-nerve}, 对于 $\infty$-范畴 $\mathcal X$, $\mathfrak {C}[\mathcal X]$ 是一个 $\mathsf {sSet}$-充实范畴; 将其中的态射单纯集替换为连通分支便得到同伦范畴. 见 HTT \cite{HTT} 定义 1.1.5.14. 总结起来, 我们有如下图表.
% https://q.uiver.app/#q=WzAsMyxbMiwyLCJcXG1hdGhzZiB7c0NhdH0iXSxbMCwwLCJcXG1hdGhzZiB7c1NldH0iXSxbNCwwLCJcXG1hdGhzZiB7Q2F0fSJdLFsxLDIsIlxcSG97fSIsMCx7Im9mZnNldCI6LTN9XSxbMiwxLCJcXHRleHR7Tn0iLDAseyJvZmZzZXQiOi0xfV0sWzAsMSwiXFx0ZXh0e059XntcXHRleHR7Y319IiwwLHsib2Zmc2V0IjotM31dLFsxLDAsIlxcbWF0aGZyYWsge0N9W3stfV0iLDAseyJvZmZzZXQiOi0xfV0sWzIsMCwiaSIsMCx7Im9mZnNldCI6LTMsInN0eWxlIjp7InRhaWwiOnsibmFtZSI6Imhvb2siLCJzaWRlIjoiYm90dG9tIn19fV0sWzAsMiwiXFx0ZXh0e2hvfSIsMCx7Im9mZnNldCI6LTF9XSxbMyw0LCIiLDAseyJsZXZlbCI6MSwic3R5bGUiOnsibmFtZSI6ImFkanVuY3Rpb24ifX1dLFs2LDUsIiIsMCx7ImxldmVsIjoxLCJzdHlsZSI6eyJuYW1lIjoiYWRqdW5jdGlvbiJ9fV0sWzgsNywiIiwwLHsibGV2ZWwiOjEsInN0eWxlIjp7Im5hbWUiOiJhZGp1bmN0aW9uIn19XV0=
\[\begin{tikzcd}[ampersand replacement=\&]
	{\mathsf {sSet}} \&\&\&\& {\mathsf {Cat}} \\
	\\
	\&\& {\mathsf {sCat}}
	\arrow[""{name=0, anchor=center, inner sep=0}, "{\Ho{}}", shift left=3, from=1-1, to=1-5]
	\arrow[""{name=1, anchor=center, inner sep=0}, "{\text{N}}", shift left, from=1-5, to=1-1]
	\arrow[""{name=2, anchor=center, inner sep=0}, "{\text{N}^{\text{c}}}", shift left=3, from=3-3, to=1-1]
	\arrow[""{name=3, anchor=center, inner sep=0}, "{\mathfrak {C}[{-}]}", shift left, from=1-1, to=3-3]
	\arrow[""{name=4, anchor=center, inner sep=0}, "i", shift left=3, hook', from=1-5, to=3-3]
	\arrow[""{name=5, anchor=center, inner sep=0}, "{\text{ho}}", shift left, from=3-3, to=1-5]
	\arrow["\dashv"{anchor=center, rotate=-90}, draw=none, from=0, to=1]
	\arrow["\dashv"{anchor=center, rotate=-135}, draw=none, from=3, to=2]
	\arrow["\dashv"{anchor=center, rotate=-45}, draw=none, from=5, to=4]
\end{tikzcd}\]



我们还需要描述 $\infty$-范畴的满子范畴.

%\todo{子范畴 , HTT 1.2.11}

\begin{definition}
	{}
	设 $\mathcal C$ 是 $\infty$-范畴, $\mathcal S\to \Ho (\mathcal C)$ 是其同伦范畴的子范畴.
	定义 \emph{$\mathcal S$ 张成的 $\mathcal C$ 的满子范畴}为如下 (作为单纯集的) 拉回.
	\[
	\begin{tikzcd}
		\mathcal {S}\times_{\text{N}(\Ho (\mathcal C))} \mathcal C \ar[r]\ar[d]& \mathcal C\ar[d]\\
		\mathcal S \ar[r]& \text{N}(\Ho (\mathcal C))
	\end{tikzcd}
	\]
\end{definition}

\subsection{单纯范畴}

$\infty$-范畴的另一种模型是用单纯范畴描述的, 其优点包括
\begin{itemize}
	\item 用单纯范畴模型方便给出某些具体的 $\infty$-范畴以及函子;
	\item 单纯范畴中态射的复合唯一定义;
\end{itemize}
但这种模型的同伦论较难处理.

\begin{definition}
	{(单纯范畴)}
	\emph{单纯范畴}是指充实于 $\mathsf {sSet}$ 的范畴.
	具体地, 我们有一个对象集合 $\operatorname{Ob}(\mathcal C)$,
	对 $x,y\in\operatorname{Ob}(\mathcal C)$ 有一个单纯集 $\operatorname{Hom}_{\mathcal C}(x,y)$,
	对 $x\in\operatorname{Ob}(\mathcal C)$ 有恒等态射 $\operatorname{id}_x\in\operatorname{Hom}(x,x)$,
	对 $x,y,z\in\operatorname{Ob}(\mathcal C)$ 有单纯集映射
	$$
	\circ\colon \operatorname{Hom}_{\mathcal C}(x,y) \times \operatorname{Hom}_{\mathcal C}(y,z) \to \operatorname{Hom}_{\mathcal C}(x,z),
	$$
	满足结合律与幺元律.
	
	等价地, 单纯范畴也可定义为小范畴范畴 $\mathsf {Cat}$ 中的内蕴单纯集 $\mathcal C\colon \Delta^{\op}\to\mathsf {Cat}$,
	满足 ``对象的单纯集'' $\operatorname{Ob}(\mathcal C) := \operatorname{Ob}\circ\mathcal C\colon \Delta^{\op} \to \mathsf {Set}$
	是常值单纯集.
	
	记 (小) 单纯范畴的范畴为 $\mathsf {sCat}$, 其中的态射是单纯范畴之间的 $\mathsf {sSet}$-充实函子.
\end{definition}

\begin{definition}
	{(纤维性单纯范畴)}
	若单纯范畴 $\mathcal C$ 的态射集 $\operatorname{Hom}_{\mathcal C}(x,y)$ 均为 Kan 复形 (即前面定义的 $\infty$-群胚), 则称之为\emph{纤维性} (fibrant) 单纯范畴; 它是 $\infty$-范畴的另一种模型. 换言之, $\infty$-范畴可视为充实于 $\infty$-群胚的范畴.
\end{definition}

\begin{example}
	{(拓扑空间范畴)}
	拓扑空间范畴 $\mathsf {Top}$ 具有单纯范畴结构:
	$$
	\operatorname{Hom}(X,Y)_n := \{ \text{连续函数 $|\Delta^n| \times X \to Y$ } \}.
	$$
\end{example}

\begin{propdef}
	{($\infty$-范畴的极大子 $\infty$-群胚)}
	设 $\mathcal X$ 为 $\infty$-范畴. 记 $\mathcal X^\simeq$ 为所有边都可逆的单形 $\Delta^n \to \mathcal X$ 构成的子单纯集, 则 $\mathcal X^\simeq$ 为 $\mathcal X$ 的\emph{极大子 $\infty$-群胚}, 即任何 $\infty$-群胚到 $\mathcal X$ 的函子唯一地穿过 $\mathcal X^\simeq$; $\infty$-群胚 $\mathcal X^\simeq$ 又称 $\mathcal X$ 的\emph{核心} (core).
	(关于普通范畴的极大子群胚, 见例 \ref{Gpd-Cat-adjunction}.)
\end{propdef}

\begin{example}
	{($\infty$-范畴的单纯范畴)}
	记 $\infty\mathsf{Cat}$ 为 (小) $\infty$-范畴的范畴 (它是一个普通范畴), 对 $\infty$-范畴 $\mathcal X,\mathcal Y$ 定义 $\operatorname{Hom}_{\infty\mathsf {Cat}}(\mathcal X,\mathcal Y)$ 为 $\mathsf {Fun}(\mathcal X,\mathcal Y)$ 的极大子 $\infty$-群胚; 这样 $\infty\mathsf {Cat}$ 构成一个纤维性单纯范畴.
\end{example}

\begin{example}
	{(链复形范畴)}
	设 $R$ 为环, $\mathsf {Ch}(R)$ 为 $R$ 上的链复形的范畴. 回忆 $R$ 上的\emph{链复形}是指 $R$-模范畴中的一个图表 $$M_\bullet = \cdots \to M_2\overset{\partial}{\to} M_1\overset{\partial}{\to} M_0\overset{\partial}{\to} M_{-1}\overset{\partial}{\to} M_{-2} \to \cdots,$$
	满足 $\partial\circ \partial= 0$.
	$\mathsf {Ch}(R)$ 可赋予单纯范畴结构. 首先构造 $\mathbb{Z}$ 上的链复形 $C_\bullet(\Delta^n)$: 对于 $0\leq k \leq n$ 其第 $k$ 位置是 $\Delta^n$ 的非退化 $k$-单形自由生成的 Abel 群, 边界映射 $\partial:= \sum_{i=0}^k (-1)^i d_i$ 来自单形的面映射 $d_i$. 例如
	$$
	C_\bullet(\Delta^2) = \cdots\to 0\to \mathbb{Z} \to \mathbb{Z}^3 \to \mathbb{Z}^3 \to 0 \to\cdots.
	$$
	(熟悉代数拓扑的读者知道, $C_\bullet(X)$ 就是用于计算单纯同调 $H_\bullet(X)$ 的那个链复形.)
	定义
	$$
	\operatorname{Hom}(M_\bullet,N_\bullet)_n := \operatorname{Hom}_{\mathsf {Ch}(R)}
	(M_\bullet \otimes_{\mathbb{Z}} C_\bullet(\Delta^n),N_\bullet).
	$$
	这个范畴是 (现代的) 代数 $K$-理论的起点. 函子 $C_\bullet\colon \Delta\to\mathsf {Ch}(\mathbb{Z})$ 给出的脉--几何实现伴随限制为单纯 Abel 群与非负位置链复形之间的范畴等价
	\[\begin{tikzcd}[ampersand replacement=\&]
		{\mathsf {sAb}} \& {\mathsf {Ch}_{\geq 0}(\mathbb{Z})}
		\arrow[""{name=0, anchor=center, inner sep=0}, "|{-}|_{C_\bullet}", shift left=2, from=1-1, to=1-2]
		\arrow[""{name=1, anchor=center, inner sep=0}, "\operatorname{N}", shift left=2, from=1-2, to=1-1]
		\arrow["\simeq"{anchor=center}, draw=none, from=0, to=1]
	\end{tikzcd},
	\]
	称为 \emph{Dold--Kan 对应}.
\end{example}

\begin{definition}
	{(偏序集的道路范畴)}
	对于偏序集 $(S,\leq)$, 定义其\emph{道路范畴}为一个单纯范畴 $\operatorname{Path}(S)$,
	其对象集为 $S$, 对两个元素 $x,y\in S$, 单纯集 $\operatorname{Hom}_{\operatorname{Path}(S)}(x,y)$ 是 ``由 $x$ 到 $y$ 道路的空间'',
	它定义为所有形如 $\{x=x_0\leq x_1\leq\cdots \leq x_m=y\}$ 的链构成的偏序集的脉, 其序关系为包含关系的反序 (最大元为 $x\leq y$).
	$\operatorname{Path}(S)$ 中态射的复合即是链的并.
\end{definition}

% https://q.uiver.app/#q=WzAsOCxbMCwwLCIwIl0sWzEsMSwiMSJdLFsyLDAsIjIiXSxbMSwyLCJTIl0sWzQsMiwiXFxvcGVyYXRvcm5hbWV7UGF0aH0oUykiXSxbMywwLCIwIl0sWzQsMSwiMSJdLFs1LDAsIjIiXSxbMCwxXSxbMSwyXSxbMCwyXSxbNSw2LCIwXFxsZXEgMSIsMix7Im9mZnNldCI6MX1dLFs2LDcsIjFcXGxlcSAyIiwyLHsib2Zmc2V0IjoxfV0sWzUsNywiMFxcbGVxIDIiLDAseyJvZmZzZXQiOi0xLCJjdXJ2ZSI6LTF9XSxbNSw3LCIwXFxsZXEgMVxcbGVxIDIiLDIseyJjdXJ2ZSI6MX1dLFsxNCwxMywiIiwxLHsic2hvcnRlbiI6eyJzb3VyY2UiOjIwLCJ0YXJnZXQiOjIwfX1dXQ==
\[\begin{tikzcd}[ampersand replacement=\&]
	0 \&\& 2 \& 0 \&\& 2 \\
	\& 1 \&\&\& 1 \\
	\& S \&\&\& {\operatorname{Path}(S)}
	\arrow[from=1-1, to=2-2]
	\arrow[from=2-2, to=1-3]
	\arrow[from=1-1, to=1-3]
	\arrow["{0\leq 1}"', shift right, from=1-4, to=2-5]
	\arrow["{1\leq 2}"', shift right, from=2-5, to=1-6]
	\arrow[""{name=0, anchor=center, inner sep=0}, "{0\leq 2}", shift left, bend left=10, from=1-4, to=1-6]
	\arrow[""{name=1, anchor=center, inner sep=0}, "{0\leq 1\leq 2}"', bend right=10, from=1-4, to=1-6]
	\arrow[shorten <=2pt, shorten >=2pt, Rightarrow, from=1, to=0]
\end{tikzcd}\]

\newcommand{\Nc}{\operatorname{N}^{\text{c}}}
\begin{definition}
	[label={coherent-nerve}]
	{(单纯范畴的融贯脉)}
	考虑函子 $\operatorname{Path}\colon \Delta \to \mathsf {sCat}$,
	其对应的脉--几何实现伴随 (命题 \ref{nerve-and-realization}, 但要使用 $\mathsf {sSet}$-充实版本)
	\[
	\begin{tikzcd}[ampersand replacement=\&]
		{\mathsf {sSet}} \& {\mathsf {sCat}}
		\arrow[""{name=0, anchor=center, inner sep=0}, "{\mathfrak {C}[{-}]}", shift left=2, from=1-1, to=1-2]
		\arrow[""{name=1, anchor=center, inner sep=0}, "{\Nc}", shift left=2, from=1-2, to=1-1]
		\arrow["\dashv"{anchor=center, rotate=-90}, draw=none, from=0, to=1]
	\end{tikzcd}
	\]
	中的脉 $\Nc$ 称为单纯范畴的\emph{融贯脉}\footnotemark (coherent nerve).
	另一边, ``几何实现'' $\mathfrak C [{-}]$ 又称为拟范畴的 \emph{Joyal 固化} (rigidification).
\end{definition}
\footnotetext{又称\emph{同伦融贯脉} (homotopy coherent nerve).}

我们不加证明地陈述如下技术性引理.
\begin{prop}
	{}
	纤维性单纯范畴的融贯脉是 $\infty$-范畴.
\end{prop}

\newcommand{\infCatinfcat}{{\infty\mathcal {C}\hspace{-1pt}at}}
\newcommand{\infGpdinfcat}{{\infty\mathcal {G}\hspace{-1pt}pd}}
\begin{definition}
	{($\infty$-范畴的 $\infty$-范畴)}
	定义 $\infty$-范畴的 $\infty$-范畴, 以及 $\infty$-群胚的 $\infty$-范畴为
	$$
	\infCatinfcat := \Nc (\infty \mathsf {Cat}),\quad
	\infGpdinfcat :=\Nc (\infty \mathsf {Gpd}).
	$$
\end{definition}

正如集合范畴 $\mathsf {Set}$ 是范畴的 ``原型'', 在 $\infty$-范畴中, 扮演这个角色的是 $\infGpdinfcat$. 它可视为某种 ``空间'' (不一定是传统意义上的拓扑空间) 的 $\infty$-范畴, 其中各阶态射表达了空间之间映射的各阶同伦; 许多作者直接称其为\emph{空间的 $\infty$-范畴}, 如 HTT \cite{HTT} 1.2 节. 我们将会看到, 类似于 $\mathsf {Set}$ 是终范畴 $1$ 自由生成的余完备范畴, $\infGpdinfcat$ 是 $1$ 自由生成的余完备 $\infty$-范畴.

%\newcommand{\Topinfcat}{{\mathcal{T}\hspace{-3pt}op}}
%\begin{definition}
%	{(拓扑空间的 $\infty$-范畴)}
%	定义拓扑空间的 $\infty$-范畴为
%	$$
%	\Topinfcat := \Nc (\mathsf {Top}).
%	$$
%\end{definition}

\subsection{伴随}

%在定义极限和余极限之前, 我们需要指出 $\infty$-范畴之间的伴随. 注意以下陈述是模型无关的.

\begin{definition}
	{($\infty$-范畴之间的伴随)}
	$\infty$-范畴之间的一对伴随 \begin{tikzcd}[ampersand replacement=\&]
		{\mathcal D} \& {\mathcal C}
		\arrow[""{name=0, anchor=center, inner sep=0}, "F"', shift right=2, from=1-2, to=1-1]
		\arrow[""{name=1, anchor=center, inner sep=0}, "G"', shift right=2, from=1-1, to=1-2]
		\arrow["\dashv"{anchor=center, rotate=-90}, draw=none, from=0, to=1]
	\end{tikzcd}
	是如下资料,
	\begin{itemize}
		\item 两个 $\infty$-范畴 $\mathcal C,\mathcal D$,
		\item 两个函子 $F\colon \mathcal C\to \mathcal D$ (称为\emph{左伴随}), $G\colon \mathcal D\to \mathcal C$ (称为\emph{右伴随}),
		\item 两个自然变换 $\eta\colon \operatorname{id}_{\mathcal C}\to GF$ (称为\emph{单位}), $\varepsilon\colon FG\to\operatorname{id}_{\mathcal D}$ (称为\emph{余单位});
	\end{itemize}
	满足如下 $2$-态射的关系,
	\[\begin{tikzcd}[ampersand replacement=\&,column sep=1.5em]
		{\mathcal C} \&\& {\mathcal C} \\
		\& {\mathcal D} \&\& {\mathcal D}
		\arrow["F"', from=1-1, to=2-2]
		\arrow["G"{description}, from=2-2, to=1-3]
		\arrow["F", from=1-3, to=2-4]
		\arrow[""{name=0, anchor=center, inner sep=0}, "{\operatorname{id}_{\mathcal C}}", from=1-1, to=1-3]
		\arrow[""{name=1, anchor=center, inner sep=0}, "{\operatorname{id}_{\mathcal D}}"', from=2-2, to=2-4]
		\arrow["\eta"', shorten <=3pt, shorten >=3pt, Rightarrow, from=0, to=2-2]
		\arrow["\varepsilon", shorten <=3pt, shorten >=3pt, Rightarrow, from=1-3, to=1]
	\end{tikzcd}\hspace{-1em}\sim \operatorname{id}_F,
	\begin{tikzcd}[ampersand replacement=\&,column sep=1.5em]
		\& {\mathcal C} \&\& {\mathcal C} \\
		{\mathcal D} \&\& {\mathcal D}
		\arrow["G", from=2-1, to=1-2]
		\arrow["F"{description}, from=1-2, to=2-3]
		\arrow["G"', from=2-3, to=1-4]
		\arrow[""{name=0, anchor=center, inner sep=0}, "{\operatorname{id}_{\mathcal D}}"', from=2-1, to=2-3]
		\arrow[""{name=1, anchor=center, inner sep=0}, "{\operatorname{id}_{\mathcal C}}", from=1-2, to=1-4]
		\arrow["\varepsilon", shorten <=3pt, shorten >=3pt, Rightarrow, from=1-2, to=0]
		\arrow["\eta"', shorten <=3pt, shorten >=3pt, Rightarrow, from=1, to=2-3]
	\end{tikzcd}\hspace{-1em}\sim\operatorname{id}_{G},\]
	其中第一个式子中 $\sim$ 表示 $\operatorname{id}_F$ 是 $\infty$-范畴 $\mathsf {Fun}(\mathcal C,\mathcal D)$ 中两个态射 $F\to FGF$, $FGF\to F$ 的一个复合.
	
	记 $\Ho_2(\infCatinfcat)$ 为 $\infCatinfcat$ 的\emph{同伦 $2$-范畴}, 其中 $\operatorname{Hom}_{\Ho_2(\infCatinfcat)}(\mathcal C,\mathcal D) := \Ho{}(\operatorname{Fun}(\mathcal C,\mathcal D))$. 由定义, $\infty$-范畴之间的伴随等同于 $\Ho_2(\infCatinfcat)$ 中的伴随.
\end{definition}

\begin{prop}
	{($\infty$-范畴之间的伴随与 $\operatorname{Hom}$-函子)}
	设 \begin{tikzcd}[ampersand replacement=\&]
		{\mathcal D} \& {\mathcal C}
		\arrow[""{name=0, anchor=center, inner sep=0}, "F"', shift right=2, from=1-2, to=1-1]
		\arrow[""{name=1, anchor=center, inner sep=0}, "G"', shift right=2, from=1-1, to=1-2]
		\arrow["\dashv"{anchor=center, rotate=-90}, draw=none, from=0, to=1]
	\end{tikzcd} 为 $\infty$-范畴之间的伴随, 则对任意对象 $C\in\mathcal C$, $D\in\mathcal D$, 函子
	\[
	\operatorname{Hom}_{\mathcal D}(F(C),D) \overset{G}{\longrightarrow} \operatorname{Hom}_{\mathcal C}(GF(C),G(D))
	\overset{\eta_C^*}{\longrightarrow}
	\operatorname{Hom}_{\mathcal C}(C,G(D))
	\]
	为\emph{同伦等价}.
\end{prop}

\subsection{极限与余极限}

\begin{definition}
	{(终对象, 始对象)}
	设 $\mathcal X$ 为 $\infty$-范畴, $x$ 为 $\mathcal X$ 中的对象. 若对任意对象 $y$, $\operatorname{Hom}_{\mathcal X}(y,x)$ 都可缩, 则称 $x$ 为 $\mathcal X$ 的一个\emph{终对象}. 对偶地定义\emph{始对象}.
	
	等价的定义是, $\mathcal X$ 的终对象是 (唯一的) 函子 $\mathcal X\to 1$ 的\emph{右伴随} $1\to\mathcal X$, $\mathcal X$ 的始对象是函子 $\mathcal X\to 1$ 的\emph{左伴随} $1\to\mathcal X$; 这告诉我们终对象和始对象的概念实际上存在于 $\infty$-范畴构成的 \emph{$2$-范畴}中, 见例 \ref{terminal-object-2-categorical}.
\end{definition}

\begin{definition}
	{(俯范畴, 仰范畴)}
	设 $\mathcal X$ 为 $\infty$-范畴, $x$ 为 $\mathcal X$ 中的对象. 定义\emph{俯范畴} $\mathcal X/x$ 为如下的拉回:
	% https://q.uiver.app/#q=WzAsNCxbMCwxLCJcXERlbHRhXjAiXSxbMSwxLCJYIl0sWzEsMCwiXFxtYXRoc2Yge0Z1bn0oXFxEZWx0YV4xLFgpIl0sWzAsMCwiWC94Il0sWzAsMSwieCIsMl0sWzIsMSwiMSJdLFszLDBdLFszLDJdXQ==
	\[\begin{tikzcd}[ampersand replacement=\&,column sep=1em,row sep=2em]
		{{\mathcal X}/x} \& {\mathsf {Fun}(\Delta^1,{\mathcal X})\hspace{-1em}} \\
		{\Delta^0} \& {\mathcal X}
		\arrow["x"', from=2-1, to=2-2]
		\arrow["t", from=1-2, to=2-2]
		\arrow[from=1-1, to=2-1]
		\arrow[from=1-1, to=1-2]
	\end{tikzcd}\]
	其中 $t\colon \mathsf {Fun}(\Delta^1,X)\to X$ 将 ${\mathcal X}$ 的态射对应到其终点.
\end{definition}

由定义, 俯范畴 $\mathcal {\mathcal X}/x$ 的对象是 ${\mathcal X}$ 中以 $x$ 为终点的箭头.
可以说明 $\operatorname{id}_x$ 是 ${\mathcal X}/x$ 的终对象.

\begin{definition}
	{(极限, 余极限)}
	设 $\mathcal C$ 为 $\infty$-范畴, $I$ 为单纯集 (称为\emph{指标集}或\emph{指标范畴}), $F\colon I\to\mathcal C$ 为单纯集映射 (称为 $\mathcal C$ 中的一个\emph{图表}).
	类似于普通范畴中极限的定义, 我们可以构造一个 (广义的) ``俯范畴'' $\mathcal C_{/F}$, 其对象为 $\mathcal C$ 的对象到 $F$ 的锥. 具体地, $\mathcal C_{/F}$ 为如下拉回.
	\[\begin{tikzcd}[ampersand replacement=\&]
		{\mathcal C_{/F}} \& {({\mathcal C}^I)_{/F}} \\
		{\mathcal C} \& {{\mathcal C}^I}
		\arrow["{\text{常值}}"', from=2-1, to=2-2]
		\arrow[from=1-2, to=2-2]
		\arrow[from=1-1, to=2-1]
		\arrow[from=1-1, to=1-2]
	\end{tikzcd}\]
	其中 $\text{``常值''}\colon \mathcal C\simeq\mathcal C^{\Delta^0}\to \mathcal C^I$ 是 $I\to \Delta^0$ 的拉回, 它将 $\mathcal C$ 的对象 $x\colon \Delta^0\to\mathcal C$ 对应到 $x$ 处的 ``常值图表'' $I\to\Delta^0\to\mathcal C$, 就像普通范畴的情形一样.
	定义图 $F$ 的\emph{极限}为 ``俯范畴'' $\mathcal C_{/F}$ 的终对象; 对偶地, 定义 $F$ 的\emph{余极限}为 ``仰范畴'' $\mathcal C_{F/}$ 的始对象.
\end{definition}

\begin{remark}
	{}
	$\infty$-范畴 $\infGpdinfcat$ 中的极限和余极限与拓扑学中的\emph{同伦极限, 同伦余极限}有关.
\end{remark}

\begin{example}
	{}
	对于 $I = \varnothing$, 空图表 $F\colon I\to \mathcal C$ 的极限即是 $\mathcal C$ 的终对象.
\end{example}

\begin{example}
	{(推出, 拉回)}
	考虑 $I = \Lambda_2^2$ (见定义 \ref{simplicial-set-horn} 及其后的插图), 则 $F\colon I\to X$ 等同于两个态射 $x\to z$, $y\to z$. 称 $F$ 的极限为 $\infty$-\emph{拉回}, 简称\emph{拉回}, 记为 $x\times_z y$.
	对偶地, 形如 $\Lambda_0^2$ 的图的余极限称为 $\infty$-\emph{推出}, 简称\emph{推出}.
	
	单纯集映射 % https://q.uiver.app/#q=WzAsMyxbMCwwLCJYIl0sWzEsMCwiWSJdLFswLDEsIloiXSxbMCwyXSxbMCwxXV0=
	$\begin{tikzcd}[ampersand replacement=\&,sep=small]
		X \& Y \\
		Z
		\arrow[from=1-1, to=2-1]
		\arrow[from=1-1, to=1-2]
	\end{tikzcd}$ 的\emph{同伦推出}为 $Y\sqcup_{\{0\}\times X}(\Delta^1\times X)\sqcup_{\{1\}\times X}Z$, 拓扑空间的同伦推出也可类似定义, 不过将 $\Delta^1$ 改为区间 $[0,1]$.
\end{example}

\subsection{$\operatorname{Hom}$ 函子, 预层与 $\infty$-米田引理}

\philoquote{Perhaps the main technical challenge in extending classical categorical results to the $\infty$-categorical context is in merely \emph{defining} the Yoneda embedding.}{Emily Riehl \& Dominic Verity,\\\emph{The Comprehension Construction}}

我们希望定义 $\infty$-版本的预层范畴, 并建立米田引理. 参考注 \ref{Yoneda-embedding-adjoint-Hom}, 我们首先需要一个函子 $\operatorname{Hom}\colon \mathcal C^\op\times\mathcal C\to \infGpdinfcat$. 这个函子同样可借助单纯范畴构造. 注意这并非 $\infty$-范畴理论中陈述米田引理的唯一方法.


\begin{definition}
	{(预层 $\infty$-范畴)}
	设 $\mathcal C$ 为 $\infty$-范畴. 定义 $$\widehat {\mathcal C} := \mathsf {Fun}(\mathcal C^\op,\infGpdinfcat)$$
	为 $\mathcal C$ 上的\emph{预层 $\infty$-范畴}.
\end{definition}

\begin{prop}
	{}
	
\end{prop}

\todo{自由余完备化}

%\section{$\infty$-范畴: 公理化}

\section{$\infty$-\topos{}}

\todo{HTT Ch.6 $\infty$-\topos{}}



\begin{definition}
	{(自反局部化)}
	设 $\mathcal C$ 为 $\infty$-范畴, 定义 $\mathcal C$ 的一个\emph{自反局部化}为函子 $a\colon \mathcal C\to \mathcal D$, 其具有全忠实的右伴随.
	进一步, 若 $a$ 为正合函子 (保持有限极限), 则称之为\emph{正合局部化}.
	这与普通范畴中的自反局部化 (定义 \ref{reflective-subcategory}) 在语法上完全相同.
\end{definition}

如下是 Grothendieck \topos{}的 $\infty$ 版本.

\begin{definition}
	{$\infty$-\topos{}}
	对于 $\infty$-范畴 $\mathcal X$, 若存在 $\infty$-范畴 $\mathcal C$ 以及一个正合局部化
	$$ \widehat {\mathcal C} \to \mathcal X, $$ 则称 $\mathcal X$ 为 \emph{$\infty$-\topos{}}.
\end{definition}

\begin{prop}
	{($\infty$-\topos{}的等价定义, $\infty$-Giraud 公理)}
	$\infty$-\topos{}等价于局部小, 可表现, 余完备, 拉回保持余极限, 且内群胚有效的 $\infty$-范畴.
	% Cisinski CSTT: 局部小, 小余极限, 小可达生成, 拉回保持余极限, 和无交, 内群胚有效.
	% nLab: 可表现, 拉回保持余极限, 和无交, 内群胚有效.
\end{prop}

%\cite{DCCT}


\section{$\infty$-层 $\infty$-\topos{}及其表现}

$\infty$-层是层在 $\infty$-范畴中的类比. 正如层构成\topos{}, $\infty$-层也构成 $\infty$-\topos{}.

% https://ncatlab.org/nlab/show/presentations+of+%28infinity%2C1%29-sheaf+%28infinity%2C1%29-toposes

\subsection{Grothendieck 拓扑与层}

\todo{层, HTT 6.2.2}

\section{上同调}

如下是一种非常广义的上同调概念, 它基本涵盖了所有名为 ``某某上同调'' 的数学概念.

\begin{definition}
	{(上同调)}
	给定 $\infty$-范畴 $\mathcal C$,
	$$
	H^0(X,A)
	$$
	$\operatorname{Hom}(X,A)$ 中的对象 $c\colon X\to A$ 称为\emph{上圈} (cocycle),
	$\operatorname{Hom}(X,A)$ 中的态射 $c_1\to c_2$ 称为\emph{上边界} (coboundary),
	等价类 $[c]\in\pi_0\operatorname{Hom}_{\mathcal C}(X,A)$ 称为\emph{上同调类} (cohomology class).
\end{definition}

\subsection{例}

\begin{example}
	{(奇异上同调, K-理论等)}
	$A = K(\mathbb{Z},n)$
\end{example}

\begin{example}
	{(层上同调)}
	% https://ncatlab.org/nlab/show/cohomology#Overview
	% 表现
\end{example}

\begin{example}
	{(群上同调)}
	% https://ncatlab.org/nlab/show/group+cohomology
	% delooping
\end{example}

\section{$n$-范畴}

\begin{definition}
	[label={local-objects-infinity-category}]
	{(局部对象)}
	设 $\mathcal C$ 为 $\infty$-范畴, $S$ 为 $\mathcal C$ 中一族态射的集合, 称 $\mathcal C$ 的对象 $x$ 为 \emph{$S$-局部对象}是指对 $S$ 中任意态射 $f\colon a\to b$,
	\[
	f^*\colon \operatorname{Hom}_{\mathcal C}(b,x) \to \operatorname{Hom}_{\mathcal C}(a,x)
	\]
	为 $\infty$-群胚的等价.
	这是定义 \ref{local-objects} 在 $\infty$-范畴中的类比.
\end{definition}

\begin{definition}
	{(球面)}
	归纳定义 $\infGpdinfcat$ 的 \emph{$n$ 维球面} $S^n\,(n\geq -1)$ 如下:
	\begin{itemize}
		\item $S^{-1}=\varnothing$;
		\item 对 $n\geq -1$, $S^{n+1}$ 是如图所示的推出.
		% https://q.uiver.app/#q=WzAsNCxbMCwwLCJTXntufSJdLFswLDEsIioiXSxbMSwwLCIqIl0sWzEsMSwiU157bisxfSJdLFswLDFdLFsxLDNdLFswLDJdLFsyLDNdXQ==
	\end{itemize}
	\vspace{-6.5em}
	\begin{flushright}
		$\begin{tikzcd}[ampersand replacement=\&,column sep=small]
			{S^{n}} \& {*} \\
			{*} \& {S^{n+1}}
			\arrow[from=1-1, to=2-1]
			\arrow[from=2-1, to=2-2]
			\arrow[from=1-1, to=1-2]
			\arrow[from=1-2, to=2-2]
		\end{tikzcd}$
	\end{flushright}
\end{definition}

\begin{definition}
	{($n$-群胚)}
	对整数 $n\geq -2$, 定义 \emph{$n$-群胚}为 $\infGpdinfcat$ 中关于 $\{S^{n+1}\to *\}$ 的局部对象 (定义 \ref{local-objects-infinity-category}).
\end{definition}

\begin{example}
	{(低维群胚的例子)}
	在同伦等价的意义下,
	\begin{itemize}
		\item $(-2)$-群胚是一个点.
		\item $(-1)$-群胚是空集或一个点.
		\item $0$-群胚是集合, 也即离散群胚.
	\end{itemize}
\end{example}

\subsection{东西, 结构, 性质}

% https://ncatlab.org/nlab/show/(0,1)-category

% 介绍 (0,1)-topos = Heyting algebra

\todo{}