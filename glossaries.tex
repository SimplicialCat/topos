\chapter*{术语和符号表}


\section*{术语的翻译}

如下是一些尚未广为流传的中文术语 (其中一部分是本人的翻译) 与外文的对应.

\begin{center}
	\begin{tabular}
		{llr}
		中文&外文&定义\\\hline
		\coherent{}逻辑 & coherent logic & \ref{kinds-of-theories}, \ref{inference-rules}\\
		%\cohesive{}\topos{} & cohesive topos \\
		教条 & doctrine & \ref{definition-doctrine}\\
		\fm{} & frame & \ref{frame-definition} \\
		联合满射族 & jointly epimorphic/surjective family & \ref{canonical-topology-on-topos} \\
		位象 & locale & \ref{locale-definition} \\
		\nc{} & nucleus & \ref{nuclei}\\
		\regular{}逻辑 & regular logic & \ref{kinds-of-theories}, \ref{inference-rules} \\
		景 & site & \ref{site-definition}\\
		提纲 & sketch & \ref{sketches} \\
		清晰空间 & sober space & \ref{sober-space} \\
		\topos{} & topos & \ref{topos-definition} \\
		旋子 & torsor & \ref{G-torsors-over-topos}
	\end{tabular}
\end{center}

\section*{符号}

如下是一些可能不通用, 或不广为人知的数学符号.

\begin{center}
	\begin{tabular}
		{ll}
		符号&含义\\ \hline
		$\internalprop{-}$&借用自然语言表达的内语言中的命题\\
		$\interpretation{-}$&逻辑公式或类型论陈述在范畴语义中的解释\\
		$\square$&Lawvere--Tierney 拓扑 (定义 \ref{Lawvere--Tierney-topology-internal-definition})\\
		$\yo$&米田嵌入 (定义 \ref{definition-yoneda-embedding})\\
		$\yo_{(\mathcal C,J)}$& 景到层范畴的米田嵌入 (定义 \ref{sheafified-yoneda})
	\end{tabular}
\end{center}