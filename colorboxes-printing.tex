

% 彩色方框, 参数的使用有点意思
% 这是打印版的参数 (用于省墨)

\usepackage{tcolorbox}

\tcbuselibrary{breakable}

\newcommand{\framecolor}{gray!50!white}

\newtcolorbox[
auto counter,
number within=section,
]{remark}[2][]{
	colback=white,
	colbacktitle=white,
	colframe=\framecolor,
	coltitle=black,
	breakable,
	title={\textsf{注~\thetcbcounter} #2},
	#1
}

\def\examplecolor{white}
\newtcolorbox[
use counter from=remark,
%number within=chapter,
]{example}[2][]{
	colback=white,
	colbacktitle=white,
	colframe=\framecolor,
	coltitle=black,
	breakable,
	title={\textsf{例~\thetcbcounter} #2},
	#1
}
\newtcolorbox[
use counter from=remark,
%number within=chapter,
]{definition}[2][]{
	colback=white,
	colbacktitle=white,
	colframe=\framecolor,
	coltitle=black,
	breakable,
	title={\textsf{定义~\thetcbcounter} #2},
	#1
}

\def\propcolor{white}
\newtcolorbox[
use counter from=remark,
]{prop}[2][]{
	colback=white,
	colbacktitle=white,
	colframe=\framecolor,
	coltitle=black,
	breakable,
	title={\textsf{命题~\thetcbcounter} #2},
	#1
}
\newtcolorbox[
use counter from=remark,
]{propdef}[2][]{
	colback=white,
	colbacktitle=white,
	colframe=\framecolor,
	coltitle=black,
	breakable,
	title={\textsf{命题-定义~\thetcbcounter} #2},
	#1
}

\newtcolorbox[
auto counter,
number within=chapter
]{exercise}[2][]{
	colback=white,
	colbacktitle=white,
	colframe=\framecolor,
	coltitle=black,
	title={\textsf{习题~\alph{\thetcbcounter}} #2},
	#1
}
\newtcolorbox[
use counter from=remark,
]{axiom}[2][]{
	colback=white,
	colbacktitle=white,
	colframe=\framecolor,
	coltitle=black,
	title={\textsf{公理~\thetcbcounter} #2},
	#1
}