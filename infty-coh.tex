\chapter{$\infty$-\topos{}与上同调}

数学中许多名为某某上同调的概念可以在同一个框架下谈论.

\begin{definition}
	{(上同调)}
	给定 $\infty$-范畴 $\mathcal C$ 及其对象 $X,A$, 定义 $X$ 的\emph{取值于 $A$ 的 $0$ 阶上同调}为
	$$
	H^0(X,A):=\pi_0\operatorname{Hom}(X,A).
	$$
	态射 $c\colon X\to A$ 称为\emph{上圈} (cocycle),
	态射的同伦 $c_1\to c_2$ 称为\emph{上边界} (coboundary),
	等价类 $[c]\in\pi_0\operatorname{Hom}_{\mathcal C}(X,A)$ 称为\emph{上同调类} (cohomology class). 通常我们考虑的范畴 $\mathcal C$ 是 $\infty$-\topos{}.
\end{definition}

\subsection{空间的奇异上同调}

%	$A = K(\mathbb{Z},n)$

% $X$ 的 $G$-系数同调是 ``$X$ 那么多个 $G$ 的和'',
% $X$ 的 $G$-系数上同调是 ``$X$ 那么多个 $G$ 的积''.
% tensoring and cotensoring?

% 本小节目标: 以 ∞-范畴语言简述代数拓扑主要结论


\subsection{等变上同调}

% X 的 G-等变上同调即是  X//G 的上同调.

\subsection{群上同调}

%	% https://ncatlab.org/nlab/show/group+cohomology
%	% delooping

\subsection{层上同调}
%\begin{example}
%	{(层上同调)}
%	% https://ncatlab.org/nlab/show/cohomology#Overview
%	% 表现
%\end{example}