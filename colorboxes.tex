

% 彩色方框, 参数的使用有点意思

\usepackage{tcolorbox}

\tcbuselibrary{breakable}

\def\remarkcolor{blue!10!white}
\newtcolorbox[
    auto counter,
    number within=section,
]{remark}[2][]{
    colback=\remarkcolor,
    colframe=\remarkcolor,
    coltitle=black,
    breakable,
    title={\textsf{注~\thetcbcounter} #2},
    #1
}

\def\examplecolor{pink!25!white}
\newtcolorbox[
    use counter from=remark,
    %number within=chapter,
]{example}[2][]{
    colback=\examplecolor,
    colframe=\examplecolor,
    coltitle=black,
    breakable,
    title={\textsf{例~\thetcbcounter} #2},
    #1
}
\newtcolorbox[
    use counter from=remark,
    %number within=chapter,
]{definition}[2][]{
    colback=red!10!white,
    colframe=red!10!white,
    coltitle=black,
    breakable,
    title={\textsf{定义~\thetcbcounter} #2},
    #1
}

\def\propcolor{green!20!white}
\newtcolorbox[
    use counter from=remark,
]{prop}[2][]{
    colback=\propcolor,
    colframe=\propcolor,
    coltitle=black,
    breakable,
    title={\textsf{命题~\thetcbcounter} #2},
    #1
}
\newtcolorbox[
    use counter from=remark,
]{propdef}[2][]{
    colback=orange!20!white,
    colframe=orange!20!white,
    coltitle=black,
    breakable,
    title={\textsf{命题-定义~\thetcbcounter} #2},
    #1
}

\newtcolorbox[
    auto counter,
    number within=chapter
]{exercise}[2][]{
    colback=gray!10!white,
    colframe=gray!10!white,
    coltitle=black,
    title={\textsf{习题~\alph{\thetcbcounter}} #2},
    #1
}
\newtcolorbox[
use counter from=remark,
]{axiom}[2][]{
	colback=cyan!30!white,
	colframe=cyan!30!white,
	coltitle=black,
	title={\textsf{公理~\thetcbcounter} #2},
	#1
}