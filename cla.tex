\chapter{句法景与分类\topos{}}

\label{cla}

\minitoc

\section{句法范畴: 语法--语义对偶}

\todo{两种不同的 syntactic category}

%\todo{写成单独的一章?}

\philoquote{The importance of syntactic categories lies in the fact that they allow us to associate with a theory (in the sense of axiomatic presentation), which is a `linguistic',
	unstructured kind of entity, a well-structured mathematical object whose `geometry' embodies the syntactic aspects of the theory.}{Olivia Caramello, \cite{TST}}

%  A most notable fact is that the models of the theory can be recovered as functors defined on its syntactic category which respect the ‘logical’ structure on it.

逻辑学上有一个一般性的现象, 即理论 (一阶逻辑, 类型论等等) 与范畴之间的对应. 某些范畴可以为理论提供语义, 某些理论可以作为范畴的语法.
%
\todo{}

我们将看到语法--语义对偶也是一种代数--几何对偶\footnote{或许也可以反过来说代数--几何对偶不过是一种语法--语义对偶.}, 语法对应代数, 语义对应几何.

\subsection{类型论的语境范畴}

附录 \ref{appendix-type-theory} 节简单介绍了类型论.

\begin{definition}
	{}
	对于一种类型论 $T$, 其\emph{语境范畴} (category of contexts) $\operatorname{Con}(T)$ 定义如下.
	\begin{itemize}
		\item $\operatorname{Con}(T)$ 的\emph{对象}为类型论 $T$ 的\emph{语境};
		\item $\operatorname{Con}(T)$ 中的\emph{态射}为语境之间的\emph{代换}.
	\end{itemize}
\end{definition}



\subsection{一阶理论的句法范畴}

\begin{definition}
	[label={first-order-theory-syntactic-category}]
	{(一阶理论的句法范畴)}
	
	设 $\mathbb T$ 是符号表 $\Sigma$ 上的一阶理论. 定义\emph{句法范畴} (syntactic category) $\mathcal C_{\mathbb T}$, 其中
	\begin{itemize}
		\item \emph{对象}是 $\Sigma$ 上带有语境的公式 $(\vec x , \phi)$ 的 $\alpha$-等价类 ($\alpha$ 等价是指两个公式仅有变量名的差异);
		\item 对象 $(\vec x , \phi)$ 到 $(\vec y , \psi)$ 的\emph{态射} (其中不妨设语境 $\vec x , \vec y$ 不交) 是公式 $\theta(\vec x,\vec y)$ 的 $\mathbb T$-可证等价类, 满足 $\mathbb T$-可证的\emph{函数性}, 即如下相继式 $\mathbb T$-可证.
		\begin{itemize}
			\item $\phi\vdash_{\vec x} \exists \vec y \,\theta$, 含义是 ``对给定的 $\vec x$, 存在 $\vec y$ 满足 $\theta(\vec x,\vec y)$'';
			\item $\theta\vdash_{\vec x,\vec y} \phi \wedge \psi$,
			\item $\theta \wedge \theta[\vec z/\vec y] \vdash_{\vec x,\vec y,\vec z} \vec y = \vec z$, 这里 $\theta[\vec z / \vec y]$ 是指将 $\theta$ 中的 $\vec y$ 替换为 $\vec z$, 公式的含义是 ``对给定的 $\vec x$, 至多存在一个 $\vec y$ 满足 $\theta(\vec x,\vec y)$''.
		\end{itemize}
	\end{itemize}
\end{definition}

\begin{remark}
	{}
	句法范畴的态射又称\emph{变量代换}.
\end{remark}

\begin{remark}
	{}
	由定义, 一个理论的句法范畴依赖于理论所属的类别 (代数, 正则, 凝聚, $\cdots$). 一个代数理论也可以视为凝聚理论, 但一个代数理论的句法范畴与它作为凝聚理论的句法范畴不同.
\end{remark}

\subsection{代数理论的句法范畴--Lawvere 理论}

\todo{万有代数}

\begin{prop}
	{}
	对于代数理论 $\mathbb T$,
\end{prop}

\begin{example}
	[label={Lawvere-theory-CartSp}]
	{}
	$\mathsf {CartSp}$
\end{example}

\section{分类\topos{}}

\philoquote{
    The notion of classifying topos really formalizes the notion of ``content'' of a mathematical theory.
    If you discover that two theories have the same classifying topos, this means that the two theories tell the same story in different languages.
}{Olivia Caramello}

%分类\topos{}的角色类似于拓扑学中的分类空间:
有代数拓扑背景的读者知道, 对于拓扑群 $G$ 存在 ``分类空间'' $BG$,
使得 (足够好的) 空间 $X$ 上 $G$-主丛的等价类
一一对应于 $X$ 到 $BG$ 的映射同伦类 (被 $BG$ ``分类''),
$$
G\mathsf{-Bund} (X) \simeq [X,BG];
$$
从而恒等映射 $\operatorname{id}_{BG}$ 对应着 $BG$ 上一个 ``万有'' 的 $G$-主丛 $EG \to BG$,
``万有'' 意指任何 $G$-主丛都是通过某个映射 $X \to BG$ 将其拉回得到.
分类\topos{}不仅仅是分类空间的一个精神上的类比; 它某种意义上是分类空间的推广, 这种推广正是沿着 Grothendieck 的 ``\topos{}作为空间概念的推广'' 的思路.

类似地, 在许多情形下, \topos{}上的一种结构 $T$ 可由它到一个特殊的\topos{}的态射来分类, 这就是\emph{分类\topos{}}的概念;
某种结构 $T$ 的分类\topos{}应当视为 ``所有 $T$ 构成的空间'', 术语叫做\emph{模空间}.
若将\topos{} $\mathsf C$ 视为空间, 那么 $\mathsf C$ 上的一个结构 $T$ 可视为空间的 ``每一点'' 上都有一个结构 $T$,
从而这给出 $\mathsf C$ 到模空间的一个几何态射.
例如, 设 $\mathsf E$ 为环的分类\topos{}, 那么 $\mathsf E$ 应当视为 ``所有环构成的空间'' (上的层范畴), 拓扑空间 $X$ 上的一个环层即是 $X$ 的 ``每一点'' 上都有一个环,
这等同于 $\text{Sh}(X)$ 到 $\mathsf E$ 的一个几何态射.

分类\topos{}到自身的恒等态射对应着其上的 ``万有'' 的 $T$ 结构, 又称\emph{重言} (tautological) $T$ 结构.

\subsection{$G$-旋子的分类\topos{}}


下面的概念是 ``空间上的 $G$-主丛'' 的推广, 取自 \cite{SGL} VIII.2 节.
\begin{definition}
	[label={G-torsors-over-topos}]
	{(\topos{}上的 $G$-旋子)}
	设 $G$ 为离散群, 即 $\mathsf {Set}$ 中的群.
	设 $\gamma\colon \mathsf C\to \mathsf {Set}$ 是 $\mathsf {Set}$ 上的\topos{} (见命题 \ref{global-sections-geometric-morphism}). 定义 $\mathsf C$ 上的一个 $G$-\emph{旋子}\footnotemark (torsor) 为 $\mathsf C$ 的对象 $T$, 以及群对象 $\gamma^*(G)$ 在 $T$ 上的作用 $\mu\colon \gamma^*(G)\times T \to T$, 满足
	\begin{enumerate}[(i)]
		\item 映射 $T\to 1$ 为满射;
		\item 群作用 $\mu$ 诱导同构 $(\mu,\pi_2)\colon \gamma^*(G)\times T \to T\times T$.
	\end{enumerate}
\end{definition}
\footnotetext{Torsor 似乎没有通行的中文译名, 这可能是因为它在拓扑学中一般被称作\emph{主丛}. 这里我跟随我的一位老师译为\emph{旋子}.}

%\begin{remark}
%	{(条件 ii 的解读)}
%	群作用 $\mu$ 诱导同构 $(\mu,\pi_2)\colon \gamma^*(G)\times T \to T\times T$ 等价于 $\gamma^*(G)$ 在 $T$ 上的作用是自由且传递的.
%	
%\end{remark}

\begin{example}
	{(集合范畴中的 $G$-旋子)}
	由于集合范畴 $\mathsf {Set}$ 是一个点上的层范畴,
	$\mathsf {Set}$ 中的 $G$-旋子即是 ``一个点上的 $G$-主丛'',
	即一个非空集合 $T$, 带有 $G$-左作用 $\mu\colon G\times T \to T$, 满足 $(\mu,\pi_2)\colon G\times T \to T\times T$ 为双射.
	后一个条件等价于这个作用是自由且传递的:
	``自由'' 等价于 $(\mu,\pi_2)\colon G\times T \to T\times T$ 为单射, ``传递'' 等价于其为满射.
	
	有趣的是, 两个映射 $\mu,\pi_2\colon G\times T \to T$ 也可视为一个范畴 (具体地, $G$ 在 $T$ 上作用的\emph{作用群胚}) 的箭头集合到对象集合的两个映射, 分别将一个箭头映射到其终点与起点. 那么 $(\mu,\pi_2)\colon G\times T \to T\times T$ 为双射就是说, 作用群胚的任何两个对象 $x,y$ 之间有且仅有一个态射 $x\to y$.
\end{example}

\begin{example}
	{(空间上的 $G$-主丛)}
	仍设 $G$ 为离散群. 拓扑空间 $X$ 上的 $G$-主丛等价于 (或可定义为) 平展映射 $E \to X$, 带有 $X$ 上的 $G$-左作用 $G\times E \to E$, 使得每个纤维 $E_x$ 非空且带有 $G$ 的自由传递作用.
	
	可以证明, 这个定义等价于层\topos{} $\operatorname{Sh}(X)$ 上 $G$-旋子的定义.
\end{example}

\begin{example}
	{($G\mathsf {Set}$ 上的万有 $G$-旋子)}
	对于离散群 $G$, 考虑其自身作成的 $G$-右作用集合 $R_G$. 我们断言它是 $G\mathsf {Set}$ 上的 $G$-旋子.
	
	首先回忆几何态射 $\gamma\colon G\mathsf {Set} \to \mathsf {Set}$ (例 \ref{group-homomorphism-adjoint-triple-example-G-1}),
	其逆像函子 $\gamma^*\colon \mathsf {Set} \to G\mathsf {Set}$
	将集合对应到其自身, 带有平凡 $G$-右作用.
	群作用 $\mu\colon \gamma^*(G)\times R_G \to R_G$, $(g,h)\mapsto gh$ 是 $G$-集合的态射, 因为 $\mu(g,hk)=ghk=\mu(g,h)k$.
	映射 $(\mu,\pi_2)\colon \gamma^*(G)\times R_G\to R_G\times R_G$ 为 $(g,h)\mapsto (gh,h)$, 从而为 $G$-集合的同构.
	%\todo{写完万有 $G$-旋子}
\end{example}

\topos{}上的旋子是 ``旋子的理论'' 的模型.

\begin{definition}
	{($G$-旋子的理论)}
	设 $G$ 为离散群, 定义 $G$-旋子的理论 $\mathbb T_G$.
	其中只有一个类型 $T$, 对每个元素 $g\in G$ 都有一个一元函数符号 $g$, 公理如下.
	\begin{itemize}
		\item (\emph{群作用}) 对每一对 $g,h\in G$ 有一条公理 $\vdash_x g(hx)=(gh)x$ (注意这里没有用到任意量词 $\forall$);
		\item (\emph{非空}) $\vdash \exists x. \top$;
		\item (\emph{自由性}) 对每一对不同的 $g,h\in G$ 有一条公理 $(gx=hx)\vdash_x \bot$ (同上, 这里没有用到 $\forall$);
		\item (\emph{传递性}) $\displaystyle\vdash_{(x,y)}\bigvee_{g\in G}gx=y$.
	\end{itemize}
\end{definition}

理论 $\mathbb T_G$ 是一种几何理论 (定义 \ref{kinds-of-theories}), 因为它用到了 ``真'' $\top$, 存在量词 $\exists$, ``假'' $\bot$ 和无穷析取 $\bigvee$.

\subsection{子终对象的分类\topos}

考虑范畴 $\mathsf 2 = \{\bullet\longrightarrow\bullet\}$. 例 \ref{varying-set-topos} 介绍的 ``变集范畴'' $\mathsf {Fun}(\mathsf {2},\mathsf {Set})\simeq\widehat {\mathsf {2}}$ 还有一个特殊的名字叫 \emph{Sierpi\'nski \topos{}}, 因为它与 Sierpi\'nski 空间有关.

\begin{propdef}
	[label={Sierpinski-space}]
	{(Sierpi\'nski 空间)}
	开集函子 $\operatorname{Open}\colon \mathsf {Top}\to \mathsf {Set}$ 是可表函子, 其表示对象称为 \emph{Sierpi\'nski 空间}.
\end{propdef}

\todo{}

\subsection{群的分类\topos{}}

本节构造群的分类\topos{}. 群是一种 Lawvere 理论 (附录 \ref{universal-algebra} 节); 下面的论述适用于一般的 Lawvere 理论, 从而逐字逐句地替换可得到环, Boole 代数等结构的分类\topos{}.
%
% 回忆范畴中的群对象的概念. 设范畴 $\mathsf C$ 有有限积 (包括始对象 $1$), 那么 $\mathsf C$ 中的\emph{群}为对象 $G$, 带有元素 $e \colon 1 \to G$, 乘法 $\cdot \colon G\times G \to G$, 满足群的交换图.

%满足通常的交换环的图表, 例如分配律 $x(y+z)=xy+xz$ 对应交换图
%% https://q.uiver.app/#q=WzAsNCxbMCwwLCJSXFx0aW1lcyBSXFx0aW1lcyBSIl0sWzEsMCwiUlxcdGltZXMgUiJdLFsxLDEsIlIiXSxbMCwxLCJSXFx0aW1lcyBSIl0sWzAsMSwiKHgseSx6KVxcbWFwc3RvICh4LHkreikiXSxbMSwyLCJcXHRpbWVzIl0sWzAsMywiKHgseSx6KVxcbWFwc3RvICh4eSx4eikiLDJdLFszLDIsIisiLDJdXQ==
%\[\begin{tikzcd}[ampersand replacement=\&]
%	{R\times R\times R} \& {R\times R} \\
%	{R\times R} \& R.
%	\arrow["{(x,y+z)}", from=1-1, to=1-2]
%	\arrow["\times", from=1-2, to=2-2]
%	\arrow["{(xy,xz)}"', from=1-1, to=2-1]
%	\arrow["{+}"', from=2-1, to=2-2]
%\end{tikzcd}\]
%(这里使用了 \ref{Mitchell--Benabou-language} 节介绍的 Mitchell--B\'enabou 语言.)

设范畴 $\mathsf C$ 有有限积. $\mathsf C$ 中的群对象构成一个范畴 $\mathsf {Grp}(\mathsf C)$.
\ref{universal-algebra} 节提到它等同于群的 Lawvere 理论 $\mathbb T_{\text{Grp}}$ 到 $\mathsf C$ 的保持有限积的函子的范畴.
进一步, 对于两个具有有限积的范畴 $\mathsf C,\mathsf D$, 以及保持有限积的函子 $f \colon \mathsf C \to \mathsf D$, 有对应的函子
$\mathsf {Grp}(\mathsf C) \to \mathsf {Grp}(\mathsf D)$.% $\mathsf {Ring}(\mathsf C) \to \mathsf {Ring}(\mathsf D)$.

对于\topos{}间的几何态射 $f \colon \mathsf C \to \mathsf D$, 其逆像部分 (见定义 \ref{geometric-morphism}) $f^* \colon \mathsf D \to \mathsf C$ 保持有限极限, 故保持有限积, 从而诱导了函子 $\mathsf {Grp}(\mathsf D) \to \mathsf {Grp}(\mathsf C)$;
这表示 $\mathsf {Grp}(-)$ 关于\topos{}是 ``反变'' 的.

下面我们将构造一个\topos{} $\mathsf E$, 称为\emph{群的分类\topos{}}, 使得有自然的范畴等价
$$
\mathsf {Grp}(\mathsf C) \simeq \mathsf{Hom}(\mathsf C,\mathsf E).
$$

\subsection{环的分类\topos{}}

\todo{单独讲一下和代数几何的关系}

% 参考 Caramello 的 TST

%\paragraph{群}
%
%考虑范畴 $\mathsf A = (\mathsf {Grp}_{\text{f}})^\op$,
%
%\paragraph{环}
%
%考虑范畴 $\mathsf A = (\mathsf {Ring}_{\text{fp}})^\op$,
%其中 $\mathsf {Ring}_{\text{fp}}$ 是 ($\mathsf{Set}$ 中) \emph{有限表现} (finitely presented) 环的范畴 (见例 \ref{zariski-site}).
%$\mathsf A$ 中有一个特殊的对象 $A = \mathbb{Z}[x]$,
%也即仿射直线.

\subsection{几何理论的分类\topos{}}

\begin{definition}
    {(几何理论的句法景和分类\topos{})}
    设 $\mathbb T$ 为一几何理论.
    
    $\text{Sh}(\mathcal C_{\mathbb T},)$
\end{definition}

\todo{以 torsor 的句法景举例}