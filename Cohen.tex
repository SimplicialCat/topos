\section{Cohen 力迫法}

\label{Cohen-forcing}

1874 年, Georg Cantor 证明了自然数与实数 (又称连续统) 之间不存在一一对应\footnote{不过他的第一个证明并非现在流行的对角线论证.}. Cantor 接着于 1878 年提出了\emph{连续统假设} (continuum hypothesis),
$$
\fbox{在自然数集合 $N$ 与连续统 $PN$ 之间不存在其它的基数.}
$$

1940 年, Kurt G\"odel 证明连续统假设与 Zermelo--Fraenkel 集合论相容. 1963 年, Paul Cohen 证明了连续统假设独立于带有选择公理的 Zermelo--Fraenkel 集合论 (ZFC), 即 ZFC 既不能证明, 也不能证伪连续统假设.

% Boole 意象的内容放在了第一章末尾.

在 \ref{Boolean-topos} 节我们介绍了 Boole \topos{}. 我们可以在这样的\topos{}中做 ``经典数学''. 下面的内容本质上等同于 Cohen 证明连续统假设独立于 ZFC 所使用的方法, 只不过翻译到了\topos{}的语境.

\begin{prop}
	{}
	存在一个 Boole \topos{}, 其中选择公理成立, 而连续统假设不成立.
\end{prop}

\subsubsection{基础知识}

回忆任何\topos{}上都有一个 Lawvere--Tierney 拓扑 $\neg\neg$. 有趣的是, 它总是给出一个 Boole 意象.

\begin{prop}
	{}
	对任意\topos{} $\mathcal C$, $\operatorname{Sh}_{\neg\neg}\mathcal C$ 为 Boole \topos{}.
\end{prop}
\begin{proof}
	由 Boole \topos{}的内语言刻画 (\ref{internal-Boolean-topos}), 我们要在 $\operatorname{Sh}_{\neg\neg}\mathcal C$ 中证明 $\forall p\in\Omega (p\lor \neg p)$.
	% 需要几何态射在命题上的作用
\end{proof}

\begin{definition}
	{(基数的比较)}
	对于一个\topos{}中的两个对象 $X,Y$, 若存在单射 $X\to Y$, 且 $\operatorname{Epi}(X,Y)\simeq 0$, 则称 $X$ 的基数小于 $Y$, 记为 $X<Y$.
\end{definition}

回忆 $\operatorname{Epi}(X,Y)$ 的定义 (\ref{set-of-epimorphisms}), $\operatorname{Epi}(X,Y)\simeq 0$ 当且仅当公式 $\forall f\in Y^X\,\neg(\operatorname{im}f = Y)$ 成立.

