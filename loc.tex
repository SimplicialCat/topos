\chapter{位象: 无点拓扑学}

%\todo{变成单独的一章?}

常常拓扑空间的重点不在于\emph{点}, 而在于开集, 以及开集之间的关系.
将开集的性质提炼出来, 使其不再依赖于点, 就成为\emph{位象} (locales) 的概念. 它是介于拓扑空间和景之间的一个推广.\footnote{说位象是拓扑空间的 ``推广'' 不甚准确, 因为一般拓扑空间到位象的对应不是全忠实的 (有的空间开集太少). 但是 ``好'' 的空间范畴如 Hausdorff 空间范畴确实嵌入位象的范畴.}
位象理论又称为\emph{无点拓扑学} (pointless topology\footnote{双关笑话: 无点拓扑学不是 pointless (无用的).})

Andr\'e Joyal \cite{Joyal-Crash-Course} 建议在\topos{}理论之前先学习位象理论.

在高阶的观点中, 位象又可称作 $0$-\topos{}.

\newcommand{\fm}{位格}

\section{基本概念}

\begin{definition}
	{(\fm{})}
	\emph{\fm{}} (frame, 又称 local lattice)\footnotemark 是满足如下条件的偏序集:
	\begin{itemize}
		\item 存在有限交 $\wedge$ 与任意并 $\bigvee$, 其中一族元素的\emph{交} (meet) 是指同时小于等于这些元素的最大元, \emph{并} (join) 是指同时大于等于这些元素的最小元;
		\item 有结合律
		$$
		a \wedge \bigvee_{i \in I} b_i = \bigvee_{i \in I} (a \wedge b_i),
		$$
		其中 $I$ 是任意集合.
	\end{itemize}
	
	\fm{}的态射是偏序集之间保持有限交与任意并的态射. \fm{}的范畴记为 $\mathsf {Frm}$.
\end{definition}
\footnotetext{Frame 一词似乎没有通行的中文翻译. 这里试译为\fm{}, 因为它是一种与拓扑相关的格 (lattice).}
\begin{remark}
	{}
	由于任意交可由任意并表示,
	$$
	\bigwedge A = \bigvee \big\{ b \colon b\leq a \ \forall a\in A\big\},
	$$
	可以证明\fm{}中任意交也存在, 即\fm{}是\emph{完备格} (complete lattice). 但由定义, \fm{}的态射不一定保持任意交.
\end{remark}

\begin{definition}
	[label={locale-definition}]
	{(位象)}
	\emph{位象} (locale) 是\fm{}的形式对偶,
	即我们定义位象的范畴 $\mathsf {Loc}$ 是位格范畴 $\mathsf {Frm}$ 的对偶范畴.
	
	对于位象 $X$, 我们记对应的\fm{}为 $\mathcal O(X)$, 称其中的元素为 $X$ 的\emph{开子集}或\emph{开子空间}; 对于位象的态射 $f \colon X \to Y$,
	记对应的\fm{}的态射为 $f^* \colon \mathcal O(Y) \to \mathcal O(X)$.
\end{definition}

\begin{remark}
	{}
	\fm{}与位象的对偶类似于环与仿射概形的对偶,
	是代数--几何对偶的一例.
	%位象的定义中引入的记号都是为了体现这种对偶.
\end{remark}

\begin{example}
	{(拓扑空间作为位象)}
	拓扑空间 $X$ 的开集范畴 $\operatorname{Open}(X)$ 构成一个\fm{}.
	对于连续映射 $f \colon X \to Y$,
	我们有反向的\fm{}态射 $f^* \colon \operatorname{Open}(Y) \to \operatorname{Open}(X)$,
	将 $Y$ 的开集 $U$ 映射到 $X$ 的开集 $f^{-1}(U)$, 保持有限交与任意并.
	这构成了函子 $\operatorname{Open} \colon \mathsf {Top} \to \mathsf {Frm}^{\op}$; 由定义 $\mathsf {Loc} = \mathsf {Frm}^{\op}$, 我们得到函子 $$\operatorname{Open} \colon \mathsf {Top} \to \mathsf {Loc}.$$
\end{example}

\begin{example}
	{(环的谱作为位象)}
	在代数几何中, (交换) 环的谱 $\operatorname{Spec}A$ 是由 $A$ 的素理想的集合赋予 Zariski 拓扑定义的. 然而, 我们也可以直接刻画谱的 ``基础开集'' $U_f$ 生成的\fm, 从而不需要使用环的素理想\footnotemark:
	$$
	\mathcal O(\operatorname{Spec}A):= \big\langle U_f\colon f\in A\mid U_0 = \bot, U_1 = \top,
	U_{f+g} \leq U_f \vee U_g, U_{fg} = U_f \wedge U_g \big\rangle.
	$$
	(\fm{}就像群, 环等代数结构一样, 可使用生成元和关系刻画, 并且可由给定生成元的 ``自由代数'' 的商得到.)
	
	将 $U_f$ 想象为一个空间上函数 $f$ ``非零'' 的地方,
	$f+g$ 非零的地方包含于 $f$ 与 $g$ 各自非零的地方的并,
	而 $fg$ 非零的地方恰为两者的交.
	可以证明如此定义的位象 $\operatorname{Spec}A$ 正是传统上定义的拓扑空间 $\operatorname{Spec}A$.
\end{example}
\footnotetext{Jacob Lurie 在 \emph{Derived Algebraic Geometry V: Structured Spaces} 第 37 页写道, ``It is in some sense coincidental that $\operatorname{Spec} A$ is described by a topological space. What arises more canonically is the lattice of open subsets of $\operatorname{Spec} A$, which is generated by basic open sets of the form $U_f$. This lattice naturally forms a \emph{locale} ...'' \par
	值得一提的是, 构造主义数学青睐这种位象的观点, 因为环的素理想的存在性依赖选择公理. 关于构造主义数学与位象, 我们推荐读者观看 Andrej Bauer 的讲座. \url{https://www.youtube.com/watch?v=21qPOReu4FI}}

%\begin{example}
%	[label={complete-boolean-algebras}]
%	{(完备 Boole 代数)}
%	Boole 代数是满足如下条件的
%\end{example}

\begin{example}
	{(点作为位象)}
	单点空间 $\text{pt}$ 对应的位象记作 $1 = \operatorname{Open}(\text{pt})$.
	它对应的位格是两个元素的偏序集 $\mathcal O(1) = \{\bot,\top\}$.
	位象 $1$ 也称为\emph{终位象} (terminal locale).
\end{example}

% 下面的定义表明从位象中可以还原出 ``点'' 的概念.

\begin{definition}
	[label={points-of-locale}]
	{(位象的点)}
	定义位象 $X$ 的\emph{点}为位象的态射 $p\colon 1 \to X$.
	记位象 $X$ 的点的集合为 $\operatorname{Pt}(X)$.
\end{definition}

在\fm{}的层面, 位象 $X$ 的一个点 $p\colon 1 \to X$ 是保持有限交与任意并的态射 $p^*\colon \mathcal O(X) \to \{\bot,\top\}$.
考虑 $\top$ 的原像, 我们得到 $\mathcal O(X)$ 的一个\emph{完全素滤子} (completely prime filter). (偏序集中的\emph{滤子}是关于交封闭的向上封闭子集, 称 $F$ 为完全素滤子是指当 $\bigvee_i U_i \in F$ 时, 至少有一个 $U_i$ 属于 $F$.)

\begin{remark}
	{}
	滤子是理想的 ``对偶''. 我们有如下类比: 态射 $\mathcal O(X)\to \{\bot,\top\}$ 中 $\top$ 的原像是完全素滤子, 类似于环同态 $A\to k$ 的核是素理想 ($k$ 为整环). 位象 $X$ 的点对应 $\mathcal O(X)$ 的完全素滤子, 类似于谱 $\operatorname{Spec}A$ 的点对应环 $A$ 的素理想.
\end{remark}

对每个元素 $U\in\mathcal O(X)$, 规定 $\operatorname{Pt}(X)$ 的子集
$\{p \in \operatorname{Pt}(X) \colon p^*(U) = \top\}$ 为开集,
我们得到了 $\operatorname{Pt}(X)$ 上的一个拓扑. 于是有函子
$$\operatorname{Pt} \colon \mathsf {Loc} \to \mathsf {Top}.$$

\begin{prop}
	[label={top-loc-adjunction}]
	{}
	拓扑空间与位象之间有伴随
	% https://q.uiver.app/#q=WzAsMixbMCwwLCJcXG1hdGhzZiB7VG9wfSJdLFsxLDAsIlxcbWF0aHNmIHtMb2N9Il0sWzAsMSwiXFxvcGVyYXRvcm5hbWV7T3Blbn0iLDAseyJvZmZzZXQiOi0yfV0sWzEsMCwiXFxvcGVyYXRvcm5hbWV7UHR9IiwwLHsib2Zmc2V0IjotMn1dLFsyLDMsIiIsMCx7ImxldmVsIjoxLCJzdHlsZSI6eyJuYW1lIjoiYWRqdW5jdGlvbiJ9fV1d
	\[\begin{tikzcd}[ampersand replacement=\&]
		{\mathsf {Top}} \& {\mathsf {Loc}.}
		\arrow[""{name=0, anchor=center, inner sep=0}, "{\operatorname{Open}}", shift left=2, from=1-1, to=1-2]
		\arrow[""{name=1, anchor=center, inner sep=0}, "{\operatorname{Pt}}", shift left=2, from=1-2, to=1-1]
		\arrow["\dashv"{anchor=center, rotate=-90}, draw=none, from=0, to=1]
	\end{tikzcd}\]
\end{prop}

上述伴随远远不是范畴等价; 首先, 从拓扑空间对应的位象中不一定能重构出原来的拓扑空间, 例如所有平凡拓扑对应的位象都是终位象 $1$.

然而, 对于分离性较好的空间, 其对应的位象确实能重构出这个空间:
\begin{definition}
	{(清晰空间)}
	设 $X$ 为拓扑空间. 若命题 \ref{top-loc-adjunction} 中伴随的单位 $X \to \operatorname{Pt}\operatorname{Open}(X)$ 是同胚,
	则称 $X$ 为\emph{清晰空间} (sober space\footnotemark). 换言之, 清晰空间的范畴是上述伴随给出的等价的满子范畴 (命题 \ref{adjoint-full-subcategory-equivalence}).
\end{definition}
\footnotetext{Sober 的原义是清醒, 未喝醉. 直观上一个 sober 空间中的点没有过于 ``糊在一起'', 清晰可辨.}

\begin{prop}
	{}
	Hausdorff ($T_2$) 空间都是清晰空间, 清晰空间都是 $T_0$ 空间; 而清晰与 $T_1$ 互不蕴涵, 例如 Sierpi\'nski 空间清晰而不 $T_1$, 无限集上的余有限拓扑 $T_1$ 而不清晰.
\end{prop}

另一方面, 从一个位象的所有点的信息也无法重构出这个位象: 下面的例子表明一个位象甚至可能没有点!

\begin{example}
	{(没有点的位象的例子: 完备无原子 Boole 代数)}
	
	\emph{Boole 代数}是有二元交, 二元并以及补运算 (满足适当条件) 的偏序集; \emph{完备 Boole 代数}是指存在任意交和任意并的 Boole 代数.
	由定义, 完备 Boole 代数是\fm.
	称偏序集中的一个元素为\emph{原子}, 是指除 $\bot$ 以外没有比它更小的元素.
	对任何完备 Boole 代数 $B$, 设 $P$ 是 $B$ 的完全素滤子,
	则 $\bigwedge P$ 是 $B$ 的原子. 这是因为, 假设 $x < \bigwedge P$,
	则 $x\notin P$, 从而由 $\top=x\vee\neg x\in P$, 得 $\neg x \in P$. 这说明 $x<\neg x$, 故 $x=\bot$.
	因此, 完备无原子 Boole 代数对应没有点的位象.
	下面是两个完备无原子 Boole 代数的例子.
	
	\begin{itemize}
		\item 设 $(X,\mathcal A,\mu)$ 为 $\sigma$-有限测度空间, $N\subset \mathcal A$ 为零测集的理想,
		那么商代数 $\mathcal A / N$ 是完备 Boole 代数; 并且当 $\mu$ 无原子 (没有正测度的点) 时, $\mathcal A/N$ 无原子.
		\item	
	设 $X$ 为拓扑空间. 对于开集 $U\subset X$, 定义 $\neg U$ 是 $U$ 的补集的内部.
	考虑偏序集
	$$
	\text{RO}(X)
	=
	\big\{U\in\operatorname{Open}(X)\mid U = \neg\neg U\big\},
	%P\mathbb{N}\big/ \sim,\quadA\sim B \Leftrightarrow A\setminus B, B\setminus A\text{为有限集}.
	%$$% 是 $\mathbb{N}$ 的子集在相差有限个元素的等价关系下构成的偏序集.
	$$
	称为 $X$ 的\emph{正则开集代数} (regular open algebra). 可以证明 $\text{RO}(X)$ 也是一个完备 Boole 代数\footnotemark. 例如其中的任意并由下式给出:
	$$
	\bigvee_{i\in I}U_i = \neg\neg\Big(\bigcup_{i\in I}U_i\Big).
	$$
	
	若取空间 $X$ 为欧氏空间, 则 $\text{RO}(X)$ 没有原子 (任何正则开集内都有一个更小的正则开集).
	\end{itemize}
\end{example}
\footnotetext{\url{https://planetmath.org/regularopenalgebra}}

\subsection{子位象}

正如许多 ``空间'' 的概念 (拓扑空间, 向量空间, 概形 ...) 一样, 位象有一种自然的 ``子空间'' 的概念; 当然, 子位象不是通过点集, 而是通过开集的代数定义的.

\begin{definition}
	{(子位象)}
	设 $X$ 为位象. 定义 $X$ 的\emph{子位象} (sublocale) $Y$ 如下: $\mathcal O(Y)$ 为 $\mathcal O(X)$ 在一个保持交与任意并运算的等价关系 (即 $\mathcal O(X)\times\mathcal O(X)$ 的子\fm{}) $\equiv_Y$ 下的\emph{商}. 直观上, 这个等价关系是说两者与子空间的 ``交'' 相同.
	
	对于两个子位象 $Y,Z$, 若 $U\equiv_Z V \Rightarrow U\equiv_Y V$, 则称 $Y$ \emph{包含于} $Z$.
\end{definition}

子位象是\fm{}范畴 $\mathsf {Frm}$ 中的\emph{正则满射}, 即位象范畴 $\mathsf {Loc}$ 中的\emph{正则单射} (定义 \ref{regular-epi-and-mono}).

对于位象 $X$, $\mathcal O(X)$ 的元素可视为其开子空间.

\begin{definition}
	{(开子位象)}
	设 $X$ 为位象, $W\in\mathcal O(X)$. 定义 $X$ 的\emph{开子位象} $W$:
	$$U \equiv_W V \quad \text{当且仅当} \quad U\cap W = V \cap W.$$
\end{definition}



\subsection{测度与随机性}

本小节介绍 Alex Simpson 的文章 \cite{SIMPSON20121642} 引入的\emph{随机序列的位象}. 它是 $0$ 和 $1$ 组成的所有可数序列的空间的 ``子空间'', 但没有任何一个序列属于这个子空间, 因为 ``一个确定的序列永远不是随机序列''.

首先引入位象上的测度; 这是测度空间的一个推广. 设 $X$ 为位象, 其上的\emph{测度}是满足如下条件的函数 $\mu\colon \mathcal O(X)\to [0,+\infty]$:

\begin{definition}
	{(位象上的测度)}
	\begin{itemize}
		\item $\mu(\bot) = 0$,
		\item $x\leq y \Rightarrow \mu(x)\leq \mu(y)$,
		\item $\mu(x)+\mu(y)=\mu(x\vee y)+\mu(x\wedge y)$,
		\item 对任意 (可数) 递增序列 $x_1\leq x_2\leq\cdots$, $\mu(\bigvee_i x_i)=\sup_{i}\mu(x_i)$.
	\end{itemize}
	若 $\mu(\top)=1$, 称 $\mu$ 为\emph{概率测度}.
\end{definition}

\begin{propdef}
	{(随机元素的子位象)}
	设位象 $X$ 上有测度 $\mu$. 定义 $X$ 的 \emph{$\mu$-随机元素的子位象} $\operatorname{Ran}_\mu(X)$ 如下:
	$$
	U\equiv_{\operatorname{Ran}_\mu(X)} V \quad \text{当且仅当}\quad \mu(U\cap V) = \mu(U\cup V).
	$$
	可以验证 $\equiv_{\operatorname{Ran}_\mu(X)}$ 确实是一个等价关系; 直观上, 这个等价关系是相差一个 $\mu$-零测集.
\end{propdef}

\begin{prop}
	{}
	设 $X$ 为清晰空间, 带有测度 $\mu$, 则 $\operatorname{Ran}_\mu(X)$ 的点一一对应于 $X$ 的原子.
\end{prop}

\todo{解释原子的含义}

\todo{}

\section{位象与逻辑}

\philoquote{[A locale is a] propositional geometric theory pretending to be a space.}{Steven Vickers}

建议在阅读本节之前阅读附录 \ref{first-order-languages} 节, 尤其关注几何逻辑的内容.

% frame = complete Heyting algebra (但作为范畴不同构)

我们将建立如下对应.

\begin{itemize}
	\item 位象 = 命题逻辑理论
	\item 开集 = 命题
	\item 点 = 模型
\end{itemize}

\subsection{经典命题逻辑与 Lindenbaum 代数}

Lindenbaum 代数是用代数方法研究逻辑的工具, 代数--几何对偶. 粗略地说, 它是一个理论中的 ``命题的代数''; 在经典逻辑的情形, 它就是 \emph{Boole 代数}, 而对应的 ``空间'' 是 \emph{Stone 空间}, 两者的对偶由 \emph{Stone 表示定理}描述.

设 $\Sigma$ 是由若干命题符号 $p,q,r,\cdots$ 组成的符号表\footnote{在定义 \ref{def-first-order-signature} 的框架中, 我们可以说 $\Sigma$ 包括 $0$ 个类型, $0$ 个函数符号 (固然, 因为 $\Sigma$ 中没有类型), 以及若干个 $0$ 元关系符号 $p,q,r,\cdots$.}.
定义 $\mathsf {Sen}_\Sigma$ 为 $\Sigma$ 上由 $\lor,\land,\neg,\Rightarrow,\top,\bot$ 组成的公式\footnote{这里的公式也可以放在定义 \ref{formula} 的框架中. 注意, 在这个框架中没有等式 $p=q$, 因为 $p,q$ 不是某个类型的项 ($\Sigma$ 中没有类型), 而是 $0$ 元关系符号.}的集合, 包括 $(p\land q)\Rightarrow r$, $p\lor\neg p$, $p\Rightarrow\bot$ 等等.

\begin{definition}
	{}
	在经典逻辑中, 对于符号表 $\Sigma$ 上的命题理论 $\mathbb T$ (即一些公理的集合), 定义 $\mathbb T$ 的 \emph{Lindenbaum 代数} $\text{LA}_{\mathbb T}$ 为 $\Sigma$ 上公式的 $\mathbb T$-\emph{可证等价类}构成的 Boole 代数:
	$$
	\text{LA}_{\mathbb T} := \mathsf {Sen}_\Sigma \big/ {\equiv_{\mathbb T}},\quad
	\text{其中 $\phi\equiv_{\mathbb T} \psi$ 当且仅当 $\mathbb T$ 可证 $\phi\Leftrightarrow\psi$}.
	$$
\end{definition}

\begin{definition}
	{(命题理论之间的态射)}
	对于两个命题理论 $(\Sigma,\mathbb T)$, $(\Sigma',\mathbb T')$, 定义态射 $f\colon \mathbb T \to \mathbb T'$ 为映射 $f\colon \Sigma \to \mathsf {Sen}_{\Sigma'}$, 它自然诱导映射 $\mathsf {Sen}_\Sigma\to \mathsf {Sen}_{\Sigma'}$, 使得 $f$ 保持公理, 即 $f(\mathbb T)\subset \mathbb T'$. 等价地, $f$ 为 Boole 代数同态 $\text{LA}_{\mathbb T} \to \text{LA}_{\mathbb T'}$.
\end{definition}

\begin{remark}
	{}
	命题理论的态射可类比于环同态: 对于两个环 $R=\mathbb{Z}[x_1,\cdots,x_n]/I$ 与 $R'=\mathbb{Z}[y_1,\cdots,y_m]/J$, 环同态 $f\colon R\to R'$ 为映射 $\{x_1,\cdots,x_n\}\to R'$ (它自然诱导映射 $\mathbb{Z}[x_1,\cdots,x_n]\to \mathbb{Z}[y_1,\cdots,y_m]$), 使得 $f(I)\subset J$.
\end{remark}

\begin{definition}
	{(命题理论的模型)}
	一个命题理论 $(\Sigma,\mathbb T)$ 的\emph{经典模型} (classical model, 又称\emph{解释}, interpretation) 是一个函数 $f\colon \Sigma \to \{\top,\bot\}$, 它自然诱导一个映射 $\bar f\colon \mathsf {Sen}_\Sigma\to\{\top,\bot\}$, 使得 $\bar f(\mathbb T)\subset \{\top\}$. 记 $\mathbb T$ 的经典模型的集合为 $\text{Mod}(T)$.
	
	一般地, 对任意 Boole 代数 $A$, $(\Sigma,\mathbb T)$ 的 \emph{$A$-模型}是一个映射 $f\colon \Sigma \to A$, 它自然诱导一个映射 $\bar f\colon \mathsf {Sen}_\Sigma\to A$, 使得 $\bar f(\mathbb T)\subset \{\top\}$. 记 $\mathbb T$ 的 $A$-模型的集合为 $\text{Mod}_A(\mathbb T)$.
\end{definition}



\begin{prop}
	{}
	命题理论 $(\Sigma,\mathbb T)$ 的 $A$-模型一一对应于 Boole 代数同态 $\text{LA}_{\mathbb T} \to A$.
\end{prop}

\begin{proof}
	设 $f\colon \Sigma\to A$ 为 $\mathbb T$ 的 $A$-模型. 我们需要证明 $\bar f\colon \mathsf {Sen}_\Sigma\to A$ 将 $\mathbb T$-等价的公式对应到相同的元素. 若 $\phi \equiv_{\mathbb T} \psi$, 设 $\mathbb T$ 对 $\phi \Leftrightarrow\psi$ 的证明用到了公理 $t_1,\cdots,t_n$, 那么由经典逻辑的性质可知
	$$
	\big( \bar f(t_1) \land \cdots \land \bar f (t_n) \big) \leq \big( \bar f(\phi) \Leftrightarrow \bar f(\psi) \big),
	$$
	但由 $f$ 是 $\mathbb T$ 的 $A$-模型, $\bar f(t_1)=\cdots = \bar f(t_n) = \top\in A$, 这说明 $\bar f(\phi) = \bar f(\psi)$.
\end{proof}

\begin{remark}
	{}
	上述命题可类比于环同态的如下性质:
	对于多项式 $f_1,\cdots,f_m\in \mathbb{Z}[x_1,\cdots,x_n]$ 以及任何环 $A$, 集合
	$$
	\big\{
	(a_1,\cdots,a_n)\in A^n\mid f_i(a_1,\cdots,a_n)=0\,(i=1,2,\cdots,m)
	\big\}
	$$
	一一对应于环同态 $\mathbb{Z}[x_1,\cdots,x_n]/(f_1,\cdots,f_m) \to A$.
\end{remark}

\begin{remark}
	{}
	由上述命题, $\mathbb T$ 的经典模型一一对应于 Boole 代数同态 $\text{LA}_{\mathbb T} \to \{\top,\bot\}$, 即 $\text{LA}_{\mathbb T}$ 的\emph{超滤}.
	回忆位象 $X$ 的点是\fm{}同态 $\mathcal O(X) \to \{\top,\bot\}$, 即 $\mathcal O(X)$ 的完全素滤子. 两种情形的区别在于逻辑规则: Boole 代数中没有任意并, 而有排中律; \fm{}中有任意并, 没有排中律.
\end{remark}

\begin{prop}
	{}
	命题理论的态射 $f\colon (\Sigma,\mathbb T) \to (\Sigma',\mathbb T')$ 诱导模型的映射 $\text{Mod}_A(\mathbb T') \to \text{Mod}_A(\mathbb T)$.
\end{prop}

注意方向的反转 (这是代数--几何对偶的一部分).
%集合 $\text{Mod}(\mathbb T)$ 可附加一个拓扑, 以 $\{f\in\text{Mod}(\mathbb T)\mid \bar f(\phi)=\top\}$ 为开集基 ($\phi$ 取所有公式).

\begin{definition}
	{(Stone 空间)}
	称紧的完全不连通 (仅有的连通分支为单点集) Hausdorff 空间为 \emph{Stone 空间}. 等价的定义是紧且\emph{完全分离} (任何两个点都能通过把空间分成两个连通分支而分开) 的空间.
\end{definition}

\begin{propdef}
	{}
	对 Boole 代数 $B$, 在集合 $\operatorname{Hom}_{\mathsf {BooleAlg}}(B,\{\top,\bot\})$ 上附加一个拓扑, 以 $\{b\in B\mid f(b)=\top\}$ 为开集基; 那么 $\operatorname{Hom}_{\mathsf {BooleAlg}}(B,\{\top,\bot\})$ 是 Stone 空间.
\end{propdef}

\begin{proof}
	首先证明 $\operatorname{Hom}_{\mathsf {BooleAlg}}(B,\{\top,\bot\})$ 是紧空间. 事实上, 它是积空间 $\{\top,\bot\}^{B}$ 的闭子集. 由 Tychonoff 定理, 紧空间的任意乘积是紧空间, 故 $\{\top,\bot\}^{B}$ 是紧空间, 从而 $\operatorname{Hom}_{\mathsf {BooleAlg}}(B,\{\top,\bot\})$ 是紧空间.
	\todo{}
\end{proof}

\begin{prop}
	{(Stone 对偶)}
	
	Boole 代数与 Stone 空间之间存在范畴等价
	% https://q.uiver.app/#q=WzAsMixbMCwwLCJcXG1hdGhzZiB7Qm9vbGVBbGd9Il0sWzIsMCwiXFxtYXRoc2Yge1N0b25lU3B9Il0sWzAsMSwiXFxvcGVyYXRvcm5hbWV7SG9tfSgtLFxce1xcdG9wLFxcYm90XFx9KSIsMix7Im9mZnNldCI6Mn1dLFsxLDAsIlxcb3BlcmF0b3JuYW1le0hvbX0oLSxcXHtcXHRvcCxcXGJvdFxcfSkiLDIseyJvZmZzZXQiOjJ9XV0=
	\[\begin{tikzcd}[ampersand replacement=\&]
		{\mathsf {BooleAlg}} \&\& {\mathsf {StoneSp}.}
		\arrow["{\operatorname{Hom}_{\mathsf {BooleAlg}}(-,\{\top,\bot\})}"', shift right=2, from=1-1, to=1-3]
		\arrow["{\operatorname{Hom}_{\mathsf {StoneSp}}(-,\{\top,\bot\})}"', shift right=2, from=1-3, to=1-1]
	\end{tikzcd}\]
\end{prop}

\begin{proof}
	
\end{proof}

\todo{}

% sublocale 与 product locale 对应的理论是什么
