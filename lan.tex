\chapter{\topos{}的内语言}

\philoquote{
    A mathematical statement is just a story you tell about some devices. Some of those stories are clever, some are stupid; some of those stories are true, some others are false. Doing mathematics is telling clever stories which are true.\footnotemark
}{
    Francis Borceux, \cite{HCA3}
}
\footnotetext{一句数学陈述不过是你对某些东西讲的一个故事. 这些故事或妙, 或蠢, 或真, 或假. 做数学就是要讲出又妙又真的故事.}

\minitoc

\topos{}的用处在于许多数学可以在一个\topos{}的内部进行. 相比于画交换图, 一种更优雅的做数学的方式是使用\topos{}的\emph{内语言} (internal language) 进行推理. 本章介绍这种语言, 以及它在各个数学分支中的应用.

\begin{remark}
	{}
	建议读者在阅读本章之前先阅读附录 \ref{logic-appendix}.
\end{remark}

\section{Mitchell--B\'enabou 语言}

\label{Mitchell--Benabou-language}

\label{logical-functor-internal}

本节描述一种重要的语言, 称作 \emph{Mitchell--B\'enabou 语言}; 它是由给定的\topos{} $\mathcal C$ 定义出的一种一阶或高阶语言, 其特点是利用子对象分类器 $\Omega$, 将\emph{公式}一视同仁地解释为 $\Omega$ 类型的\emph{项}. 使用这种语言, 可将\topos{}中的对象在语法上当作集合一样处理.

\begin{definition}
    {(类型)}
    Mitchell--B\'enabou 语言中的\emph{类型}是 $\mathcal C$ 的对象.
\end{definition}


%\begin{itemize}
%	\item 类型 $X$ 的\emph{项}被解释为指向 $X$ 的态射, 而\emph{公式}被解释为 $\Omega$ 类型的项;
%	项 (公式) 的定义域代表此项 (公式) 中自由变量的类型.
%	例如, 当类型 $X$ 的一个项含有自由变量 $y\colon Y, z\colon Z$ 时, 该项解释为一个态射 $Y\times Z \to X$.
%	\item 
%	\item 
%\end{itemize}


\begin{definition}
	{(函数符号, 关系符号)}
	Mitchell--B\'enabou 语言中的\emph{函数符号} $f\colon A_1\cdots A_n \to B$ 是 $\mathcal C$ 中的态射
	$$f\colon A_1\times\cdots\times A_n\to B.$$
	特别地, 类型 $X$ 的\emph{常量} (零元函数) 是态射 $1 \to X$, 也即对象 $X$ 的\emph{整体元素}.
	
	Mitchell--B\'enabou 语言中的\emph{关系符号} $R\hookrightarrow A_1\cdots A_n$ 是 $\mathcal C$ 中的态射
	$$
	R\colon A_1\times\cdots\times A_n \to \Omega,
	$$
	也即 $A_1\times\cdots\times A_n$ 的子对象, 其直观为 ``满足关系 $R$ 的元素构成的子集''.
	特别地, 原子命题 (零元关系) 是态射 $1\to\Omega$, 也即\emph{真值}.
	
	%我们看到, 函数符号与项被解释成了同一种东西. 这是有道理的, 因为\emph{项}在这里就是其自由变量的函数.
\end{definition}

% 至此, Mitchell--B\'enabou 语言已经定义完成.

以上两个定义给出了一个符号表 $\Sigma$, 自然, $\mathcal C$ 中具有典范的 $\Sigma$-结构.
由定义 \ref{term-interpretation}, \ref{higher-order-term-interpretation}, 可以归纳地得到所有项和公式的\emph{解释} (interpretation).
注意在附录 \ref{logic-appendix} 中公式被解释为子对象, 而在\topos{}中它等同于类型 $\Omega$ 的项.

%\begin{definition}
%    {(变量的解释)}
%    
%\end{definition}


%\begin{definition}
%	{(等式)}
%	
%\end{definition}


%\begin{propdef}
%	{(项的解释)}
%	类型 $X$ 的项解释为指向 $X$ 的态射 $\sigma\colon U\to X$, 其中 $U$ 是该项所含的变量的类型.
%	\begin{itemize}
%		\item (变量)
%		类型 $X$ 的一个\emph{变量} (可视为含一个自由变量的项) $x\colon X$ (为了体现与集合论的相似性也记为 $x\in X$) 解释为恒等态射 $\operatorname{id}\colon X \to X$.
%		\item (函数取值)
%		对于函数符号 $f\colon X \to Y$ 与类型 $X$ 的项 $\sigma\colon U \to X$, $f(\sigma)$ 解释为复合 $f\circ\sigma \colon U \to Y$.
%	\end{itemize}
%	由 (一阶语言中) 项的定义 (\ref{definition-terms}), 上述条款归纳地解释了所有的项.
%	高阶语言的项还需要如下的解释.
%	\begin{itemize}
%		\item (二元对, )
%		\item ()
%		\item (函数引入法则, 即 ``$\lambda$-演算'') 设 $x$ 是类型 $X$ 的变量, $\sigma\colon X\times U\to Z$ 是任意项,
%		那么 $\lambda x.\sigma$ 解释为 $\sigma$ 在伴随之下对应的态射 $U\to Z^X$.
%		\item (函数消去法则) 对于
%	\end{itemize}
%\end{propdef}

\begin{example}
	{(变量, 一般元素)}
	设 $x$ 是类型 $X$ 的一个变量. 由定义 \ref{term-interpretation}, 其解释为 $\operatorname{id}\colon X\to X$.
	
	变量 $x$ 可视为类型 $X$ 的\emph{一般元素} (generic element).
	其中 ``元素'' 是指广义元素, 即指向 $X$ 的态射; ``一般'' 是指如下的泛性质: 假若我们证明了含变量 $x$ 的公式 $\phi(x)$,
	那么对 $X$ 的任意具体的项 (广义元素) $x_0\colon U\to X$, 都有 $\phi(x_0)$ 成立.
	这是由于 $\operatorname{id}_X$ 是所有指向 $X$ 的态射中的终对象.
\end{example}

\begin{example}
	{(函数的取值)}
	使用取值映射 $\operatorname{ev}\colon Y^X\times X\to Y$ (定义 \ref{evaluation-map}),
	可将 $Y^X$ 类型的项 (内语言中的 ``函数'') $\theta\colon V\to Y^X$
	作用于 $X$ 类型的项 $\sigma \colon U \to X$,
	得到 $Y$ 类型的项
	% https://q.uiver.app/#q=WzAsMyxbMCwwLCJWXFx0aW1lcyBVIl0sWzEsMCwiWV5YXFx0aW1lcyBYIl0sWzIsMCwiWSJdLFswLDEsIihcXHRoZXRhLFxcc2lnbWEpIl0sWzEsMiwiXFxvcGVyYXRvcm5hbWV7ZXZ9Il1d
	\[\theta(\sigma)\colon
	\begin{tikzcd}[ampersand replacement=\&]
		{V\times U} \& {Y^X\times X} \& Y.
		\arrow["{(\theta,\sigma)}", from=1-1, to=1-2]
		\arrow["{\operatorname{ev}}", from=1-2, to=1-3]
	\end{tikzcd}\]
\end{example}

\begin{example}
	{(成员关系)}
	考虑成员关系 (例 \ref{membership-relation}) ${\in_X} \colon \Omega^X\times X \to \Omega$,
	可对 $PX=\Omega^X$ 类型的项 (内语言中的 ``子集'') $\eta\colon V \to \Omega^X$ 与 $X$ 类型的项 $\sigma\colon U\to X$ 定义 $\Omega$ 类型的项
	% https://q.uiver.app/#q=WzAsMyxbMCwwLCJWXFx0aW1lcyBVIl0sWzEsMCwiXFxPbWVnYV5YXFx0aW1lcyBYIl0sWzIsMCwiXFxPbWVnYSJdLFswLDEsIihcXGV0YSxcXHNpZ21hKSJdLFsxLDIsIlxcaW5fWCJdXQ==
	\[
	(\sigma \in \eta) \colon 
	\begin{tikzcd}[ampersand replacement=\&]
		{V\times U} \& {\Omega^X\times X} \& \Omega.
		\arrow["{(\eta,\sigma)}", from=1-1, to=1-2]
		\arrow["{\in_X}", from=1-2, to=1-3]
	\end{tikzcd}\]
\end{example}

\begin{example}
	{(等式)}
	对于类型 $X$ 的变量 $x_1,x_2$, \emph{等式} $x_1 = x_2$ 被解释为 $\chi_\Delta\colon X\times X\to\Omega$, 即对角线 $\Delta\colon X\to X\times X$ 的特征函数 (例 \ref{diagonal}).
\end{example}

\begin{example}
	{(存在量词与任意量词)}
	设 $\phi(x,y)$ 是含两个变量 $x\in X,y\in Y$ 的公式.
		考虑投影 $\pi\colon X\times Y\to Y$ 诱导的子对象偏序集之间的三元伴随
		\[\begin{tikzcd}[ampersand replacement=\&,background color=\propcolor]
			{\operatorname{Sub}(X\times Y)} \&\& {\operatorname{Sub}(Y).}
			\arrow[""{name=0, anchor=center, inner sep=0}, "{\pi^*}"{description, pos=0.3}, from=1-3, to=1-1]
			\arrow[""{name=1, anchor=center, inner sep=0}, "{\exists_\pi}"{description, pos=0.3}, shift left=5, from=1-1, to=1-3]
			\arrow[""{name=2, anchor=center, inner sep=0}, "{\forall_\pi}"{description, pos=0.3}, shift right=5, from=1-1, to=1-3]
			\arrow["\dashv"{anchor=center, rotate=-90}, draw=none, from=1, to=0]
			\arrow["\dashv"{anchor=center, rotate=-90}, draw=none, from=0, to=2]
		\end{tikzcd}\]
		(命题 \ref{forall}, \ref{image-vs-exists})
		公式 $\exists x\in X\, \phi(x,y)$, $\forall x\in X\, \phi(x,y)$
		分别解释为子对象 $\{(x,y)\in X\times Y \mid \phi(x,y)\}$ 在 $\exists_\pi$ 与 $\forall_\pi$ 下的像 (的特征函数).
		(在合适的语境下, 记号中的 ``$\in X$'' 可省略.)

	注意到 $\operatorname{Hom}(-,\Omega^X) \simeq\operatorname{Sub}(X\times {-})$, 由米田引理, 子对象偏序集之间的三元伴随给出三个态射
	\[\begin{tikzcd}[ampersand replacement=\&,background color=\propcolor]
		{\Omega^X} \&\& {\Omega.}
		\arrow[""{name=0, anchor=center, inner sep=0}, "{X^*}"{description, pos=0.3}, from=1-3, to=1-1]
		\arrow[""{name=1, anchor=center, inner sep=0}, "{\exists_X}"{description, pos=0.3}, shift left=5, from=1-1, to=1-3]
		\arrow[""{name=2, anchor=center, inner sep=0}, "{\forall_X}"{description, pos=0.3}, shift right=5, from=1-1, to=1-3]
		%\arrow["\dashv"{anchor=center, rotate=-90}, draw=none, from=1, to=0]
		%\arrow["\dashv"{anchor=center, rotate=-90}, draw=none, from=0, to=2]
	\end{tikzcd}\]
	它们是所谓 ``内蕴伴随''.
\end{example}

\begin{remark}
	{}
	注意我们在两处使用了 ``属于'' 符号 $\in$: 一处是含全称量词或存在量词的公式 $\forall x\in X\,\phi(x)$ 或 $\exists x\in X\,\phi(x)$,
	一处是成员关系 $x\in S$. 这并不会引起歧义, 因为对于子对象 $S\hookrightarrow X$ 有
	\[
	\begin{aligned}
		\forall x\in S\, \phi(x)&\quad \Leftrightarrow\quad
		\forall x\in X (x\in S\Rightarrow \phi(x))\\
		\exists x\in S\, \phi(x)&\quad \Leftrightarrow\quad
		\exists x\in X (x\in S\land \phi(x)).
	\end{aligned}
	\]
	甚至当 $S$ 是类型 $PX$ 的一般的项时, 我们也可简记 $\forall x\in X(x\in S\Rightarrow\phi(x))$ 为 $\forall x\in S\,\phi(x)$, 简记  $\exists x\in X(x\in S\land\phi(x))$ 为 $\exists x\in S\,\phi(x)$.
\end{remark}

\begin{example}
	[label={subobject-from-formula}]
	{(公式确定的子对象)}
	对公式 $\phi(x)$, 设其解释为 $\phi(x)\colon X\to\Omega$, 那么 $\{x\in X \mid \phi(x)\}$ 的解释是以 $\phi(x)$ 为特征函数的子对象.
	我们称之为公式 $\phi(x)$ 的\emph{外延} (extension)\footnotemark{}.
	
	更一般地, $\phi$ 不仅可以是 $X$ 上的公式, 而且可以是 $\Omega^X$ 的项 (即内语言中的 ``$X$ 上的公式''); 此时我们沿用记号 $\{x\in X\mid\phi(x)\}$ 表示 $\phi$ 自身, 视为 $PX$ 的项.
	
	任何子对象 $S\hookrightarrow X$ 都至少有一个公式与之对应, 即公式 $x\in S$, 因为 $\in$ 的定义是 $\operatorname{id}_{\Omega^X}$ 对应的态射 $\Omega^X\times X \to \Omega$.
\end{example}
\footnotetext{这与哲学上的用法是一致的: 外延是一个词语适用的对象的集合.}

%\begin{definition}
%	{(真值, 内语言定义)}
%	公式 $\phi(x)$ 的\emph{真值}是集合
%	$$
%	\{\star \in 1 \mid \phi\} \in \Omega,
%	$$
%	其中 $\star$ 代表任意一个不出现在 $\phi(x)$ 中的变量.
%\end{definition}

\begin{propdef}
	[label={universally-valid}]
	{(恒成立)}
	对公式 $\phi(x)$, 如下条件等价:
	\begin{enumerate}[(1)]
		\item $\{x\in X\mid \phi(x)\} = X$;
		\item $\phi(x)\colon X\to\Omega$ 穿过 $\top\colon 1\to\Omega$;
		\item 公式 $\forall x\in X\, \phi(x) \colon 1 \to \Omega$ 等于 $\top$.
	\end{enumerate}
	我们称满足上述条件的公式 $\phi(x)$ (在 $X$ 上) \emph{恒成立} (is universally valid), 简称\emph{成立}.
\end{propdef}
\begin{proof}~
	\begin{itemize}
		\item $(1)\Leftrightarrow (2)$ 由 $\{x\in X\mid \phi(x)\}$ 的定义即得.
		\item $(2)\Leftrightarrow (3)$. 由任意量词的解释, 对于 $p\in\operatorname{Sub}(1)$,
		\[
		X\times p\leq \{x\in X\mid \phi(x)\}\quad \text{当且仅当}\quad p\leq \forall x\in X\, \phi(x).
		\]
		因此 $\{x\in X\mid \phi(x)\} = X$ 等价于 $p=1$ 满足以上两式, 等价于 $\big(\forall x\in X\, \phi(x)\big) = \top$.
	\end{itemize}
\end{proof}

\begin{remark}
	{}
		为了避免混淆, 作如下约定: 在本章中如无特殊说明, 自然语言 (中文) 的逻辑连接词 ``对任意'' ``存在'' ``或'' 等等是通常的 (``外部'' 的) 数学语言,
	而符号 $\forall,\exists,\lor$ 等等是某个\topos{}的内语言, 即 Mitchell--B\'enabou 语言.
\end{remark}

\subsection{使用 Mitchell--B\'enabou 语言表达\topos{}中的对象和态射}

\begin{example}
	[label={image-internal-definition}]
	{(像)}
	映射 $f\colon X\to Y$ 的像为
	\[
	\operatorname{im}(f) = \{y\in Y\mid \exists x\in X\, f(x) = y\}.
	\]
	记 $\pi\colon X\times Y\to Y$ 为投影, 那么对于 $U\in\operatorname{Sub}(Y)$, 有 $\pi^* U=X\times U$.
	由存在量词的解释, 上述 $\operatorname{im}(f)$ 的定义可翻译为
	\[
	\operatorname{im}(f)\leq U \quad \text{当且仅当} \quad \{(x,y)\in X\times Y \mid f(x)=y\} \leq X\times U.
	\]
	而后者等价于 $f$ 穿过 $U\hookrightarrow Y$; 故这个条件正是像的范畴论定义 (\ref{image-definition}).
	为了体现与通常数学的一致性, 我们也将 $\operatorname{im}(f)$ 记为 $\{f(x)\mid x\in X\}$.
%	注意此时我们有
%	\[
%	\forall y\in \operatorname{im}(f)\,\exists x\in X\, f(x)=y.
%	\]
\end{example}

\begin{example}
	{(单射与满射)}
	对于\topos{}中的态射 $f\colon X\to Y$,
	\begin{itemize}
		\item $f$ 为单射当且仅当公式 $\forall x_1\in X\,\forall x_2\in X\,\big(f(x_1)=f(x_2)\Rightarrow x_1=x_2\big)$ 成立.
		\item $f$ 为满射当且仅当公式 $\forall y\in Y\, \exists x\in X\, f(x)=y$ 成立.
	\end{itemize}
\end{example}
\begin{proof}~
	\begin{itemize}
		\item 公式 $\forall x_1\in X\,\forall x_2\in X\,\big(f(x_1)=f(x_2)\Rightarrow x_1=x_2\big)$ 翻译为外部语言即左下图交换 (见命题 \ref{universally-valid}).
		% https://q.uiver.app/#q=WzAsOCxbMCwwLCJYXFx0aW1lcyBYIl0sWzAsMSwiXFxPbWVnYVxcdGltZXNcXE9tZWdhIl0sWzEsMSwiXFxPbWVnYSJdLFsxLDAsIjEiXSxbMiwwLCJYIl0sWzMsMCwiWSJdLFsyLDEsIlhcXHRpbWVzIFgiXSxbMywxLCJZXFx0aW1lcyBZIl0sWzAsMSwiXFxiaWcoXFxjaGlfXFxEZWx0YVxcY2lyYyAoZlxcdGltZXMgZiksXFxjaGlfXFxEZWx0YVxcYmlnKSIsMl0sWzEsMiwiXFxSaWdodGFycm93IiwyXSxbMCwzXSxbMywyLCJcXHRvcCJdLFs2LDcsImZcXHRpbWVzIGYiLDJdLFs0LDYsIlxcRGVsdGEiLDJdLFs1LDcsIlxcRGVsdGEiXSxbNCw1LCJmIl1d
		\[\begin{tikzcd}[ampersand replacement=\&]
			{X\times X} \& 1 \& X \& Y \\
			\Omega\times\Omega \& \Omega \& {X\times X} \& {Y\times Y}
			\arrow["{\big(\chi_\Delta\circ (f\times f),\chi_\Delta\big)}"', from=1-1, to=2-1]
			\arrow["\Rightarrow"', from=2-1, to=2-2]
			\arrow[from=1-1, to=1-2]
			\arrow["\top", from=1-2, to=2-2]
			\arrow["{f\times f}"', from=2-3, to=2-4]
			\arrow["\Delta"', from=1-3, to=2-3]
			\arrow["\Delta", from=1-4, to=2-4]
			\arrow["f", from=1-3, to=1-4]
		\end{tikzcd}\]
		注意到 $f$ 是单射当且仅当上面的右图为拉回, 而右图为 (子对象的) 拉回等价于特征函数满足 $\chi_\Delta\circ (f\times f) = \chi_\Delta$.
		这等价于上面的左图交换.
		\item 由例 \ref{image-internal-definition} 中的讨论, 公式 $\forall y\in Y\, \exists x\in X\, f(x)=y$ 成立等价于 $\operatorname{im}f=Y$, 即 $f$ 为满射.
	\end{itemize}
\end{proof}

\begin{example}
	[label={inhabited-vs-nonempty}]
	{(``有物'' 与 ``非空'')}
	在\topos{}中, 称对象 $X$ ``\emph{有物}'' (inhabited), 是指 $X\to 1$ 为满射; 在内语言中此即公式 $\exists x\in X \top$ 成立. 称 $X$ ``\emph{非空}'' (nonempty) 是指公式 $\neg(\forall x\in X \bot)$ 成立. 一般而言, 有物与非空是\emph{不同的}概念; 这一点从后面介绍的拓扑空间上的层语义的例子 (\ref{sheaf-semantics-topological-space}) 可以看出 (考虑一个稠密开集作为子终对象, 它不是有物的, 却是非空的).
	
	另外需要注意的是, 一个对象 $X$ 有物, $X$ 不一定有整体元素, 即 (以外部观点) 不一定存在态射 $1\to X$. 这是因为一般的\topos{}中没有选择公理, 见例 \ref{example-no-AC}.
\end{example}

\begin{example}
	[label={set-of-epimorphisms}]
	{(满射的集合)}
	$X$ 到 $Y$ 的满射的 ``集合'' 为
	\[
	\operatorname{Epi}(X,Y) = \{f\in Y^X \mid \forall y\in Y\, \exists x\in X\, f(x)=y\} = \{f\in Y^X\mid\operatorname{im}f=Y\}.
	\]
	后一种写法使用了像的内语言定义 (命题 \ref{image-internal-definition}).
	%这个对象有另一种使用外部语言的定义, 其中用到了内蕴版本的 ``像'' $\operatorname{im}\colon Y^X\to \Omega^Y$.
	%见 \cite{SGL} VI.3 节.
\end{example}

\begin{example}
	[label={internal-language-singleton}]
	{(单元集)}
	回忆 ``单元集映射'' $\{-\}\colon X\to PX$ 是 $\delta_X\colon X\times X\to\Omega$ 对应的态射 (例 \ref{singleton}).
	换言之,
	\[
	\{x\} = \{y\in X\mid y=x\}.
	\]
	
	$PX$ 上有一个谓词 ``是单元集''.
	首先我们定义\emph{子单元集} (subsingleton), 也即单元集的子集\footnotemark{}:
	\[
	\internalprop{\text{$S$ 是子单元集}} := \forall x\in S\,\forall y\in S\,x=y.
	\]
	接着定义
	\[
	\internalprop{\text{$S$ 是单元集}} := \internalprop{\text{$S$ 是子单元集}} \land \exists x\in S\,\top
	=(\exists ! x\in S\,\top).
	\]
	(其中 $\exists !$ 表示存在唯一, 见定义 \ref{formula}) 可以证明
	\[
	\forall S\in PX\,\big(\internalprop{\text{$S$ 是单元集}}\,\Leftrightarrow\,\exists x\in X\,S=\{x\}\big).
	\]
\end{example}
\footnotetext{在经典逻辑中只有两个不同的子单元集, 即空集与单元集; 但在\topos{}的内语言中则不然: 子单元集对应\topos{}的子终对象, 见定义 \ref{subterminal-object-definition}.}

\begin{example}
	[label={function-vs-graph}]
	{(函数及其图像)}
	设 $\Gamma\in P(X\times Y)$ 使得公式 $\forall x\,\exists ! y\, (x,y)\in\Gamma$ 成立. 设 $\Gamma$ 对应态射 $\widetilde f\colon X\to PY$, 则有 $\forall x\,\exists y\,\widetilde f(x)=\{y\}$, 也即 $\operatorname{im}f\subset \operatorname{im}\{{-}\}$. 这说明 $\widetilde f$ 穿过一个态射 $f\colon X\to Y$. 进一步, 可以证明
	\[
	Y^X \simeq \{\Gamma\in P(X\times Y)\mid \forall x\,\exists ! y\, (x,y)\in\Gamma\}.
	\]
\end{example}

\begin{example}
	[label={quotient-set-internal}]
	{(商集)}
	%内语言中的商集即范畴论中的余等化子.
	集合论中, 商集由等价类的集合构造; 在内语言中我们也可仿照这一构造.
	首先回顾等价关系 (\ref{equivalence-relation}) 的定义.
	内语言中, 集合 $X$ 上的等价关系是满足如下条件的二元关系 $\sim$:
	\begin{itemize}
		\item $x\sim x$;
		\item $x\sim y\Rightarrow y\sim x$;
		\item $(x\sim y \land y\sim z) \Rightarrow x\sim z$.
	\end{itemize}
	对于 $x\in X$ 定义 ``等价类''
	\[
	[x] := \{x'\in X\mid x'\sim x\},
	\]
	从而有映射 $[-]\colon X\to PX$.
	于是商集 $X/{\sim}$ 可定义为
	\[
	X/{\sim} := \{[x] \mid x\in X\} := \operatorname{im}([-]\colon X\to PX).
	\]
	注意此时我们有
	\[
	\forall \alpha\in (X/{\sim}) \,\exists x\in X\, \alpha = [x].
	\]
\end{example}

\begin{example}
	[label={internal-Boolean-topos}]
	{(Boole \topos{})}
	一个\topos{}是 Boole 的, 当且仅当
	\begin{itemize}
		\item (排中律) $\forall p\in\Omega\, (p\lor \neg p)$;
		\item (双重否定消去) $\forall p\in\Omega\, (\neg\neg p\Rightarrow p)$.
	\end{itemize}
	注意公式 $p\lor \neg p$ 成立\emph{不能}推出 $p=\top$ 成立或 $p=\bot$ 成立 (后者是\emph{二值性}, 见注 \ref{boolean-not-two-valued}), 因为 $\lor$ 是内语言, 而 ``或'' 是 ``外部'' 语言.
	
	下面证明排中律成立等价于双重否定消去成立.
	假设排中律, 那么
	\begin{align*}
		\neg\neg p &=\neg\neg p\land (p\lor\neg p)\\& \Rightarrow (\neg\neg p \land p) \lor (\neg\neg p \land\neg p)\\
		&\Rightarrow (\neg\neg p \land p) \lor\bot\\
		&= (\neg\neg p\land p)\\
		&\Rightarrow p.
	\end{align*}
	另一方面, 假设双重否定消去成立. 注意到
	\begin{align*}
		\neg\neg (p\lor\neg p) & = \neg (\neg p\land\neg\neg p) &\text{(命题 \ref{Heyting-algebra-implication-properties} (2))}\\
		&=\neg \bot \\&= \top,
	\end{align*}
	故由双重否定消去知 $p\lor\neg p$ 成立.
\end{example}

\begin{example}
	{(内蕴选择公理)}
	一个\topos{}满足内蕴选择公理 (定义 \ref{internal-axiom-of-choice}), 当且仅当对任意对象 $X,Y$ 如下公式成立.
	\[
	\forall f\in\operatorname{Epi}(X,Y)\, \exists g\in X^Y\, \forall y\in Y\, f(g(y))=y.
	\]
\end{example}

因为一般的\topos{}不一定是 Boole 的, 也不一定满足内蕴选择公理, 所以在内语言中进行推理一般不能使用以上两例中写出的公式.
内语言中可以使用的推导法则是\emph{直觉主义谓词演算} (intuitionistic predicate calculus).

% 讲逻辑态射?
% nLab https://ncatlab.org/nlab/show/logical+functor




\section{Kripke--Joyal 语义}

\emph{语义} (semantics) 是将一种形式语言 (如前面介绍的 Mitchell--B\'enabou 语言) 的公式转化为另一种语言 (如通常数学语言) 的方法. 回忆 Mitchell--B\'enabou 语言中, 含一个变量 $x\colon X$ 的公式 $\phi(x)$ 被解释为一个态射 $\phi(x) \colon X \to \Omega$.
子对象 $\{x \mid \phi(x)\}$ 是 $\top\colon 1\to\Omega$ 沿 $\phi(x)$ 的拉回 (定义 \ref{subobject-from-formula}).
设 $X$ 有整体元素 $x_0\colon 1\to X$, 那么 $x_0$ 满足公式 $\phi$ 当且仅当下图中虚线态射存在.
% https://q.uiver.app/#q=WzAsNSxbMCwxLCIxIl0sWzEsMSwiWCJdLFsyLDEsIlxcT21lZ2EiXSxbMiwwLCIxIl0sWzEsMCwiXFx7eFxcbWlkXFx2YXJwaGkoeClcXH0iXSxbNCwxXSxbMCwxLCJ4XzAiLDJdLFsxLDIsIlxcdmFycGhpKHgpIiwyXSxbMCw0LCIiLDAseyJzdHlsZSI6eyJib2R5Ijp7Im5hbWUiOiJkYXNoZWQifX19XSxbNCwzXSxbMywyLCJcXHRvcCJdXQ==
\[\begin{tikzcd}[ampersand replacement=\&]
	\& {\{x\mid\phi(x)\}} \& 1 \\
	1 \& X \& \Omega
	\arrow[from=1-2, to=2-2]
	\arrow["{x_0}"', from=2-1, to=2-2]
	\arrow["{\varphi(x)}"', from=2-2, to=2-3]
	\arrow[dashed, from=2-1, to=1-2]
	\arrow[from=1-2, to=1-3]
	\arrow["\top", from=1-3, to=2-3]
\end{tikzcd}\]
将 ``整体元素'' 推广为 ``广义元素'', 我们得到如下的定义.

\newcommand{\forces}{\models}

\begin{definition}
	{}
	设 $X$ 是一个\topos{}的对象, $\phi(x)$ 是含变量 $x\colon X$ 的公式. 称广义元素 $x_0 \colon U\to X$ 满足公式 $\phi$, 是指下图中虚线态射存在, 也即 $U$ 包含于公式 $\phi$ 确定的子对象.
	\[\begin{tikzcd}[ampersand replacement=\&]
		\& {\{x\mid\phi(x)\}} \& 1 \\
		U \& X \& \Omega
		\arrow[from=1-2, to=2-2]
		\arrow["{x_0}"', from=2-1, to=2-2]
		\arrow["{\phi(x)}"', from=2-2, to=2-3]
		\arrow[dashed, from=2-1, to=1-2]
		\arrow[from=1-2, to=1-3]
		\arrow["\top", from=1-3, to=2-3]
	\end{tikzcd}\]
	此时记 $U\forces \phi(x_0)$, 也称为 $U$ \emph{力迫} (forces) $\phi(x_0)$. (力迫这个词来源于 Cohen, 见 \ref{Cohen-forcing} 节.)
\end{definition}

\begin{prop}
	[label={epi-forcing}]
	{}
	对于广义元素 $x_0 \colon U\to X$, 设 $p\colon U' \twoheadrightarrow U$ 为满射, 若 $U'\forces\phi(x_0 p)$, 则 $U\forces \phi(x_0)$.
\end{prop}
\begin{proof}
	由条件有交换图
	% https://q.uiver.app/#q=WzAsNCxbMCwwLCJVJyJdLFswLDEsIlUiXSxbMSwxLCJYIl0sWzEsMCwiXFx7eFxcbWlkXFxwaGkoeClcXH0iXSxbMCwxLCJmIiwyLHsic3R5bGUiOnsiaGVhZCI6eyJuYW1lIjoiZXBpIn19fV0sWzAsM10sWzMsMiwiIiwyLHsic3R5bGUiOnsidGFpbCI6eyJuYW1lIjoiaG9vayIsInNpZGUiOiJib3R0b20ifX19XSxbMSwyLCJcXGFscGhhIiwyXV0=
	$\begin{tikzcd}[ampersand replacement=\&,row sep=small,column sep=tiny]
		{U'} \& {\{x\mid\phi(x)\}} \\
		U \& X
		\arrow["p"', two heads, from=1-1, to=2-1]
		\arrow[from=1-1, to=1-2]
		\arrow[hook', from=1-2, to=2-2]
		\arrow["x_0"', from=2-1, to=2-2]
	\end{tikzcd}$.
	作 $\{x\mid\phi(x)\}$ 沿 $x_0$ 的拉回 $\widetilde {U}$,
	那么 $\widetilde U \to U$ 既单又满, 故为同构.
	这说明 $U\forces \phi(x_0)$.
	这是满射关于单射的一种提升性质.
\end{proof}

\begin{propdef}
	[label={Kripke--Joyal-recursive-definition}]
	{(Kripke--Joyal 语义的递归定义)}
	设 $x_0 \colon U\to X$ 为广义元素, 如下条款递归地给出了 $U\forces\phi(x_0)$ 的定义 (``按公式 $\phi$ 的复杂度归纳''):
	\begin{enumerate}[(1)]
		\item 对于公式 $\phi(x), \psi(x)$, $U\forces \phi(x_0) \land \psi(x_0)$ 当且仅当 $U\forces \phi(x_0)$ 且 $U\forces\psi(x_0)$.
		\item 对于公式 $\phi(x), \psi(x)$, $U\forces \phi(x_0) \lor \psi(x_0)$ 当且仅当存在 $p\colon V\to U$ 与 $q\colon W\to U$,
		使得 $(p,q)\colon V+W\to U$ 满, 且 $V\forces \phi(x_0 p)$, $W\forces\psi(x_0 q)$.
		\item 对于公式 $\phi(x), \psi(x)$, $U\forces \phi(x_0) \Rightarrow \psi(x_0)$ 当且仅当对任意 $p\colon V\to U$, 只要 $V\forces \phi(x_0 p)$, 就有 $V\forces \psi(x_0p)$; 这又等价于 $U\land\{x\mid \phi(x)\}\forces \psi(x_0)$.
		\item 对于公式 $\phi(x)$, $U\forces \neg\phi(x_0)$ 当且仅当对任意 $p\colon V\to U$, 只要 $V\forces\phi(x_0p)$, 就有 $V\simeq 0$.
		\item 对于公式 $\phi(x,y)$, $U \forces \exists y \phi(x_0,y)$ 当且仅当存在满射 $p\colon V\twoheadrightarrow U$ 以及广义元素 $y_0\colon V\to Y$ 使得 $V\forces \phi(x_0p,y_0)$.
		\item 对于公式 $\phi(x,y)$, $U \forces \forall y \phi(x_0,y)$ 当且仅当对任意 $y_0\colon V\to Y$, $p\colon V\to U$, 都有 $V\forces \phi(x_0p,y_0)$; 这又等价于 $U\times Y \forces \phi(x_0\pi_1,y\pi_2)$, 其中 $\pi_1,\pi_2$ 为 $U\times Y$ 向两个分量的投影.
	\end{enumerate}
\end{propdef}
\begin{proof}
	这些性质的证明基本上是直接验证定义, 我们提供一些细节.
	
	在 (2) 中, $x_0\colon U\to X$ 穿过 $\{x\mid\phi(x)\lor\psi(x)\} = \{x\mid\phi(x)\}\lor\{x\mid\psi(x)\}$,
	当且仅当 $U$ 的两个子对象 $\{u\in U\mid \phi(x_0u)\}$, $\{u\in U\mid \psi(x_0u)\}$ 之并等于 $U$.
	取 $V,W$ 为这两个子对象即可.
	
	在 (4) 中, 若 $x_0\colon U\to X$ 穿过 $\{x\mid \exists y \phi(x,y)\}$, 作如下拉回,
	% https://q.uiver.app/#q=WzAsNCxbMCwwLCJcXHsodSx5KVxcbWlkXFxwaGkoeF8wdSx5KVxcfSJdLFswLDEsIlUiXSxbMSwxLCJcXHt4XFxtaWRcXGV4aXN0cyB5XFxwaGkoeCx5KVxcfSJdLFsxLDAsIlxceyh4LHkpXFxtaWRcXHBoaSh4LHkpXFx9Il0sWzAsMV0sWzAsM10sWzEsMl0sWzMsMl1d
	\[\begin{tikzcd}[ampersand replacement=\&]
		{\{(u,y)\mid\phi(x_0u,y)\}} \& {\{(x,y)\mid\phi(x,y)\}} \\
		U \& {\{x\mid\exists y\phi(x,y)\}}
		\arrow[from=1-1, to=2-1]
		\arrow[from=1-1, to=1-2]
		\arrow[from=2-1, to=2-2]
		\arrow[from=1-2, to=2-2]
	\end{tikzcd}\]
	因为右边为满射, 所以左边 $\{(u,y)\mid \phi(x_0u,y)\}\to U$ 为满射.
\end{proof}

\subsection{层语义}

Kripke--Joyal 语义在景上 (特别地, 在拓扑空间上) 有更直观的表述. 对于景 $(\mathcal C,J)$, 将 $\mathcal C$ 的一个对象 $c$ 想象为景的 ``开集'', 我们将定义记号 $c\forces\phi$ 表示一个公式 $\phi$ 在这个 ``开集'' 上每处 ``局部地'' 成立, 而 ``局部'' 的含义与覆盖结构 $J$ 有关.

\begin{propdef}
	{}
	设 $\phi(x)$ 是 Grothendieck \topos{} $\operatorname{Sh}(\mathcal C,J)$ 的内语言中含一个变量 $x\in X$ 的公式. 记 $\{x\mid\phi(x)\}$ 为 $\operatorname{Sh}(\mathcal C,J)$ 中 $\phi$ 确定的 $X$ 的子对象.
	对于 $\mathcal C$ 的对象 $c$ 与 $x_0\in X(c)$,
	如下条件等价,
	\begin{enumerate}[(1)]
		\item $x_0\in \{x\mid\phi(x)\}(c)$;
		\item $\widehat {\mathcal C}$ 中的态射 $x_0\colon \yo(c) \to X$ 穿过 $\{x\mid\phi(x)\}\hookrightarrow X$;
		\item $\operatorname{Sh}(\mathcal C,J)$ 中的态射 $x_0\colon \yo_{(\mathcal C,J)}(c) \to X$ 穿过 $\{x\mid\phi(x)\}\hookrightarrow X$, 即 $\yo_{(\mathcal C,J)}(c)\forces \phi(x_0)$.
	\end{enumerate}
	当上述条件成立时, 记 $c\forces \phi(x_0)$.
\end{propdef}
\begin{proof}
	$(1)\Leftrightarrow (2)$ 是米田引理, $(2)\Leftrightarrow (3)$ 是由于自反子范畴的性质.
\end{proof}

\begin{prop}
	{(层语义的局部性)}
	若存在 $c$ 的覆盖 (筛) $\{f_i\colon c_i \to c\}$ 使得对每个 $i$ 都有 $c_i\forces \phi(x_0f_i)$, 那么 $c\forces \phi(x_0)$.
\end{prop}

\begin{proof}
	取 $f_i$ 生成的筛 $S\to \yo(c)$, 则有满射 $\coprod_i \yo(c_i) \to S$. 层化后有满射 $\coprod_i \yo_{(\mathcal C,J)}(c_i) \to \yo_{(\mathcal C,J)}(c)$ (层化保持满射, 命题 \ref{adjoints-preserve-mono-epi}). 于是由命题 \ref{epi-forcing} 即证.
\end{proof}

\begin{propdef}
	[label={sheaf-semantics-inductive}]
	{(层语义)}
	设 $X$ 是 $\operatorname{Sh}(\mathcal C,J)$ 的对象, $\phi(x)$ 是 $X$ 上的公式, $c$ 是 $\mathcal C$ 的对象, $x_0 \in X(c)$. 如下条款递归地给出了 $c\forces\phi(x_0)$ 的定义 (``按公式 $\phi$ 的复杂度归纳''):
	\begin{enumerate}[(1)]
		\item 对于公式 $\phi(x), \psi(x)$, $c\forces \phi(x_0) \land \psi(x_0)$ 当且仅当 $c\forces \phi(x_0)$ 且 $c\forces\psi(x_0)$.
		\item 对于公式 $\phi(x), \psi(x)$, $c\forces \phi(x_0) \lor \psi(x_0)$ 当且仅当存在覆盖 $\{f_i\colon c_i\to c\}$, 使得对每个 $i$, $c_i\forces \phi(x_0 f_i)$ 或 $c_i\forces\psi(x_0 f_i)$.
		\item 对于公式 $\phi(x), \psi(x)$, $c\forces \phi(x_0) \Rightarrow \psi(x_0)$ 当且仅当对任意 $p\colon d\to c$, 只要 $d\forces \phi(x_0 p)$, 就有 $d\forces \psi(x_0p)$.
		\item 对于公式 $\phi(x)$, $c\forces \neg\phi(x)$ 当且仅当对任意 $p\colon d\to c$, 只要 $d\forces\phi(x_0p)$, 就有 $d$ 被空集覆盖. (特别地, $c\forces\bot$ 当且仅当 $\varnothing$ 覆盖 $c$.)
		\item \label{sheaf-semantics-exists}对于公式 $\phi(x,y)$, $c \forces \exists y \phi(x_0,y)$ 当且仅当存在覆盖 $\{f_i\colon c_i\to c\}$ 以及元素 $y_i\in Y(c_i)$ 使得对每个 $i$ 有 $c_i\forces \phi(x_0f_i,y_i)$.
		\item 对于公式 $\phi(x,y)$, $c \forces \forall y \phi(x_0,y)$ 当且仅当对任意 $y_0\in Y(d)$, $p\colon d\to c$, 都有 $d\forces \phi(x_0p,y_0)$.
	\end{enumerate}
\end{propdef}
\begin{proof}
	这些性质的证明基本上是翻译命题 \ref{Kripke--Joyal-recursive-definition}. 我们提供一些细节.
	
	在 (2) 中, $c\forces \phi(x_0)\lor\psi(x_0)$ 当且仅当存在 $V,W\in\operatorname{Sh}(\mathcal C,J)$, $p\colon V\to \yo_{(\mathcal C,J)}(c)$ 与 $q\colon W\to \yo_{(\mathcal C,J)}(c)$ 使得 $(p,q)\colon V+W\to \yo_{(\mathcal C,J)}(c)$ 满, 且 $V\forces\phi(x_0p)$, $W\forces\psi(x_0q)$.
	而每个层 $F$ 都是形如 $\yo_{(\mathcal C,J)}(c')$ 的对象的余极限 (命题 \ref{sheaf-as-colimit-of-sheafified-representables}), 特别地, 存在一族对象 $c_i$ 以及满射 $\coprod_i\yo_{(\mathcal C,J)}(c_i) \to F$.
	又因为一族态射 $\{f_i\colon c_i\to c\}$ 构成覆盖当且仅当对应的态射 $\coprod_i\yo_{(\mathcal C,J)}(c_i) \to\yo_{(\mathcal C,J)}(c)$ 为满射 (命题 \ref{cover-sheafified-continued}), 这证明了 (2).
\end{proof}

\begin{example}
	{(一个点上的层语义)}
	一个点上的层\topos{} $\mathsf {Set} \simeq \operatorname{Sh}(*)$ 中 $*\forces\phi(x_0)$ 即是公式 $\phi(x_0)$ 在通常数学的意义下成立.
\end{example}

\begin{example}
	[label={sheaf-semantics-topological-space}]
	{(拓扑空间上的层语义)}
	设 $X$ 为拓扑空间, 那么 $\operatorname{Sh}(X)$ 的内语言中的命题 (即终对象 $1$ 上的公式) 对应于 $X$ 的开集, 直观上这个开集表示该命题成立的范围. 此时对于 $U,V\in\operatorname{Open}(X)$, 有 $U\forces V$ 等价于 $U\leq V$. 例如,
	\begin{itemize}
		\item $\neg U$ 对应开集 $U$ 的补集的内部. 因此, $X\forces (\neg\neg U)$ 当且仅当 $\neg U = \varnothing$, 当且仅当 $U$ 是 $X$ 的稠密开集.
		\item 设 $(X,\mathcal O_X)$ 为环化空间, 即拓扑空间 $X$ 配备 $\operatorname{Sh}(X)$ 中的环 $\mathcal O_X$.
		对于 $f\in O_X$, 命题 $$\internalprop{\text{$f$ 可逆}} := \exists g\,fg=1$$
		直观上表示 ``$f$ 可逆的地方''. 由命题 \ref{sheaf-semantics-inductive} (\ref{sheaf-semantics-exists}), 对于开集 $U$, $U\forces \internalprop{\text{$f$ 可逆}}$ 当且仅当存在 $U$ 的开覆盖 $\{U_i\}$ 以及 $g_i\in \mathcal O_X(U_i)$ 使得 $U_i\forces (fg_i = 1)$.
		\item 所谓\emph{局部环化空间} (locally ringed space) 是指一个环化空间 $(X,\mathcal O_X)$, 使得 $\mathcal O_X$ 的每个茎 $\mathcal O_{X,x}$ 都是局部环. 可以证明, 这等价于 $\mathcal O_X$ 在内语言中是一个局部环, 即如下公式成立:
		\[
			\forall x\in\mathcal O_X\forall y\in\mathcal O_X\,\internalprop{\text{$x+y$ 可逆}} \Rightarrow \internalprop{\text{$x$ 可逆}} \lor \internalprop{\text{$y$ 可逆}}.
		\]
		%使用这种语言可以简化代数几何中的许多命题以及它们的证明, 见 Ingo Blechschmidt 的博士论文 \cite{ILAG}.
	\end{itemize}
\end{example}

%\begin{example}
%	{(Zariski 景上的层语义)}
%	回忆 Zariski 景 (定义 \ref{zariski-site}) 是有限表现环的范畴的对偶范畴, 配备如下覆盖结构: 对于环 $A$ 中生成单位理想的一族元素 $f_i$, $\operatorname{Spec}A[f_i^{-1}]\to \operatorname{Spec}A$ 为覆盖.
%	
%\end{example}

% Zariski 景上的层语义挪到应用章综合代数几何节讲.

%\todo{stack semantics}

% https://ncatlab.org/nlab/show/stack+semantics

%\todo{Zariski 景}

% https://ncatlab.org/nlab/show/Zariski+site

\section{模态与层化}

将 Lawvere--Tierney 拓扑 (定义 \ref{Lawvere--Tierney-topology}) 视为一种模态 (\ref{appendix-modal-logic}), 可以在内语言中构造其层化. 首先我们以内语言重新定义 Lawvere--Tierney 拓扑. 本节中的 ``集合'' 是指某个固定的\topos{}中的对象.

\begin{definition}
	[label={Lawvere--Tierney-topology-internal-definition}]
	{(Lawvere--Tierney 拓扑, 内语言定义)}
	满足如下条件的映射 $\square\colon \Omega\to\Omega$ 称作\emph{Lawvere--Tierney 拓扑}, 或\emph{模态}:
	\begin{itemize}
		\item $\varphi\Rightarrow \square \varphi$;
		\item $\square\square\varphi \Rightarrow \square\varphi$;
		\item $\square(\varphi \land \psi) = \square\varphi\land \square\psi$.
		\\
		由命题 \ref{internal-LT-equivalent-definition}, 该条件可替换为 $\square(\varphi \Rightarrow \psi) \Rightarrow (\square\varphi \Rightarrow \square\psi)$.
	\end{itemize}
	此处以及后文中我们省略了公式中的任意量词 $\forall\varphi\forall\psi$.
\end{definition}

%对于景的 Grothendieck 拓扑给出的 Lawvere--Tierney 拓扑, $\square \varphi$ 的直观是 ``$\varphi$ 在局部上成立'' (注 \ref{Lawvere--Tierney-topology-as-modality}).
一般地, $\square \varphi$ 是 $\varphi$ 的某种弱化.

\begin{prop}
	[label={internal-LT-equivalent-definition}]
	{(Lawvere--Tierney 拓扑的等价定义)}
	对于映射 $\square\colon\Omega\to\Omega$, 条件 $\square(\varphi \land \psi) = \square\varphi\land \square\psi$ 等价于
	%对任意 $\varphi,\psi \colon \Omega$, 若 $\varphi \Rightarrow \psi$, 则 $\square\varphi \Rightarrow \square \psi$.
	$\square(\varphi \Rightarrow \psi) \Rightarrow (\square\varphi \Rightarrow \square\psi)$.%, 即 $\square$ 保持偏序 $\Rightarrow$.
\end{prop}
\begin{proof}~
	\begin{itemize}
		\item 假设 $\square(\varphi \land \psi) = \square\varphi\land \square\psi$.
		要证明
		$\square(\varphi\Rightarrow\psi)\Rightarrow (\square\varphi \Rightarrow \square\psi)$,
		只需证明
		$$\big(\square(\varphi\Rightarrow\psi)\land \square\varphi\big) \Rightarrow \square\psi.$$
		由假设,
		\begin{align*}
			\big(\square(\varphi\Rightarrow\psi)\land\square\varphi\big) &=
			\square\big(
			(\varphi\Rightarrow\psi)\land \varphi
			\big)
			\\
			&=\square (\psi\land\varphi)\\
			&=\square\psi\land\square\varphi\\
			&\Rightarrow \square\psi.
		\end{align*}
		\item 另一方面, 假设 $\square(\varphi\Rightarrow\psi)\Rightarrow (\square\varphi \Rightarrow \square\psi)$.
		由于 $\varphi\land \psi\Rightarrow \varphi$,
		有 $\square(\varphi \land\psi)\Rightarrow\square\varphi$, 同理有 $\square(\varphi \land\psi)\Rightarrow\square\psi$, 故
		$\square (\varphi\land\psi)\Rightarrow(\square\varphi\land\square\psi)$.
		下面要证明 $$(\square\varphi\land\square\psi)\Rightarrow\square (\varphi\land\psi),$$
		即
		$$\square\varphi\Rightarrow\big(\square\psi\Rightarrow\square (\varphi\land\psi)\big).$$
		由假设, 这只需证明
		$$\square\big(\varphi\Rightarrow\big(\psi\Rightarrow(\varphi\land\psi)\big)\big).$$
		上式显然成立.
	\end{itemize}
\end{proof}

作为命题 \ref{internal-LT-equivalent-definition} 的推论, 我们得到 Lawvere--Tierney 拓扑是单调递增的:
$$
(\varphi\Rightarrow\psi)\Rightarrow (\square\varphi\Rightarrow\square\psi).
$$

\begin{example}
	{}
	在 $\mathsf {Set}$ (即经典逻辑) 中只有两个模态, $\square \varphi = \varphi$ 以及 $\square \varphi = \top$.
\end{example}

\begin{example}
	[label={internal-Lawvere--Tierney-examples}]
	{}
	在\topos{}中, 如下映射都是 Lawvere--Tierney 拓扑:
	\begin{itemize}
		\item $\square\varphi = (\mu\Rightarrow\varphi)$, 其中 $\mu$ 是一个固定的命题 (即类型 $\Omega$ 的常量);
		\item $\square\varphi = (\nu\lor\varphi)$, 其中 $\nu$ 是一个固定的命题;
		\item $\square\varphi = \neg\neg\varphi$, 见定义 \ref{double-negation-LT}.%(其中用到 $\neg\neg\neg\neg\varphi \Rightarrow \neg\neg\varphi$, 这是因为 $\neg\neg\neg\varphi\Rightarrow\neg\varphi$, 见命题 \ref{xyyy-xy}).
	\end{itemize}
\end{example}

\begin{definition}
	{(分离性与层条件, 内语言定义)}
	对于给定的 Lawvere--Tierney 拓扑 $\square$, 称集合 $F$ ``$\square$-\emph{分离}'' 是指如下公式成立:
	$$
	\forall x,y\in F,\quad \square (x=y) \Rightarrow x=y.
	$$
	称集合 $F$ 为 $\square$-\emph{层}是指 $F$ $\square$-分离, 且如下公式成立:
	$$
	\forall S \subset F, \quad \square (\internalprop{\text{$S$ 是单元集}}) \Rightarrow \exists x\in F.\,\square(x\in S).
	$$
	(其中 $\forall S\subset F$ 是 $\forall S\in\Omega^F$ 的意思.)
\end{definition}

若 $\square$ 的直观是某命题局部上成立, 那么 $F$ $\square$-分离的直观就是 ``若 $F$ 的两个元素 $x,y$ 在局部上相等, 则它们相等'';
$F$ 为 $\square$-层的直观就是 ``若子集 $S\subset F$ 在局部上是单元集, 则存在 $F$ 的元素 $x$ 局部上属于 $S$''.

\begin{definition}
	{($+$ 构造)}
	对任意集合 $F$, 定义 $F^+$ 为如下商集 (\ref{quotient-set-internal}):
	\[
	F^+ = \{S\subset F \mid \square\internalprop{\text{$S$ 为单元集}} \} /{\sim},\quad\text{其中}\quad S\sim T :\Leftrightarrow \square (S=T).
	\]
	典范的映射 $F\to F^+$ 由 $x\mapsto [\{x\}]$ 给出.
\end{definition}

注意 $S\sim T \Rightarrow \big( \square(x\in S) \Leftrightarrow \square(x\in T) \big)$.

\begin{prop}
	{}
	对任意集合 $F$,
	\begin{itemize}
		\item $F^+$ $\square$-分离;
		\item 若 $F$ $\square$-分离, 则 $F^+$ 为 $\square$-层.
	\end{itemize}
\end{prop}
\begin{proof}~
	\begin{itemize}
		\item 要证明 $F^+$ $\square$-分离, 即如下公式在 $F^+$ 上恒成立:
		\[
		\square(x=y) \Rightarrow x=y.
		\]
		在内语言中这个命题是明显的: 由 $F^+$ 的定义, 有
		\[
			\begin{aligned}
				&\forall S,T\in \{S\subset F \mid \square\internalprop{\text{$S$ 为单元集}} \}\\
				& \square([S] = [T]) \Leftrightarrow \square \square(S = T) \Leftrightarrow \square(S=T)\Leftrightarrow [S]=[T].
			\end{aligned}
		\]
		\item \footnotemark{}要证明 $F^+$ 为层, 即证明如下公式成立:
		\[
		\forall S \subset F^+,\quad\square(\internalprop{\text{$S$ 是单元集}}) \Rightarrow\exists \alpha\in F^+\,\alpha\in S.
		\]
		定义\emph{压平} $f\colon PF^+\to PF$ 如下:
		\[
			f(S) := \{x\in F \mid \exists [A]\in S\, \square(x\in A)\}.
		\]
		这里, $\exists [A]\in S$ 是 $\exists A\,\big([A]\in S\land\cdots\big)$ 的意思.
		其中 ``压平'' 的直观是将多层的 $\{\,\}$ 替换为单层.
		例如, 当 $S$ 为单元集 $\big\{[\{a\}]\big\}$ 时, $f(S)=\{a\}$, 这是因为
		\[
		\begin{aligned}
			\exists [A]\in S\,\square (x\in A)
			&\Rightarrow
			\exists A\,\big([A]=[\{a\}]\big)\land \square(x\in A)\\
			&\Rightarrow
			\exists A\,\square(A=\{a\}\land x\in A)\\
			&\Rightarrow \square(x=a)\\
			&\Rightarrow x=a\,(\text{由 $\square$-分离性}).
		\end{aligned}
		\]
		
		%我们将使用 $\square$ 的一个重要性质: 要证明 $\square\phi \Rightarrow \square \psi$, 只需证明 $\phi\Rightarrow\square\psi$.
		由单元集的定义,
			\[
			\internalprop{\text{$S$ 为单元集}} \Rightarrow \exists  A\subset F \,\big(\square\internalprop{\text{$A$ 为单元集}} \land S=\{[A]\}\big).
			\]
		而
			\[
			\begin{aligned}
				\internalprop{\text{$A$ 为单元集}} \land S=\{[A]\}
				&\Rightarrow
				\exists a\in F\,S=\big\{[\{a\}]\big\}\\
				&\Rightarrow
				\exists a\in F\,f(S) = \{a\}\\
				&\Rightarrow
				\internalprop{\text{$f(S)$ 为单元集}}.
			\end{aligned}
			\]
		综合上述论证, 使用模态 $\square$ 的性质, 我们证明了
		\[
		\square(\internalprop{\text{$S$ 为单元集}}) \Rightarrow \square(\internalprop{\text{$f(S)$ 为单元集}}).
		\]
	\end{itemize}
	%
	%\todo{证明}
\end{proof}
\footnotetext{此证明来自杨家同.}

%\section{几何理论}



%\todo{将综合可计算性理论, 综合微分几何, 综合代数几何变成本章的节}

\begin{prop}
	[label={sheaf-subobject-classifier-internal-definition}]
	{(层范畴的子对象分类子, 内语言刻画)}
	设 $\square\colon \Omega\to\Omega$ 是 Lawvere--Tierney 拓扑, 则 $\square$-层范畴的子对象分类子为
	$$
	\Omega_{\square} = \{p\in\Omega\mid \square p = p\}.
	$$
\end{prop}

\section{内位象}

\begin{prop}
	[label={internal-frame}]
	{(内\fm{})}
	设 $\mathcal S$ 为\topos{}. 定义 $\mathcal S$ 的\emph{内\fm{}} (internal frame) 为 $\mathcal S$ 的内语言中定义的\fm{} (定义 \ref{frame-definition}). 具体地, 一个内\fm{}包含如下信息:
	\begin{itemize}
		\item 一个对象 $A$;
		\item 运算 $\land\colon A\times A\to A$, 以及 $\bigvee\colon PA\to A$, 满足 ``有限交'' 和 ``任意并'' 的条件;
		\item 结合律
		\[
		\forall a\in A\,\forall I\in PA\,\,\Big(a\land \bigvee I = \bigvee \{a\land b\mid b\in I\}\Big).
		\]
	\end{itemize}
\end{prop}

\todo{Elephant C1.6}



