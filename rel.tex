\chapter{相对\topos{}}

\minitoc

在相对的观点下, \topos{}之间的几何态射 $\mathcal E\to \mathcal S$, 即 ``相对于'' $\mathcal S$ 的 \topos{}, 可视为 $\mathcal E$ 是某个景上的 $\mathcal S$-值层范畴. 当然, 这里的 ``景'' 与 ``层'' 的概念是在 $\mathcal S$ 的内语言中谈论的.

% Joyal--Tierney EGG

% internal locale 写在内语言那一章

%\section{位象}
%
%\begin{definition}
%	{(位象)}
%	
%\end{definition}

\section{}

\begin{definition}
	{(旋子)}
	设 $\mathcal S$ 为\topos{}, $\mathcal C$ 为 $\mathcal S$ 的内蕴范畴, $p\colon \mathcal E\to\mathcal S$ 为相对\topos{}. 定义 $\mathcal E$ 中的 $\mathcal C$-旋子为 $\mathcal S$-索引函子 $F\colon \mathcal C^{\op}\to\mathcal E$, 使得
\end{definition}

\begin{prop}
	{(Diaconescu 定理)}
	设 $\mathcal S$ 为\topos{}, $\mathcal C$ 为 $\mathcal S$ 的内蕴范畴, $p\colon \mathcal E\to\mathcal S$ 为相对\topos{}. 那么有范畴等价
	\[
	\operatorname{Hom}_{\Top/\mathcal S}
	(\mathcal E,\mathsf {Fun}(\mathcal C,\mathcal S))
	\simeq \mathsf {Tors}(\mathcal C,\mathcal E).
	\]
\end{prop}

