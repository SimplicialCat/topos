\chapter{$\infty$-范畴论的补充知识}

\section{$\infty$-范畴的单纯集模型}

粗略地说, 一个 $\infty$-范畴含有如下成分: 对象, 对象之间的态射, 态射之间的 $2$-态射, $\cdots$, $k$-态射之间的 $(k+1)$-态射, 以至于无穷. 我们使用的 $\infty$-范畴又称 $(\infty ,1)$-范畴, 意为对所有 $k>1$, $k$-态射都可逆.

在实践中, $\infty$-范畴有许多不同而可以互相转化的\emph{模型}, 就像一个算法由许多不同的编程语言实现. 单纯集就是一种实用的 ``编程语言'', 它提供了从集合开始模拟出 $\infty$-范畴的方法.

\begin{definition}
	[label={simplicial-set-horn}]
	{(角形)}
	回忆单纯集 $\Delta^n$ 为 $[n]\in\Delta$ 在米田嵌入下的像 $\yo([n])$. 对于单射 $[m]\to [n]$, 设其像为 $J$, 定义单纯集 $\Delta^J$ 为对应的态射 $\Delta^m\to\Delta^n$ 的像 (作为 $\Delta^n$ 的子对象). 对于 $0\leq k \leq n$, 定义
	$$
	\Lambda_k^n : = \bigcup_{k\in J\neq [n]}\Delta^J,
	$$
	称为\emph{角形} (horn). 其中对应 $0<k<n$ 的角形称为\emph{内角形} (inner horn).
\end{definition}

角形是用来描述单纯集模型 $\infty$-范畴中一些结构的图形. 如下是角形 $\Lambda_k^2$ ($k=0,1,2$) 的示意图. 可以看到它们是互不同构的单纯集 (尽管它们的几何实现都是互相同胚的), 其中内角形 $\Lambda_1^2$ 中的两个箭头可以复合, 而 $\Lambda_0^2,\Lambda_2^2$ 中的箭头不能复合.
% https://q.uiver.app/#q=WzAsMTIsWzAsMiwiMCJdLFsxLDAsIjEiXSxbMiwyLCIyIl0sWzEsMywiXFxMYW1iZGFfMF4yIl0sWzMsMiwiMCJdLFs0LDAsIjEiXSxbNSwyLCIyIl0sWzQsMywiXFxMYW1iZGFfMV4yIl0sWzcsMywiXFxMYW1iZGFfMl4yIl0sWzYsMiwiMCJdLFs4LDIsIjIiXSxbNywwLCIxIl0sWzAsMV0sWzAsMl0sWzQsNV0sWzUsNl0sWzksMTBdLFsxMSwxMF1d
\[\begin{tikzcd}[ampersand replacement=\&,column sep=0, row sep=0.8em]
	\& 1 \&\&\& 1 \&\&\& 1 \\
	\\
	0 \&\& 2 \& 0 \&\& 2 \& 0 \&\& 2 \\
	\& {\Lambda_0^2} \&\&\& {\Lambda_1^2} \&\&\& {\Lambda_2^2}
	\arrow[from=3-1, to=1-2]
	\arrow[from=3-1, to=3-3]
	\arrow[from=3-4, to=1-5]
	\arrow[from=1-5, to=3-6]
	\arrow[from=3-7, to=3-9]
	\arrow[from=1-8, to=3-9]
\end{tikzcd}\]

\begin{definition}
	[label={infinity-category-definition}]
	{($\infty$-范畴)}
	\emph{$\infty$-范畴} (又称\emph{拟范畴}) 是满足如下条件的单纯集 $\mathcal X$: 对所有整数 $0<k<n$,
	$$
	\operatorname{Hom}(\Delta^n,\mathcal X) \to \operatorname{Hom}(\Lambda_k^n,\mathcal X)
	$$
	是满射; 换言之, 如下提升总存在 (但\emph{不要求唯一}):
	\[\begin{tikzcd}[ampersand replacement=\&]
		{\Lambda_k^n} \& {X.} \\
		{\Delta^n}
		\arrow[from=1-1, to=2-1]
		\arrow[from=1-1, to=1-2]
		\arrow["\exists"', dashed, from=2-1, to=1-2]
	\end{tikzcd}\]
	称之为内角形的\emph{填充} (filler).
	
	设单纯集 $\mathcal X$ 是 $\infty$-范畴. 定义
	\begin{itemize}
		\item $\mathcal X$ 中的\emph{对象}为 $X_0$ 的元素, 即单纯集映射 $\Delta^0\to X$;
		\item $\mathcal X$ 中的\emph{态射} (\emph{箭头}) 为 $X_1$ 的元素, 即单纯集映射 $\Delta^1\to X$, 对象 $x$ 上的恒等态射 $\operatorname{id}_x$ 为映射 $\Delta^1\to \Delta^0\overset{x}{\to}X$;
		\item $\mathcal X$ 中的\emph{实心三角形}为 $X_2$ 的元素, 即单纯集映射 $\Delta^2\to X$, 对于实心三角形 $\begin{tikzcd}[ampersand replacement=\&,column sep=0,row sep=0.8em]
			\& y \\
			x \&\& z,
			\arrow["f", from=2-1, to=1-2]
			\arrow["g", from=1-2, to=2-3]
			\arrow["h"', from=2-1, to=2-3]
		\end{tikzcd}$
		称 $h$ 为 $f$ 与 $g$ 的一个\emph{复合}. (很明显, 复合不是唯一的.) 若 $\operatorname{id}_x$ 为 $f,g$ 的一个复合, 则称 $g$ 为 $f$ 的一个\emph{逆}. (当然, 逆也不是唯一的.)
	\end{itemize}
\end{definition}

\begin{definition}
	{(单纯集的对偶)}
	考虑函子 $(-)^{\op} \colon \Delta \to \Delta$.
	对于单纯集 $\mathcal X$, 定义其\emph{对偶} $\mathcal X^{\op}$ 为 $\mathcal X^{\op}:= \mathcal X\circ (-)^{\op}\colon \Delta \to\mathsf {Set}$.
\end{definition}

\begin{example}
	{(普通范畴的脉)}
	回忆一个普通范畴 $\mathcal C$ 的脉 $\text{N}(\mathcal C)$ (例 \ref{sset-geometric-realization}) 定义如下,
	$$
	\text{N}\mathcal{C}_n = \mathsf {Fun}(0\to 1\to \cdots \to n,\mathcal C),
	$$
	即 $\text{N}(\mathcal C)_n$ 的元素是 $\mathcal C$ 中连续的 $n$ 个箭头. 由于对任意 $0<k<n$, $\Lambda_k^n$ 都包含一条折线 $0\to 1\to\cdots\to n$, 故映射 $\Lambda_k^n\to \text{N}(\mathcal C)$ 总能提升为 $\Delta^n\to \text{N}(\mathcal C)$, $\text{N}(\mathcal C)$ 是一个 $\infty$-范畴.
\end{example}


\begin{remark}
	{($\infty$-范畴单纯集模型的注意事项)}
	定义 \ref{infinity-category-definition} 中有两点需要注意; 如果忽视这两点, 就会得到另外两种东西.
	\begin{itemize}
		\item 只有内角形可以填充. 若所有角形都可以填充, 则可证明 $\mathcal X$ 的所有态射都可逆, 我们称之为 \emph{$\infty$-群胚}.
		\item 内角形填充不要求唯一. 若内角形填充存在且唯一, 则 $\mathcal X$ 实际上来自一个普通范畴的脉. 直观上, $\infty$-范畴是一种 ``弱化'' 的范畴, 其中的复合是在同伦意义下谈论的. 可以证明\footnotemark{}, 对 $\infty$-范畴中的两个态射 $f\colon x\to y, g\colon y\to z$, 其所有可能的复合构成一个\emph{可缩 Kan 复形} (定义 \ref{equivalence-contractible}).
	\end{itemize}
	因此, $\infty$-范畴可视为普通范畴与 $\infty$-群胚的共同推广.
\end{remark}
\footnotetext{\url{https://kerodon.net/tag/0078}}

%\begin{prop}
%	{}
%	对于单纯集 $X$, 以下条件等价:
%	\begin{itemize}
	%		\item
	%		对所有整数 $0<k<n$,
	%		$
	%		\operatorname{Hom}(\Delta^n,X) \to \operatorname{Hom}(\Lambda_k^n,X)
	%		$
	%		是双射;
	%		\item
	%		对所有整数 $0<n$,
	%		$
	%		\operatorname{Hom}(\Delta^n,X) \to \operatorname{Hom}(\operatorname{N}(0\to 1 \to \cdots\to n),X)
	%		$
	%		是双射;
	%		\item
	%		$X$ 同构于某个范畴的脉.
	%	\end{itemize}
%\end{prop}
%
%第二个条件表明单纯集 $X$ 完全由 $X_0,X_1$ 即 ``$0$ 维和 $1$ 维的信息'' 决定.

\begin{propdef}
	{($\infty$-群胚, Kan 复形模型)}
	定义 \emph{$\infty$-群胚} (又称 \emph{Kan 复形}) 是满足如下等价条件之一的 $\infty$-范畴 $\mathcal X$:
	\begin{itemize}
		\item $\mathcal X$ 中所有态射都可逆;
		\item $\mathcal X$ 中所有角形都可填充, 即对所有整数 $0\leq k\leq n$,
		$$
		\operatorname{Hom}(\Delta^n,X) \to \operatorname{Hom}(\Lambda_k^n,X)
		$$
		是满射.
	\end{itemize}
\end{propdef}

\begin{proof}
	\todo{}
\end{proof}

\begin{example}
	{(基本 $\infty$-群胚)}
	拓扑空间 $X$ 的奇异单纯集 $\operatorname{Sing}X$ 是 $\infty$-群胚, 称为其\emph{基本 $\infty$-群胚} $\pi_{\infty}(X)$.
\end{example}

\begin{propdef}
	{(函子, 函子范畴)}
	定义 $\infty$-范畴之间的\emph{函子}为单纯集的映射.
	对于 $\infty$-范畴 $\mathcal X$ 与任意单纯集 $A$, 单纯集的指数对象 ${\mathcal X}^A$ 都是 $\infty$-范畴. 特别地, 对于无穷范畴 $\mathcal X,\mathcal Y$ 定义\emph{函子范畴} $\mathsf {Fun}(\mathcal X,\mathcal Y) := {\mathcal Y}^{\mathcal X}$. 函子范畴中的态射称为\emph{自然变换} (换言之, 两个函子 $\mathcal X\to \mathcal Y$ 之间的自然变换是单纯集映射 $\Delta^1\times \mathcal X \to \mathcal Y$).
\end{propdef}

\begin{definition}
	[label={equivalence-contractible}]
	{(范畴等价, 可缩)}
	对于 $\infty$-范畴 $\mathcal X,\mathcal Y$ 之间的函子 $u\colon \mathcal X\to \mathcal Y$, 称 $u$ 为一个\emph{等价}是指存在函子 $v\colon \mathcal Y\to \mathcal X$, 以及两个可逆的自然变换 $uv\to \operatorname{id}_{\mathcal Y}$, $\operatorname{id}_{\mathcal X}\to vu$.
	称等价于 $\Delta^0$ 的 $\infty$-范畴是\emph{可缩}的.
\end{definition}

\begin{definition}
	{(态射集)}
	对于 $\infty$-范畴 $\mathcal X$ 以及其中的对象 $x,y$, 定义单纯集 $\operatorname{Hom}_{\mathcal X}(x,y)$ 为如下的拉回.
	\[\begin{tikzcd}[ampersand replacement=\&,column sep=1em,row sep=2em]
		{\hspace{-1em}\operatorname{Hom}_{\mathcal X}(x,y)} \& {\mathsf {Fun}(\Delta^1,\mathcal X)\hspace{-1em}} \\
		{\Delta^0} \& {\mathcal X\times \mathcal X}
		\arrow["{(s,t)}", from=1-2, to=2-2]
		\arrow["{(x,y)}"', from=2-1, to=2-2]
		\arrow[from=1-1, to=2-1]
		\arrow[from=1-1, to=1-2]
	\end{tikzcd}\]
	其中 $s,t\colon \mathsf {Fun}(\Delta^1,\mathcal X)\to \mathcal X$ 将 $\mathcal X$ 的态射对应到其起点与终点.
	
	更一般地, 对 $\mathcal X$ 中 $n+1$ 个对象 $x_0,x_1,\cdots,x_n$, 定义 $\operatorname{Hom}_{\mathcal X}(x_0,x_1,\cdots,x_n)$ 为如下的拉回.
	\[\begin{tikzcd}[ampersand replacement=\&,column sep=1em,row sep=2em]
		{\hspace{-1em}\operatorname{Hom}_{\mathcal X}(x_0,\cdots,x_n)} \& {\mathsf {Fun}(\Delta^n,\mathcal X)\hspace{-1em}} \\
		{\Delta^0} \& {\mathcal X^{n+1}}
		\arrow["{}", from=1-2, to=2-2]
		\arrow["{(x_0,\cdots,x_n)}"', from=2-1, to=2-2]
		\arrow[from=1-1, to=2-1]
		\arrow[from=1-1, to=1-2]
	\end{tikzcd}\]
\end{definition}

如果 $\infty$-群胚是空间的模型, 那么 $\operatorname{Hom}_{\mathcal X}(x,y)$ 就是两点 $x,y$ 之间的道路的空间.

\begin{example}
	{}
	设 $\mathcal X$ 为 $\infty$-群胚, $x$ 为其中的对象, 那么单纯集 $\operatorname{Hom}_{\mathcal X}(x,x)$ 在同伦论上又叫\emph{环路空间} $\Omega(\mathcal X,x)$, 其连通分支的集合给出\emph{基本群} $\pi_1({\mathcal X},x)$.
\end{example}

如下命题表示我们考虑的 $\infty$-范畴中 ``$k$-态射都可逆'' ($k>1$).

\begin{prop}
	{}
	对 $\infty$-范畴 $\mathcal X$ 中的任意两个对象 $x,y$, $\operatorname{Hom}_{\mathcal X}(x,y)$ 是 $\infty$-群胚.
\end{prop}

\subsection{同伦}


\begin{propdef}
	{(态射的同伦, 同伦范畴)}
	对于两个态射 $f,g\colon x\to y$, 称 $f$ \emph{同伦}于 $g$ 是指 $g$ 为 $f$ 与 $\operatorname{id}_y$ 的一个复合; 记 $f\sim g$. 态射的同伦为等价关系.
	
	对于 $\infty$-范畴 $\mathcal X$, 定义其\emph{同伦范畴} $\Ho{}(\mathcal X)$ 为如下的范畴: $\Ho(\mathcal X)$ 的对象即为 $\mathcal X$ 的对象, 态射为 $\mathcal X$ 中态射的同伦类, 态射的复合是良定义的.
\end{propdef}

% HTT 1.2.3.1

\begin{proof}
	
	我们证明同伦是一个等价关系: 由下面的示意图以及 $\infty$-范畴的定义, 可知若 $f\sim g$ 则 $g\sim f$.
	\[\begin{tikzcd}[ampersand replacement=\&,sep=1.6em]
		x \&\& y \\
		\\
		y \&\& y
		\arrow["f"', from=1-1, to=3-1]
		\arrow["{\operatorname{id}_y}"', from=3-1, to=3-3]
		\arrow["g\!"'{pos=0.7}, from=1-1, to=3-3]
		\arrow["{\operatorname{id}_y\!\!\!\!}"{pos=0.7}, from=3-1, to=1-3]
		\arrow["{\operatorname{id}_y}"', from=3-3, to=1-3]
		\arrow["f", from=1-1, to=1-3]
	\end{tikzcd}\]
	\vspace{-2em}
\end{proof}

有趣的是, 对于 $\infty$-范畴 $\mathcal X$ (视为单纯集), 同伦范畴 $\operatorname{Ho}(\mathcal X)$ 正是 $\mathcal X$ 在 $\mathsf {Cat}$ 中的几何实现 (\tomeun{}例 \ref{t1-sset-geometric-realization}). 它是将 $\infty$-范畴 ``截断'' 为 $1$-范畴的结果.

\begin{prop}
	{}
	设 $\mathcal X$ 为 $\infty$-范畴, 考虑函子 $\mathcal X\to \text{N}(\Ho{}(\mathcal X))$ 将 $\mathcal X$ 的点映射到 $\Ho{}(\mathcal X)$ 的对象, $n$-单形映射到 $\Ho{}(X)$ 的连续 $n$ 个态射. 那么这个函子给出了范畴的同构
	$$
	|\mathcal X|\simeq\Ho{}(\mathcal X),
	$$
	其中 $|{-}|$ 是\tomeun{}例 \ref{t1-sset-geometric-realization} 提到的脉函子的左伴随.
\end{prop}

同伦范畴还可通过单纯范畴定义: 由定义 \ref{coherent-nerve}, 对于 $\infty$-范畴 $\mathcal X$, $\mathfrak {C}[\mathcal X]$ 是一个 $\mathsf {sSet}$-充实范畴; 将其中的态射单纯集替换为连通分支便得到同伦范畴. 见 HTT \cite{HTT} 定义 1.1.5.14. 总结起来, 我们有如下图表.
% https://q.uiver.app/#q=WzAsMyxbMiwyLCJcXG1hdGhzZiB7c0NhdH0iXSxbMCwwLCJcXG1hdGhzZiB7c1NldH0iXSxbNCwwLCJcXG1hdGhzZiB7Q2F0fSJdLFsxLDIsIlxcSG97fSIsMCx7Im9mZnNldCI6LTN9XSxbMiwxLCJcXHRleHR7Tn0iLDAseyJvZmZzZXQiOi0xfV0sWzAsMSwiXFx0ZXh0e059XntcXHRleHR7Y319IiwwLHsib2Zmc2V0IjotM31dLFsxLDAsIlxcbWF0aGZyYWsge0N9W3stfV0iLDAseyJvZmZzZXQiOi0xfV0sWzIsMCwiaSIsMCx7Im9mZnNldCI6LTMsInN0eWxlIjp7InRhaWwiOnsibmFtZSI6Imhvb2siLCJzaWRlIjoiYm90dG9tIn19fV0sWzAsMiwiXFx0ZXh0e2hvfSIsMCx7Im9mZnNldCI6LTF9XSxbMyw0LCIiLDAseyJsZXZlbCI6MSwic3R5bGUiOnsibmFtZSI6ImFkanVuY3Rpb24ifX1dLFs2LDUsIiIsMCx7ImxldmVsIjoxLCJzdHlsZSI6eyJuYW1lIjoiYWRqdW5jdGlvbiJ9fV0sWzgsNywiIiwwLHsibGV2ZWwiOjEsInN0eWxlIjp7Im5hbWUiOiJhZGp1bmN0aW9uIn19XV0=
\[\begin{tikzcd}[ampersand replacement=\&]
	{\mathsf {sSet}} \&\&\&\& {\mathsf {Cat}} \\
	\\
	\&\& {\mathsf {sCat}}
	\arrow[""{name=0, anchor=center, inner sep=0}, "{\Ho{}}", shift left=3, from=1-1, to=1-5]
	\arrow[""{name=1, anchor=center, inner sep=0}, "{\text{N}}", shift left, from=1-5, to=1-1]
	\arrow[""{name=2, anchor=center, inner sep=0}, "{\text{N}^{\text{c}}}", shift left=3, from=3-3, to=1-1]
	\arrow[""{name=3, anchor=center, inner sep=0}, "{\mathfrak {C}[{-}]}", shift left, from=1-1, to=3-3]
	\arrow[""{name=4, anchor=center, inner sep=0}, "i", shift left=3, hook', from=1-5, to=3-3]
	\arrow[""{name=5, anchor=center, inner sep=0}, "{\text{ho}}", shift left, from=3-3, to=1-5]
	\arrow["\dashv"{anchor=center, rotate=-90}, draw=none, from=0, to=1]
	\arrow["\dashv"{anchor=center, rotate=-135}, draw=none, from=3, to=2]
	\arrow["\dashv"{anchor=center, rotate=-45}, draw=none, from=5, to=4]
\end{tikzcd}\]



我们还需要描述 $\infty$-范畴的全子范畴.

%\todo{子范畴 , HTT 1.2.11}

\begin{definition}
	{(全子范畴)}
	设 $\mathcal C$ 是 $\infty$-范畴, $\mathcal S\to \Ho (\mathcal C)$ 是其同伦范畴的子范畴.
	定义 \emph{$\mathcal S$ 张成的 $\mathcal C$ 的全子范畴}为如下 (作为单纯集的) 拉回.
	\[
	\begin{tikzcd}
		\mathcal {S}\times_{\text{N}(\Ho (\mathcal C))} \mathcal C \ar[r]\ar[d]& \mathcal C\ar[d]\\
		\mathcal S \ar[r]& \text{N}(\Ho (\mathcal C))
	\end{tikzcd}
	\]
\end{definition}

\subsection{单纯范畴}

$\infty$-范畴的另一种模型是用单纯范畴描述的, 其优点包括
\begin{itemize}
	\item 用单纯范畴模型方便给出某些具体的 $\infty$-范畴以及函子;
	\item 单纯范畴中态射的复合唯一定义;
\end{itemize}
但这种模型的同伦论较难处理.

\begin{definition}
	{(单纯范畴)}
	\emph{单纯范畴}是指充实于 $\mathsf {sSet}$ 的范畴.
	具体地, 我们有一个对象集合 $\operatorname{Ob}(\mathcal C)$,
	对 $x,y\in\operatorname{Ob}(\mathcal C)$ 有一个单纯集 $\operatorname{Hom}_{\mathcal C}(x,y)$,
	对 $x\in\operatorname{Ob}(\mathcal C)$ 有恒等态射 $\operatorname{id}_x\in\operatorname{Hom}(x,x)$,
	对 $x,y,z\in\operatorname{Ob}(\mathcal C)$ 有单纯集映射
	$$
	\circ\colon \operatorname{Hom}_{\mathcal C}(x,y) \times \operatorname{Hom}_{\mathcal C}(y,z) \to \operatorname{Hom}_{\mathcal C}(x,z),
	$$
	满足结合律与幺元律.
	
	等价地, 单纯范畴也可定义为小范畴范畴 $\mathsf {Cat}$ 中的内蕴单纯集 $\mathcal C\colon \Delta^{\op}\to\mathsf {Cat}$,
	满足 ``对象的单纯集'' $\operatorname{Ob}(\mathcal C) := \operatorname{Ob}\circ\mathcal C\colon \Delta^{\op} \to \mathsf {Set}$
	是常值单纯集.
	
	记 (小) 单纯范畴的范畴为 $\mathsf {sCat}$, 其中的态射是单纯范畴之间的 $\mathsf {sSet}$-充实函子.
\end{definition}

\begin{definition}
	{(纤维性单纯范畴)}
	若单纯范畴 $\mathcal C$ 的态射集 $\operatorname{Hom}_{\mathcal C}(x,y)$ 均为 Kan 复形 (即前面定义的 $\infty$-群胚), 则称之为\emph{纤维性} (fibrant) 单纯范畴; 它是 $\infty$-范畴的另一种模型. 换言之, $\infty$-范畴可视为充实于 $\infty$-群胚的范畴.
\end{definition}

\begin{example}
	{(拓扑空间范畴)}
	拓扑空间范畴 $\mathsf {Top}$ 具有单纯范畴结构:
	$$
	\operatorname{Hom}(X,Y)_n := \{ \text{连续函数 $|\Delta^n| \times X \to Y$ } \}.
	$$
\end{example}

\begin{propdef}
	{($\infty$-范畴的极大子 $\infty$-群胚)}
	设 $\mathcal X$ 为 $\infty$-范畴. 记 $\mathcal X^\simeq$ 为所有边都可逆的单形 $\Delta^n \to \mathcal X$ 构成的子单纯集, 则 $\mathcal X^\simeq$ 为 $\mathcal X$ 的\emph{极大子 $\infty$-群胚}, 即任何 $\infty$-群胚到 $\mathcal X$ 的函子唯一地穿过 $\mathcal X^\simeq$; $\infty$-群胚 $\mathcal X^\simeq$ 又称 $\mathcal X$ 的\emph{核心} (core).
	(关于普通范畴的极大子群胚, 见例 \ref{Gpd-Cat-adjunction}.)
\end{propdef}

\begin{example}
	{($\infty$-范畴的单纯范畴)}
	记 $\infty\mathsf{Cat}$ 为 (小) $\infty$-范畴的范畴 (它是一个普通范畴), 对 $\infty$-范畴 $\mathcal X,\mathcal Y$ 定义 $\operatorname{Hom}_{\infty\mathsf {Cat}}(\mathcal X,\mathcal Y)$ 为 $\mathsf {Fun}(\mathcal X,\mathcal Y)$ 的极大子 $\infty$-群胚; 这样 $\infty\mathsf {Cat}$ 构成一个纤维性单纯范畴.
\end{example}

\begin{example}
	{(链复形范畴)}
	设 $R$ 为环, $\mathsf {Ch}(R)$ 为 $R$ 上的链复形的范畴. 回忆 $R$ 上的\emph{链复形}是指 $R$-模范畴中的一个图表 $$M_\bullet = \cdots \to M_2\overset{\partial}{\to} M_1\overset{\partial}{\to} M_0\overset{\partial}{\to} M_{-1}\overset{\partial}{\to} M_{-2} \to \cdots,$$
	满足 $\partial\circ \partial= 0$.
	$\mathsf {Ch}(R)$ 可赋予单纯范畴结构. 首先构造 $\mathbb{Z}$ 上的链复形 $C_\bullet(\Delta^n)$: 对于 $0\leq k \leq n$ 其第 $k$ 位置是 $\Delta^n$ 的非退化 $k$-单形自由生成的 Abel 群, 边界映射 $\partial:= \sum_{i=0}^k (-1)^i d_i$ 来自单形的面映射 $d_i$. 例如
	$$
	C_\bullet(\Delta^2) = \cdots\to 0\to \mathbb{Z} \to \mathbb{Z}^3 \to \mathbb{Z}^3 \to 0 \to\cdots.
	$$
	(熟悉代数拓扑的读者知道, $C_\bullet(X)$ 就是用于计算单纯同调 $H_\bullet(X)$ 的那个链复形.)
	定义
	$$
	\operatorname{Hom}(M_\bullet,N_\bullet)_n := \operatorname{Hom}_{\mathsf {Ch}(R)}
	(M_\bullet \otimes_{\mathbb{Z}} C_\bullet(\Delta^n),N_\bullet).
	$$
	这个范畴是 (现代的) 代数 $K$-理论的起点. 函子 $C_\bullet\colon \Delta\to\mathsf {Ch}(\mathbb{Z})$ 给出的脉--几何实现伴随限制为单纯 Abel 群与非负位置链复形之间的范畴等价
	\[\begin{tikzcd}[ampersand replacement=\&]
		{\mathsf {sAb}} \& {\mathsf {Ch}_{\geq 0}(\mathbb{Z})}
		\arrow[""{name=0, anchor=center, inner sep=0}, "|{-}|_{C_\bullet}", shift left=2, from=1-1, to=1-2]
		\arrow[""{name=1, anchor=center, inner sep=0}, "\operatorname{N}", shift left=2, from=1-2, to=1-1]
		\arrow["\simeq"{anchor=center}, draw=none, from=0, to=1]
	\end{tikzcd},
	\]
	称为 \emph{Dold--Kan 对应}.
\end{example}

\begin{definition}
	{(偏序集的道路范畴)}
	对于偏序集 $(S,\leq)$, 定义其\emph{道路范畴}为一个单纯范畴 $\operatorname{Path}(S)$,
	其对象集为 $S$, 对两个元素 $x,y\in S$, 单纯集 $\operatorname{Hom}_{\operatorname{Path}(S)}(x,y)$ 是 ``由 $x$ 到 $y$ 道路的空间'',
	它定义为所有形如 $\{x=x_0\leq x_1\leq\cdots \leq x_m=y\}$ 的链构成的偏序集的脉, 其序关系为包含关系的反序 (最大元为 $x\leq y$).
	$\operatorname{Path}(S)$ 中态射的复合即是链的并.
\end{definition}

% https://q.uiver.app/#q=WzAsOCxbMCwwLCIwIl0sWzEsMSwiMSJdLFsyLDAsIjIiXSxbMSwyLCJTIl0sWzQsMiwiXFxvcGVyYXRvcm5hbWV7UGF0aH0oUykiXSxbMywwLCIwIl0sWzQsMSwiMSJdLFs1LDAsIjIiXSxbMCwxXSxbMSwyXSxbMCwyXSxbNSw2LCIwXFxsZXEgMSIsMix7Im9mZnNldCI6MX1dLFs2LDcsIjFcXGxlcSAyIiwyLHsib2Zmc2V0IjoxfV0sWzUsNywiMFxcbGVxIDIiLDAseyJvZmZzZXQiOi0xLCJjdXJ2ZSI6LTF9XSxbNSw3LCIwXFxsZXEgMVxcbGVxIDIiLDIseyJjdXJ2ZSI6MX1dLFsxNCwxMywiIiwxLHsic2hvcnRlbiI6eyJzb3VyY2UiOjIwLCJ0YXJnZXQiOjIwfX1dXQ==
\[\begin{tikzcd}[ampersand replacement=\&]
	0 \&\& 2 \& 0 \&\& 2 \\
	\& 1 \&\&\& 1 \\
	\& S \&\&\& {\operatorname{Path}(S)}
	\arrow[from=1-1, to=2-2]
	\arrow[from=2-2, to=1-3]
	\arrow[from=1-1, to=1-3]
	\arrow["{0\leq 1}"', shift right, from=1-4, to=2-5]
	\arrow["{1\leq 2}"', shift right, from=2-5, to=1-6]
	\arrow[""{name=0, anchor=center, inner sep=0}, "{0\leq 2}", shift left, bend left=10, from=1-4, to=1-6]
	\arrow[""{name=1, anchor=center, inner sep=0}, "{0\leq 1\leq 2}"', bend right=10, from=1-4, to=1-6]
	\arrow[shorten <=2pt, shorten >=2pt, Rightarrow, from=1, to=0]
\end{tikzcd}\]

\newcommand{\Nc}{\operatorname{N}^{\text{c}}}
\begin{definition}
	[label={coherent-nerve}]
	{(单纯范畴的融贯脉)}
	考虑函子 $\operatorname{Path}\colon \Delta \to \mathsf {sCat}$,
	其对应的脉--几何实现伴随 (命题 \ref{nerve-and-realization}, 但要使用 $\mathsf {sSet}$-充实版本)
	\[
	\begin{tikzcd}[ampersand replacement=\&]
		{\mathsf {sSet}} \& {\mathsf {sCat}}
		\arrow[""{name=0, anchor=center, inner sep=0}, "{\mathfrak {C}[{-}]}", shift left=2, from=1-1, to=1-2]
		\arrow[""{name=1, anchor=center, inner sep=0}, "{\Nc}", shift left=2, from=1-2, to=1-1]
		\arrow["\dashv"{anchor=center, rotate=-90}, draw=none, from=0, to=1]
	\end{tikzcd}
	\]
	中的脉 $\Nc$ 称为单纯范畴的\emph{融贯脉}\footnotemark (coherent nerve).
	另一边, ``几何实现'' $\mathfrak C [{-}]$ 又称为拟范畴的 \emph{Joyal 固化} (rigidification).
\end{definition}
\footnotetext{又称\emph{同伦融贯脉} (homotopy coherent nerve).}

我们不加证明地陈述如下技术性引理.
\begin{prop}
	{}
	纤维性单纯范畴的融贯脉是 $\infty$-范畴.
\end{prop}


\begin{definition}
	{($\infty$-范畴的 $\infty$-范畴)}
	定义 $\infty$-范畴的 $\infty$-范畴, 以及 $\infty$-群胚的 $\infty$-范畴为
	$$
	\infCatinfcat := \Nc (\infty \mathsf {Cat}),\quad
	\infGpdinfcat :=\Nc (\infty \mathsf {Gpd}).
	$$
\end{definition}

正如集合范畴 $\mathsf {Set}$ 是范畴的 ``原型'', 在 $\infty$-范畴中, 扮演这个角色的是 $\infGpdinfcat$. 它可视为某种 ``空间'' (不一定是传统意义上的拓扑空间) 的 $\infty$-范畴, 其中各阶态射表达了空间之间映射的各阶同伦; 许多作者直接称其为\emph{空间的 $\infty$-范畴}, 如 HTT \cite{HTT} 1.2 节. 我们将会看到, 类似于 $\mathsf {Set}$ 是终范畴 $1$ 自由生成的余完备范畴, $\infGpdinfcat$ 是 $1$ 自由生成的余完备 $\infty$-范畴.

%\newcommand{\Topinfcat}{{\mathcal{T}\hspace{-3pt}op}}
%\begin{definition}
%	{(拓扑空间的 $\infty$-范畴)}
%	定义拓扑空间的 $\infty$-范畴为
%	$$
%	\Topinfcat := \Nc (\mathsf {Top}).
%	$$
%\end{definition}



\section{Ind 完备化}

\begin{prop}
	{}
	对于范畴 $\mathcal C$ 上的预层 $F$, 如下条件等价:
	\begin{itemize}
		\item $F\in\operatorname{Ind}(\mathcal C)$;
		\item $\mathcal C_{/F}$ 为滤范畴.
	\end{itemize}
	进一步, 若 $\mathcal C^\op$ 具有有限极限, 上述条件还等价于
	\begin{itemize}
		\item $F\colon \mathcal C^\op\to\Grpdinf$ 保持有限极限.
	\end{itemize}
\end{prop}

\begin{prop}
	{(关于滤余极限的完备化, \tomeun\ref{t1-completion-wrt-filtered-colimits})}
	设 $\mathcal C$ 为小 $\infty$-范畴, $\mathcal D$ 为具有滤余极限的范畴,
	则函子 $\mathcal C\to\mathcal D$ 等同于保持滤余极限的函子 $\operatorname{Ind}(\mathcal C)\to\mathcal D$.
\end{prop}

\section{可表现 $\infty$-范畴}

\begin{definition}
	{(可达 $\infty$-范畴)}
	设 $\mathcal C$ 为 $\infty$-范畴, $\lambda$ 为正则基数.
	若存在小 $\infty$-范畴 $\mathcal D$ 使得 $\mathcal C\simeq\operatorname{Ind}_{\lambda}(\mathcal D)$,
	则称 $\mathcal C$ 为 \emph{$\lambda$-可达范畴}.
\end{definition}

\begin{definition}
	{(可表现 $\infty$-范畴)}
	设 $\mathcal C$ 为 $\infty$-范畴, 若 $\mathcal C$ 为可达 $\infty$-范畴, 且具有小余极限, 则称之为\emph{可表现 $\infty$-范畴}.
\end{definition}