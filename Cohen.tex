\section{Cohen 力迫法}

1874 年, Georg Cantor 证明了自然数与实数 (又称连续统) 之间不存在一一对应\footnote{不过他的第一个证明并非现在流行的对角线论证.}. Cantor 接着于 1878 年提出了\emph{连续统假设} (continuum hypothesis),
$$
\fbox{在自然数集合 $N$ 与连续统 $PN$ 之间不存在其它的基数.}
$$

1940 年, Kurt G\"odel 证明连续统假设与 Zermelo--Fraenkel 集合论相容. 1963 年, Paul Cohen 证明了连续统假设独立于带有选择公理的 Zermelo--Fraenkel 集合论 (ZFC), 即 ZFC 既不能证明, 也不能证伪连续统假设.

% Boole 意象的内容放在了第一章末尾.

\begin{prop}
	{}
	存在一个 Boole \topos{}, 其中选择公理成立, 而连续统假设不成立.
\end{prop}