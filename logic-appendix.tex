\chapter{形式逻辑基础}

\label{logic-appendix}

\minitoc
%\todo{整理本章}

\philoquote{Mathematicians are committed to rigorous reasoning, but they usually shy away from formal logic.}{William M. Farmer, \textit{The seven virtues of simple type theory}}

\section{一阶逻辑}

\label{first-order-languages}

我们即将引入的逻辑学中的若干基本概念可以视为 ``数学语言由什么构成'' 这个问题的一种完全形式化的回答. 写出语言的形式定义之前, 我们先从日常的数学语言中最熟悉的例子开始, 对语言的每个成分建立感性的认知.

%下面的例子中, $\Omega$ 是二元集合 $\{\top,\bot\}$, 其两个元素分别代表真和假.

数学语言中最常见的成分是\emph{公式} (formula).

\begin{example}
	{(公式)}
	$1+1=2$ 是一个公式, 其中
	\begin{itemize}
		\item $1,2$ 是自然数, 即是\emph{类型} (type) $\mathbb{N}$ 的\emph{项} (terms);
		\item 加号 $+ \colon \mathbb{N}\times\mathbb{N} \to \mathbb{N}$ 是一个\emph{函数符号} (function symbol);
		\item 将两个 $1$ 放在一起得到 $(1,1)$, 它的类型是 $\mathbb{N}\times\mathbb{N}$;
		\item 以 $+$ 作用于 $(1,1)$, 得到 $1+1$, 类型为 $\mathbb{N}$;
		\item 以 $=$ 连接类型 $\mathbb{N}$ 的两个项 $1+1$ 与 $2$, 得到公式 $1+1=2$. %它是类型 $\Omega$ 的项, 且这个项等于 $\top$ (``真'').
	\end{itemize}
\end{example}

公式中可以带有\emph{变量} (variables).

\begin{example}
	{(带自由变量的公式)}
	$y=x^2$ 是一个带\emph{自由变量} (free variables) 的公式, 其中
	\begin{itemize}
		\item $x,y$ 是类型 $\mathbb{R}$ 的\emph{变量}, 变量是\emph{项}的一种, 所以 $x,y$ 也是类型 $\mathbb{R}$ 的项;
		\item 平方 $(-)^2 \colon \mathbb{R} \to \mathbb{R}$ 是一个函数符号, 以 $(-)^2$ 作用于 $x$ 得到 $x^2$, 它是类型 $\mathbb{R}$ 的项, 带一个\emph{自由变量} $x$;
		\item 以 $=$ 连接类型 $\mathbb{R}$ 的两个项 $y$ 与 $x^2$, 得到公式 $y=x^2$. %它是类型 $\Omega$ 的项, 带两个自由变量 $x,y$.
	\end{itemize}
\end{example}

对公式中的自由变量, 我们可使用``存在''和``任意''这两个\emph{量词} (quantifiers).

\begin{example}
	{(带量词的公式)}
	$\neg\exists x\ x^2=-1$ 是一个不含自由变量的公式, 其中
	\begin{itemize}
		\item $x$ 是类型 $\mathbb{R}$ 的变量, $x^2$ 是类型 $\mathbb{R}$ 的项, 带一个自由变量 $x$;
		\item $x^2=-1$ 是带一个自由变量 $x$ 的公式;
		\item \emph{量词} $\exists x$ 放在含自由变量 $x$ 的公式前面, 得到\emph{不含}自由变量的公式 $\exists x\ x^2=-1$ (变量 $x$ 在这里称为\emph{约束变量} (bound variable));
		%它是类型 $\Omega$ 的项, 且这个项等于 $\bot$;
		\item 逻辑运算 $\neg$ (``非'') 放在公式 $\exists x\ x^2=-1$ 前面, 得到公式 $\neg\exists x\ x^2=-1$. %它是类型 $\Omega$ 的项, 且这个项等于 $\top$.
	\end{itemize}
\end{example}

\begin{example}
	{(素数)}
	如下公式表达了 ``$p$ 是素数'':
	$$
	\neg(p=1) \wedge \forall x \big((\exists y\ x\cdot y = p) \Rightarrow (x=1 \vee x=p)\big),
	$$
	其中 $x,y,p$ 是类型 $\mathbb{N}$ 的变量,
	整个公式有一个自由变量 $p$.
\end{example}

%\begin{definition}
%    {(类型, 项, 自由变量)} 一个\emph{类型} (type) 是\footnotemark 意象中的一个对象. 一个类型的\emph{项} (term) 是进入这个对象的一个态射, 这个态射的定义域是其中\emph{自由变量} (free variable) 所属的类型.
%\end{definition}

%\footnotetext{严格地说, 类型不\emph{是}意象中的对象, 它只是作为抽象语言的一部分被\emph{表示为}意象中具体的对象. 但本讲义介绍的主体是意象而非形式语言, 我们除了意象的内语言.}

%\begin{definition}{(公式)}
%类型 $\Omega$ 的项称为\emph{公式} (formula).
%\end{definition}

% 下面给出我们使用的 ``语言'' 概念的形式化定义\footnote{这些定义还可以更加形式化, 但那样就会牺牲可读性.}; 请注意这只是 ``语言'' 的众多定义中适合本讲义的一种; 我们的目的是引入意象的内语言.

\subsection{一阶语言的基本要件}

\subsubsection{符号表}

语言的形式化定义依赖于一个\emph{符号表} (signature).\footnote{逻辑学中的 signature 似乎没有通行的中文译名. 由于它给出了语言中所用的符号的集合, 试译为符号表.}

\begin{definition}
	[label={def-first-order-signature}]
	{(符号表)}
	一个 (一阶) \emph{符号表} $\Sigma$ 由如下内容构成:
	\begin{itemize}
		\item 一族\emph{类型} (types),
		% 其中任意有限个类型可作乘积得到一个新的类型 (包括 $0$ 个类型的乘积, 记之为 $1$),
		每个类型可有任意多个\emph{变量} (variables)\footnotemark;
		\item 一些\emph{函数符号} (function symbols),
		每一个函数符号 $f$ 具有固定的类型 $A_1,\cdots,A_n,B$, 记作 $f \colon A_1\cdots A_n \to B$, 非负整数 $n$ 称为 $f$ 的\emph{元数} (\makebox{arity}); (当 $n=0$ 时, 函数符号是 ``零元函数'', 也即类型 $B$ 的\emph{常数}.)
		\item 一些\emph{关系符号} (relation symbols), 又叫\emph{谓词} (predicates),
		每一个关系符号 $R$ 具有固定的类型 $A_1,\cdots,A_n$, 记作 $R \hookrightarrow A_1\cdots A_n$, 非负整数 $n$ 称为 $R$ 的元数. (当 $n=0$ 时, 关系符号是 ``零元关系'', 也即\emph{原子命题} (atomic proposition).)
	\end{itemize}
\end{definition}
\footnotetext{要求一个类型有任意多个变量的目的是, 在任何场景我们都可以自由地声明一个此前从未出现的变量. 类型 $X$ 的变量 $x$ 不是 $X$ 的元素, 只是一个形式上的符号, 不携带任何信息. 你可以想象 $x$ 是 ``未定元'', 但这种说法不具有数学上的含义.}

\begin{remark}{}
	如果我们在定义 \ref{def-first-order-signature} 中加入类型的有限积, 那么就不需要多元函数和多元关系; 但这样定义也有一些代价, 例如本来只有一个类型 $G$ 的语言将会需要无穷多个类型 $1,G,G^2,\cdots$.
	
	另外, 在定义 \ref{def-first-order-signature} 中, 关系符号与函数符号被区分开了; 但在通常的数学语言中我们可以认为某类型 $A$ 上的关系符号不过是 $A$ 到 ``真值集合'' 类型 $\{\top,\bot\}$ 的函数符号. 在一般意象的内语言中, $\{\top,\bot\}$ 的角色由\emph{子对象分类子} $\Omega$ 扮演.
\end{remark}

\begin{example}
	[label={natural-numbers-language}]
	{(初等算术)}
	初等算术的语言的符号表包括
	\begin{itemize}
		\item 类型 $\mathbb{N}$;
		\item 常数 $0$, 一元函数符号 $S$ (后继), 二元函数符号 $+,\times$ (加法, 乘法);
		\item 关系符号 $<$.
	\end{itemize}
\end{example}

\begin{example}
	[label={Zermelo--Fraenkel-language}]
	{(Zermelo--Fraenkel 集合论)}
	Zermelo--Fraenkel (简称 ZF) 集合论的符号表包括
	\begin{itemize}
		\item 类型 $S$ (``所有东西都是集合'');
		\item 二元关系符号 $\in$.
	\end{itemize}
\end{example}

\begin{example}
	[label={group-language}]
	{(群)}
	群的语言的符号表包括
	\begin{itemize}
		\item 类型 $G$;
		\item 常数 $1$ (单位元), 一元函数符号 $(-)^{-1}\colon G\to G$ (逆), 二元函数符号 $\cdot\colon GG \to G$ (乘法);
	\end{itemize}
	没有关系符号\footnotemark.
	
	读者可试着写出\emph{环}的符号表.
\end{example}

\footnotetext{或者说有一个关系符号 ``$=$''. 等号是默认存在的.}

\begin{example}
	[label={small-category-language}]
	{(小范畴)}
	小范畴的语言的符号表包括
	\begin{itemize}
		\item 类型 $O$ (对象), $M$ (态射);
		\item 一元函数符号 $s,t\colon M\to O$,
		(态射的起点与终点), 一元函数符号 $\operatorname{id}\colon O\to M$
		(对象的恒等态射);
		\item 三元关系符号 $C\hookrightarrow MMM$,
		$C(f,g,h)$ (``$h$ 等于 $f\circ g$'').
	\end{itemize}
	注意, 范畴中并非任意两个态射都能复合, 故表达复合关系只能使用三元关系符号, 而不能使用二元函数符号.
\end{example}

\subsubsection{项, 公式}

\begin{definition}
	[label={definition-terms}]
	{(项)}
	设 $\Sigma$ 为一符号表, 其上的\emph{项} (terms) 由如下条款归纳定义.
	\begin{itemize}
		\item 一个类型 $A$ 的单独一个变量 $x$ 是一项;
		%\item 若 $x_1,\cdots,x_n$ 分别是类型 $A_1,\cdots,A_n$ 的项, 那么 $(x_1,\cdots,x_n)$ 是类型 $A_1\times\cdots\times A_n$ 的项;
		\item 对于函数符号 $f \colon A_1\cdots A_n \to B$, 若 $x_1,\cdots,x_n$ 分别是类型 $A_1,\cdots,A_n$ 的项, 则 $f(x_1,\cdots,x_n)$ 是类型 $B$ 的项.
	\end{itemize}
\end{definition}

\begin{definition}
	[label={formula}]
	{(公式的形成规则)}
	语言中的\emph{公式} (formulae) 有如下\emph{形成规则} (formation rules), 同时我们归纳地定义公式中的\emph{自由变量} (free variables).
	\begin{enumerate}[(i)]
		\item (关系) 对于关系符号 $R \hookrightarrow A_1\cdots A_n$, 若 $x_1,\cdots,x_n$ 分别是类型 $A_1,\cdots,A_n$ 的项, 则 $f(x_1,\cdots,x_n)$ 是公式, 其中的自由变量是所有在某个 $x_i$ 中出现的变量;
		\item (等式) 对于\emph{相同类型}的项 $x,y$, $(x=y)$ 是公式, 其中的自由变量是所有出现在 $x$ 或 $y$ 中 (或两者兼有) 的变量;
		\item (真) $\top$ 是公式, 其中没有自由变量;
		\item (且, 又叫\emph{合取}, conjunction) 对于公式 $\phi,\psi$, $(\phi\wedge\psi)$ 是公式, 其中自由变量是 $\phi$ 与 $\psi$ 的自由变量的并;
		\item (假) $\bot$ 是公式, 其中没有自由变量;
		\item (或, 又叫\emph{析取}, disjunction) 对于公式 $\phi,\psi$, $(\phi\vee\psi)$ 是公式, 其中自由变量是 $\phi$ 与 $\psi$ 的自由变量的并;
		\item (蕴含) 对于公式 $\phi,\psi$, $(\phi\Rightarrow \psi)$ 是公式, 其中自由变量是 $\phi$ 与 $\psi$ 的自由变量的并;
		\item (否定) 对于公式 $\phi$, $\neg\phi$ 是公式, 其中自由变量即为 $\phi$ 的自由变量;
		\item (存在量词) 对于公式 $\phi$ 以及变量 $x$, $(\exists x\,\phi)$ 是公式, 其中的自由变量为 $\phi$ 的自由变量\emph{去掉} $x$ (我们允许 $\phi$ 中\emph{不含} $x$);
		\item (存在唯一) 对于公式 $\phi$ 以及变量 $x$, 引入一个记号 $\exists ! x\,\phi$ 表示 $$(\exists x\,\phi) \,\land\, \forall x\forall x'\,\big( \phi \land \phi[x'/x] \Rightarrow x'=x \big),$$
		其中 $\phi[x'/x]$ 为将公式 $\phi$ 中的自由变量 $x$ 替换为 $x'$ 所得的公式.
		\item (全称量词) 对于公式 $\phi$ 以及变量 $x$, $(\forall x\,\phi)$ 是公式, 其中的自由变量为 $\phi$ 的自由变量去掉 $x$;
		\item (无穷析取) 对于公式 $\phi_i\, (i\in I)$, 若其中的自由变量有限, 则 $\bigvee_{i\in I}\phi_i$ 是公式 (包括 $\bot$), 其中的自由变量为 $\phi_i$ 的自由变量的并;
		\item (无穷合取) 对于公式 $\phi_i\, (i\in I)$, 若其中的自由变量有限, 则 $\bigwedge_{i\in I}\phi_i$ 是公式 (包括 $\top$), 其中的自由变量为 $\phi_i$ 的自由变量的并.
	\end{enumerate}
\end{definition}

\begin{remark}
	{}
	在\emph{语法}的层面, 项与公式是两种不同的东西; 但项与公式可以有相同的\emph{语义}: Mitchell--B\'enabou 语言 (\ref{Mitchell--Benabou-language} 节) 中的公式不过是 $\Omega$ 类型的项.
\end{remark}



\begin{definition}
	[label={kinds-of-formulae}]
	{(几类公式)}
	对于固定的符号表 $\Sigma$,
	几类公式由如下形成规则定义.
	\begin{itemize}
		\item \emph{原子公式} (atomic formulae) : 关系与等式.
		\item \emph{Horn 公式}\footnotemark{} : 原子公式, 真, 二元合取 (有限合取).
		\item \emph{正则公式} (regular formulae) : 原子公式, 真, 二元合取, 存在量词.
		\item \emph{凝聚公式} (coherent formulae) : 原子公式, 真, 二元合取, 存在量词, 假, 二元析取.
		\item \emph{一阶公式} (first-order formulae) : 所有由有限规则 (定义 \ref{formula} 去掉最后两条) 构造的公式.
		\item \emph{几何公式} (geometric formulae) : 原子公式, 真, 二元合取, 存在量词, 假, 以及\emph{无穷析取}.
		\item \emph{无限一阶公式} (infinitary first-order formulae) : 所有由定义 \ref{formula} 的规则构造的公式.
	\end{itemize}
\end{definition}
\footnotetext{Alfred Horn (1918-2001), 美国数学家, 以格论与万有代数方面的工作闻名.}

\begin{example}
	{}
	在环的语言中, ``$x$ 幂零'' 可表达为如下的\emph{几何公式} (使用无穷析取):
	$$
	(x=0)\lor (x\cdot x=0) \lor (x\cdot x\cdot x=0)\lor\cdots;
	$$
	``$x$ 可逆'' 可表达为如下的几何公式 (使用存在量词):
	$$
	\exists y\, xy= 1.
	$$
	但如下表达 ``非零元都可逆'' 的公式\emph{不是}几何公式:
	$$
	\neg(x=0) \Rightarrow \exists y\,xy=1,
	$$
	因为使用了 ``蕴涵''.
\end{example}

\begin{remark}
	{}
	由有限规则定义的一类公式构成\emph{集合}, 而后两类公式 (几何公式, 无限一阶公式) 只能说构成\emph{类}.
	
	由于几何公式可能涉及无穷析取, 它不再属于 (有限的) 一阶逻辑.
\end{remark}

\subsection{一阶理论}

\begin{definition}
	[label={definition-context}]
	{(语境)}
	\emph{语境} (context) 是一列有限个互不相同的变量\footnotemark $\vec x = (x_1,\cdots,x_n)$. 其中每个变量 $x_i$ 有确定的类型 $A_i$, 这些类型可能相同也可能不同.
	
	设 $\vec x$ 是一个语境. 对于项 $t$ (定义 \ref{definition-terms}) 或公式 $\phi$ (定义 \ref{formula}), 若 $\vec x$ 包含了其中的所有自由变量, 则称 $\vec x$ \emph{适合于} (is suitable for) 项 $t$ 或公式 $\phi$.
	此时, 我们也以 $\vec x.\phi$, $\vec x.t$ 表示\emph{带语境的公式} (formula-in-context), \emph{带语境的项} (term-in-context).
\end{definition}

\footnotetext{注意这里 ``互不相同'' 是指变量的名称不同, 而不是 ``值'' 不同; 变量是一个形式的记号, 没有 ``值''.}

\begin{definition}
	[label={sequents}]
	{(相继式)}
	符号表 $\Sigma$ 上的一个\emph{相继式} (sequent) 是指一个形式的表达式
	$$
	\phi \vdash_{\vec x} \psi,
	$$
	意指 ``在语境 $\vec x$ 中, 若 $\phi$, 则 $\psi$'', 其中 $\phi,\psi$ 是符号表 $\Sigma$ 上的公式,
	$\vec x$ 是一个适合于 $\phi,\psi$ 的语境.
	
	我们用 $\vdash_{\vec x} \psi$ 表示 $\top\vdash_{\vec x} \psi$.
\end{definition}

\begin{remark}
	[label={remark-full-first-order-logic-no-need-sequents}]
	{}
	在完整的一阶逻辑中不需要一般的相继式, 因为
	$\phi \vdash_{(x_1,\cdots,x_n)} \psi$ 可表示为
	$\vdash \forall x_1\cdots \forall x_n (\phi\Rightarrow \psi)$.
\end{remark}

\begin{definition}
	[label={theory-on-signature}]
	{(理论)}
	符号表 $\Sigma$ 上的一个\emph{理论} (theory) $\mathbb T$ 是若干条\emph{公理} (axioms) 的集合, 每个公理是 $\Sigma$ 上的一个相继式.
\end{definition}

\begin{definition}
	[label={kinds-of-theories}]
	{(几类不同的理论)}
	\begin{itemize}
		\item 称理论 $\mathbb T$ 为\emph{命题理论} (propositional theory), 是指其符号表中没有类型.
		\item 称理论 $\mathbb T$ 为\emph{代数理论} (algebraic theory), 是指其符号表中没有关系符号 (等号除外), 且仅包含形如 $\vdash_{\vec x} (s=t)$ 的公理. 这是定义 \ref{presentation-of-an-algebraic-theory}.
		\item 称理论 $\mathbb T$ 为 \emph{Horn} (\emph{正则}, \emph{凝聚}, \emph{几何}, $\cdots$) \emph{理论}, 是指其公理只涉及 Horn (正则, 凝聚, 几何, $\cdots$) 公式. 几类公式的定义见 \ref{kinds-of-formulae}.
	\end{itemize}
\end{definition}

%\begin{remark}
%	{}
%	几何理论
%\end{remark}



每种一阶逻辑都有配套的\emph{推理系统} (deduction system). 推理系统中会给出形如
$$
\sqc{\Gamma}{\sigma}
$$
的推理规则, 表示由若干相继式 $\Gamma$ 可以得到相继式 $\sigma$.
双横线表示上下两个相继式由任何一个可得另一个.

首先是\emph{相继式演算} (sequent calculus) 的结构性规则 (structural rules). 这些规则在

\begin{definition}
	[label={first-order-structural-rules}]
	{(相继式演算的结构性规则)}
	\begin{itemize}
		\item \emph{恒等公理} (identity axiom),
		$$
		\sqc{}{\phi\vdash_{\vec x} \phi}.
		$$
		\item \emph{替换规则} (substitution rule),
		记 $\phi[t/y]$ 为将公式 $\phi$ 中的自由变量 $y$ 替换为同类型的项 $t$ 所得的公式, 那么有规则
		$$
		\sqc{\phi \vdash_{\vec x,y} \psi}{\phi[t/y]\vdash_{\vec x}\psi[t/y]}.
		$$
		\item \emph{剪切规则} (cut rule),
		$$
		\sqc{
			\Gamma \vdash_{\vec x} \Delta , \phi
			\quad
			\phi, \Sigma \vdash_{\vec x} \Pi 
		}{
			\Gamma,\Sigma \vdash_{\vec x} \Delta,\Pi
		}.
		$$
		(这里公式 $\phi$ 被 ``剪掉'' 了.)
	\end{itemize}
\end{definition}

\begin{definition}
	[label={deduction-system-first-order-logic}]
	{(一阶逻辑的推理系统)}
	在结构性规则 (定义 \ref{first-order-structural-rules}) 的基础上, 一阶逻辑的\emph{推理系统}还可能包含如下公理与规则.
	\begin{itemize}
		\item \emph{有限合取规则}, 包含公理
		$$
		\phi \vdash_{\vec x} \top
		\quad
		(\phi\wedge\psi)\vdash_{\vec x} \phi
		\quad
		(\phi\wedge\psi)\vdash_{\vec x} \psi
		$$
		与推理规则
		$$
		\sqc{\phi\vdash_{\vec x} \psi\quad \phi\vdash_{\vec x} \chi}{\phi\vdash_{\vec x} (\psi\wedge\chi)}.
		$$
		\item \emph{有限析取规则}, 包含公理
		$$
		\bot \vdash_{\vec x} \phi
		\quad
		\phi \vdash_{\vec x} (\phi\vee\psi)
		\quad
		\psi \vdash_{\vec x} (\phi\vee\psi)
		$$
		与推理规则
		$$
		\sqc{\phi \vdash_{\vec x} \chi \quad \psi \vdash_{\vec x} \chi}{ (\phi\vee\psi)\vdash_{\vec x}\chi}.
		$$
		\item \emph{无限合取与析取规则}, 其公理与推理规则与前两条类似.
		\item \emph{蕴涵规则}, 有公理
		$$
		\sqqc{\phi\wedge\psi \vdash_{\vec x} \chi}{\psi\vdash_{\vec x} (\phi\Rightarrow\chi)}.
		$$
		\item \emph{存在量词规则}, 有公理
		$$
		\sqqc{\phi \vdash_{\vec x,y} \psi}{(\exists y.\phi)\vdash_{\vec x} \psi},
		$$
		其中 $\psi$ 不含自由变量 $y$.
		\item \emph{全称量词规则}, 有公理
		$$
		\sqqc{\phi \vdash_{\vec x,y} \psi}{\phi\vdash_{\vec x} (\forall y.\psi)}.
		$$
		\item \emph{分配公理}
		$$
		\phi\wedge(\psi\vee\chi)\vdash_{\vec{x}}(\phi\wedge\psi)\vee(\phi\wedge\chi).
		$$
		\item \emph{Frobenius 公理}
		$$
		\phi\land(\exists y.\psi)\vdash_{\vec{x}}\exists y.(\phi\land\psi).
		$$
		\item \emph{排中律}
		$$
		\top\vdash_{\vec{x}}\phi\lor\neg\phi.
		$$
	\end{itemize}
\end{definition}

\begin{remark}
	{}
	上述的规则是纯粹形式的, 不具有任何含义.
	注意这些规则由逻辑连接词 (或量词) 的\emph{引入规则}和\emph{消去规则}构成. 所谓引入规则就是如何得到一个公式, 消去规则就是如何使用一个公式.
\end{remark}

%几种常见的推理系统如下.

\begin{definition}
	[label={inference-rules}]
	{}
	在结构性规则的基础上, 几种逻辑有如下\emph{推理规则} (inference rules).
	\begin{itemize}
		\item \emph{代数逻辑} (algebraic logic), 没有附加规则.
		\item \emph{Horn 逻辑}, 有限合取规则.
		\item \emph{正则逻辑} (regular logic), 有限合取规则, 存在量词规则, Frobenius 公理.
		\item \emph{凝聚逻辑} (coherent logic), 有限合取与析取规则, 存在量词规则, 分配公理, Frobenius 公理.
		\item \emph{几何逻辑} (geometric logic), 有限合取规则, 无限析取规则, 存在量词规则, 无限分配公理, Frobenius 公理.
		\item \emph{直觉主义一阶逻辑} (intuitionistic first-order logic), \emph{除排中律以外的}所有有限规则.
		\item \emph{经典一阶逻辑} (classical first-order logic), 所有有限规则.
	\end{itemize}
\end{definition}

\begin{definition}
	[label={first-order-provability}]
	{}
	称相继式 $\sigma$ 在代数 (Horn, 正则, 凝聚, $\cdots$) 理论 $\mathbb T$ 下\emph{可证},
	是指存在一个形如下图的\emph{树} (读者可自行补充树的严格定义),
	%在对应的推理系统中, 由 $\mathbb T$ 的公理可以推导出 $\sigma$.
	\begin{center}
		\begin{tabular}{llll}
			$\sigma_1$ &            & $\sigma_2$ &            \\ \cline{1-3}
			& $\sigma_3$ &            & $\sigma_4$ \\ \cline{2-4} 
			&            & $\sigma$ &           
		\end{tabular}
	\end{center}
	其中上方无横线的相继式 (图中为 $\sigma_1,\sigma_2,\sigma_4$) 为 $\mathbb T$ 中的\emph{公理}, 且每条横线的上方与下方均来自相应逻辑的某条推理规则.
\end{definition}

\begin{example}
	{}
	在经典一阶逻辑中, 分配公理 $\phi\wedge(\psi\vee\chi)\vdash_{\vec{x}}(\phi\wedge\psi)\vee(\phi\wedge\chi)$ 有如下的证明.
	简便起见, 记 $p=(\phi\wedge\psi)\vee(\phi\wedge\chi)$.
	\begin{center}
		\begin{tabular}{lll}
			$\phi\land\psi \vdash_{\vec{x}}  p$&&$\phi\land\chi \vdash_{\vec{x}}  p$  \\ \cline{1-1} \cline{3-3} 
			$\psi \vdash_{\vec{x}} \phi\Rightarrow p$
			&&
			$\chi \vdash_{\vec{x}} \phi\Rightarrow p$
			\\ \hline
			& $\psi\lor\chi \vdash_{\vec{x}} \phi\Rightarrow p$ &  \\ \cline{2-2}
			& $\phi\wedge(\psi\vee\chi)\vdash_{\vec{x}}p$&
		\end{tabular}
	\end{center}
	其中四行使用的规则依次为: 有限析取规则, 蕴涵规则, 有限析取规则, 蕴涵规则. 然而在凝聚逻辑 (定义 \ref{inference-rules}) 中缺少蕴涵规则, 故需要分配公理. 类似地, 凝聚逻辑也需要 Frobenius 公理.
\end{example}

\begin{example}
	{(初等算术的理论)}
	继续例 \ref{natural-numbers-language}, 初等算术的一种理论有如下的公理:
	\begin{multicols*}
		{2}
		\begin{itemize}
			\item $\vdash_x \neg (Sx=0)$;
			\item $\neg(x=0) \vdash_{x} \exists y. Sy=x$;
			\item $Sx=Sy\vdash_{(x,y)} x=y$;
			\item $\cdots$ (略)
		\end{itemize}
	\end{multicols*}
	
	初等算术有许多不同但互相等价的公理系统.
\end{example}

\begin{example}
	{(Zermelo--Fraenkel 集合论)}
	继续例 \ref{Zermelo--Fraenkel-language}, ZF 有如下的公理 (我们将 $\forall y.y\in x\Rightarrow\cdots$ 简记为 $\forall y\in x.\cdots$, 将 $\exists y.y\in x\land\cdots$ 简记为 $\exists y\in x.\cdots$, 将 $\forall z\in x.z\in y$ 简记为 $x\subset y$, 并且参考注 \ref{remark-full-first-order-logic-no-need-sequents} 省略符号 $\vdash_{\cdots}$):
	\begin{itemize}
		\item (外延) $\forall x.\forall y.\forall z. (z\in x  \Leftrightarrow z\in y) \Rightarrow x=y$;
		\item (配对) $\forall x.\forall y.\exists z.\forall w.\big(w\in z \Leftrightarrow (w=x \lor w=y)\big)$;
		\item (并) $\forall x.\exists y.\forall z. (z\in y \Leftrightarrow \exists w\in x. z\in w)$;
		\item (幂集) $\forall x.\exists y.\forall z. (z\in y \Leftrightarrow z\subset x)$;
		\item (无穷) $\exists x.\exists y\in x.\forall z\in x.\exists t\in x.z\subset t\land z\neq t$;
		\item (良基) $\forall x. \forall y\in x. \exists z\in x. \forall w\in x.(\neg w\in z)$;
		\item (分离公理模式) 对每个公式 $\phi$ (不含自由变量 $y$), 有一条公理 $\forall x.\exists y.\forall z. (z\in x\Leftrightarrow z\in x\land \phi)$;
		\item (替换公理模式) 对每个公式 $\phi$ (不含自由变量 $w$), 有一条公理 $\forall x.\big[(\forall y\in x.\exists ! z. \phi)
		\Rightarrow \exists w.\forall y\in x. \Rightarrow \exists z\in w.\phi)\big]$, 其中 $\exists ! z.\phi$ 是指 $\exists z.\phi \land\big( \forall z.\forall z'. (\phi\land\phi[z'/z])\Rightarrow z'=z\big)$.
	\end{itemize}
	ZF 是一种\emph{经典一阶理论}.
\end{example}

\begin{example}
	[label={theory-of-groups}]
	{(群的理论)}
	继续例 \ref{group-language}, 群的理论有如下的公理:
	\begin{itemize}
		\item $\vdash_{(x,y,z)}(x\cdot y)\cdot z = x\cdot (y\cdot z)$;
		\item $\vdash_x x^{-1}\cdot x = x\cdot x^{-1} = 1$;
		\item $\vdash_x 1\cdot x = x\cdot 1 = x$.
	\end{itemize}
	群的理论是一种代数理论 (定义 \ref{kinds-of-theories}).
	类似地, 读者可写出环的理论的公理. 环的理论也是一种\emph{代数理论}. 每一种 Lawvere 理论 (\ref{universal-algebra} 节) 都可以视为这里定义的代数理论.
\end{example}

%\begin{example}
%    {(集合的理论)}
%    
%\end{example}

\begin{example}
	[label={small-category-theory}]
	{(小范畴的理论)}
	继续例 \ref{small-category-language}, 小范畴的理论有如下的公理:
	\begin{itemize}
		\item (首尾相接的两个态射可以复合) $s(f)=t(g)\vdash_{(f,g)}\exists h\, C(f,g,h)$,
		\item (复合唯一)
		$C(f,g,h)\land C(f,g,k) \vdash_{(f,g,h,k)} h=k$.
	\end{itemize}
	%这个例子取自 \cite{TST} 1.2.1 节.
	小范畴的理论是\emph{正则理论} (定义 \ref{kinds-of-theories}), 因为其公理只涉及原子公式, 二元合取, 以及存在量词.
\end{example}

\begin{example}
	{(局部环的理论)}
	
	在代数中, \emph{局部环}\footnotemark{}是指满足如下条件的环 $R$:
	\begin{itemize}
		\item $0\neq 1$,
		\item 对任意 $x,y\in R$, 若 $x+y=1$, 则 $x$ 与 $y$ 至少有一个可逆.
	\end{itemize}
	根据这种定义, 我们写出局部环的理论, 即环的理论加上两条公理
	\begin{itemize}
		\item $(0=1)\vdash \bot$,
		\item $x+y=1\vdash_{(x,y)} (\exists z. xz=1)\vee (\exists z. yz=1)$.
	\end{itemize}
	%由于使用了 $\bot$, $\vee$ 和存在量词 $\exists$, 这个理论\emph{不是}代数理论; 但它
	局部环的理论是一种\emph{凝聚理论} (定义 \ref{kinds-of-theories}), 因为其公理只涉及原子公式, 假, 二元析取, 以及存在量词.
	读者还可试着写出\emph{整环}的理论, 并说明它也是一种凝聚理论.
\end{example}

\footnotetext{局部环还有一种使用极大理想的定义, 但它难以用一阶理论表达.}

\begin{example}
	{(域的理论)}
	
	\emph{域}是指非零元均可逆的环. 在环的理论中加入公理
	\begin{itemize}
		\item $\vdash_x (x=0) \lor (\exists y. xy=1)$
	\end{itemize}
	就得到了一种域的理论\footnotemark. 它也是一种\emph{凝聚理论}.
\end{example}

\footnotetext{在构造主义数学中, 域有不止一种理论, 它们只是恰好在集合范畴 (Boole \topos{}) 中有相同的模型.}

%\begin{example}
%	{($K$-代数的理论)}
%	设 $K$ 是一个固定的环. 回忆 \emph{$K$-代数}是带有 $K$ 的数乘作用的环. 定义 $K$-代数的理论: 在环的理论中加入 \emph{$K$ 的每个元素 $a$ 作为一个函数符号 $m_a$}, 且\emph{对每组元素 $a,b$ 加入如下公理},
%	\begin{itemize}
%		\item $\vdash_x m_a(x+y) = m_a(x) + m_a(y)$;
%		\item $\vdash_{(x,y)} m_am_b (x) = m_{ab}(x)$;
%		\item $\vdash_x m_{a+b}(a) = m_a(x) + m_b(x)$.
%	\end{itemize}
%	这样, 对于每个多项式 $p\in K[x_1,\cdots,x_n]$, 我们就可以在 $K$-代数的理论中谈论公式 $p(x_1,\cdots,x_n) = 0$; 反之, 这个理论中的原子公式都等价于这种形式的公式.
%	由定义, 对固定的环 $K$, $K$-代数的理论是一种\emph{代数理论}.
%\end{example}

\begin{remark}
	{}
	一阶逻辑能够表达的理论是有限制的. 例如\emph{挠群} (torsion group, 即所有元素都是有限阶元素的群) 没有一阶理论.
	另外, 在域的一阶理论中, 不存在一个公式表达 ``$x$ 是单位根''. 参见 \cite{Lurie-Categorical-Logic} 第一讲.
\end{remark}

\section{一阶逻辑的范畴语义}

\subsection{一阶语言在范畴中的解释}

一阶语言可在具有合适结构的范畴中获得\emph{解释} (interpretation), 又称\emph{范畴语义} (categorical semantics).
由此我们可以谈论这种语言上的理论在范畴中的模型.

\begin{definition}
	[label={sigma-structures-in-category}]
	{(范畴中的 $\Sigma$-结构)}
	固定符号表 $\Sigma$. 设范畴 $\mathcal C$ 有有限乘积.
	定义 $\mathcal C$ 中的一个 \emph{$\Sigma$-结构} $M$ 为如下信息:
	\begin{itemize}
		\item 对 $\Sigma$ 中的每个类型 $A$, 指定 $\mathcal C$ 的对象 $MA$;
		\item 对 $\Sigma$ 中的每个函数符号 $f\colon A_1\cdots A_n \to B$, 指定 $\mathcal C$ 的态射
		$Mf \colon MA_1\times\cdots\times MA_n \to MB$;
		\item 对 $\Sigma$ 中的每个关系符号 $R \hookrightarrow A_1\cdots A_n$, 指定 $\mathcal C$ 中的子对象
		$MR \hookrightarrow MA_1\times\cdots\times MA_n$.
	\end{itemize}
	
	范畴 $\mathcal C$ 中两个 $\Sigma$-结构之间的态射 $h\colon M\to N$ 是一族态射 $h_A \colon MA \to NA$, 满足合适的交换图. 所有 $\Sigma$-结构构成一个范畴 $\Sigma\text{-}\mathsf{Str}(\mathcal C)$.
	
%	定义符号表 $\Sigma$ 的\emph{分类范畴} (classifying category) $\operatorname{Cl}(\Sigma)$ 如下:
%	\begin{itemize}
%		\item 其对象为语境 (定义 \ref{definition-context});
%		\item \todo{}
%	\end{itemize}
\end{definition}

\begin{prop}
	{}
	设 $\mathcal C,\mathcal D$ 是具有有限乘积的范畴,
	$F \colon \mathcal C \to \mathcal D$ 是保持有限乘积与单射的函子. 那么 $F$ 诱导了函子
	$\Sigma\text{-}\mathsf{Str}(F)\colon \Sigma\text{-}\mathsf{Str}(\mathcal C) \to \Sigma\text{-}\mathsf{Str}(\mathcal D)$.
\end{prop}


\begin{definition}
	[label={term-interpretation}]
	{($\Sigma$-结构对项的解释)}
	给定范畴 $\mathcal C$ 中的一个 $\Sigma$-结构 $M$, 我们将 $\Sigma$ 上的项 (定义 \ref{definition-terms}) \emph{解释为} $M$ 中的态射.
	具体地, 设 $\vec x = (x_1,\cdots,x_n)$ 是一个语境 (定义 \ref{definition-context}), $x_i\colon A_i$,
	$\vec x.t$ 是带语境的项, $t\colon B$, 则 $\interpretation{\vec x.t}_M$ 是如下定义的态射 $MA_1\times\cdots\times MA_n\to MB$:
	\begin{itemize}
		\item 若 $t=x_i$, 则 $\interpretation{\vec x.t} = \pi_i$ 是第 $i$ 分量的投影;
		\item 若 $t = f(t_1,\cdots,t_m)$, 则
		$\interpretation{\vec x.t} = Mf \circ \big(\interpretation{\vec x.t_1},\cdots,\interpretation{\vec x.t_m}\big)$.
	\end{itemize}
\end{definition}

设 $\vec x$ 是适合于项 $t$ 的语境, 那么 $\vec x$ 添加一个与 $t$ 无关的变量 $y$ 后仍是适合于 $t$ 的语境.
根据上述定义, 两个带语境的项 $\vec x.t$ 与 $(\vec x,y).t$ 的解释不同 (因为定义域不同), 但仅仅相差一个投影. 一般的这种现象可总结为如下命题.

\begin{prop}
	[label={term-substitution-property}]
	{(项的代换性质)}
	设 $\vec y.t$ 是符号表 $\Sigma$ 上带语境的项, 在范畴 $\mathcal C$ 中可解释,
	其中 $\vec y = (y_1,\cdots,y_m), y_i\colon B_i$, $t\colon C$.
	设 $\vec s = (s_1,\cdots,s_m), s_i\colon B_i$ 是任意的项, $\vec x=(x_1,\cdots,x_n)\,(x_i\colon A_i)$ 是适合于 $\vec s$ 的语境.
	那么, 对 $\mathcal C$ 上任意 $\Sigma$-结构 $M$,
	有如下交换图.
	% https://q.uiver.app/#q=WzAsMyxbMCwwLCJNQV8xXFx0aW1lc1xcY2RvdHNcXHRpbWVzIE1BX24iXSxbMSwwLCJNQl8xXFx0aW1lc1xcY2RvdHNcXHRpbWVzIE1CX20iXSxbMSwxLCJNQyJdLFswLDEsIihcXGludGVycHJldGF0aW9ue1xcdmVjIHguc18xfSxcXGNkb3RzLFxcaW50ZXJwcmV0YXRpb257XFx2ZWMgeC5zX219KSJdLFsxLDIsIlxcaW50ZXJwcmV0YXRpb257XFx2ZWMgeS4gdH0iXSxbMCwyLCJcXGludGVycHJldGF0aW9ue1xcdmVjIHguIHRbXFx2ZWMgcy8gXFx2ZWMgeV19IiwyLHsibGFiZWxfcG9zaXRpb24iOjB9XV0=
	\[\begin{tikzcd}[ampersand replacement=\&,column sep=6em]
		{MA_1\times\cdots\times MA_n} \& {MB_1\times\cdots\times MB_m} \\
		\& MC
		\arrow["{(\interpretation{\vec x.s_1},\cdots,\interpretation{\vec x.s_m})}", from=1-1, to=1-2]
		\arrow["{\interpretation{\vec y. t}}", from=1-2, to=2-2]
		\arrow["{\interpretation{\vec x. t[\vec s/ \vec y]}}"'{pos=0.35}, from=1-1, to=2-2]
	\end{tikzcd}\]
	这称为项的\emph{代换性质} (substitution property).
\end{prop}

\begin{proof}
	对 $t$ 的 ``复杂度'' 归纳, 由定义直接验证.
\end{proof}

\begin{example}
	{(弱化)}
	项的代换性质 (\ref{term-substitution-property}) 的一个常见的特例是 $\vec s = \vec y\subset \vec x$ 的情形.
	此时 $(\interpretation{\vec x.s_1},\cdots,\interpretation{\vec x.s_m})$ 是一个投影映射 $\pi$,
	性质 $\interpretation{\vec x.t} = \interpretation{\vec y.t}\circ \pi$ 称为项的\emph{弱化性质} (weakening property).
\end{example}

%不同种类的逻辑需要范畴上不同的结构.

下面定义公式的解释所需的几种范畴.

\begin{definition}
	{(正则范畴)}
	若一个范畴中存在有限极限和像 (定义 \ref{image-definition}), 且拉回保持像, 则称之为\emph{正则范畴} (regular category).
	由命题 \ref{image-vs-exists}, 正则范畴中态射 $f\colon X\to Y$ 的拉回 $f^*\colon \operatorname{Sub}(Y)\to \operatorname{Sub}(X)$ 有左伴随 $\exists_f$.
	正则范畴之间的\emph{正则函子} (regular functor) 是指保持有限极限与像的函子.
\end{definition}

\begin{definition}
	[label={coherent-category}]
	{(凝聚范畴)}
	若一个\emph{正则范畴}中所有对象的子对象格都存在\emph{有限并}, 且拉回保持有限并, 则称之为\emph{凝聚范畴} (coherent category).
	凝聚范畴之间的\emph{凝聚函子}是指保持有限并的\emph{正则函子}.
\end{definition}

\begin{definition}
	[label={Heyting-category}]
	{(Heyting 范畴)}
	若一个\emph{凝聚范畴}中所有态射 $f\colon X\to Y$ 对子对象的拉回 $f^*\colon \operatorname{Sub}(Y)\to\operatorname{Sub}(X)$ 有右伴随 $\forall_f\colon \operatorname{Sub}(X)\to \operatorname{Sub}(Y)$, 则称之为 \emph{Heyting 范畴}.
	Heyting 范畴之间的函子是保持所有 $\forall_f$ 的\emph{凝聚函子}.
\end{definition}

由第一章, \topos{}是正则范畴, 凝聚范畴, Heyting 范畴.

\begin{definition}
	{(几何范畴)}
	若一个\emph{正则范畴}中所有对象的子对象格都存在\emph{任意并}, 且被拉回保持, 则称之为\emph{几何范畴} (geometric category).
\end{definition}

由偏序集的伴随函子定理 (\ref{adjoint-functor-theorem-poset}), 几何范畴一定是 Heyting 范畴.

\topos{}不一定是几何范畴, 但 Grothendieck \topos{}一定是几何范畴.

由定义 \ref{frame-definition}, \fm{}等同于构成几何范畴的偏序集.

%有了对应的结构, 我们就可以给出各种一阶公式在范畴中的解释. 如下定义参考 \cite{Elephant} D1.2 节, \cite{TST} 1.3 节.

\begin{definition}
	[label={interpretation-formolae}]
	{($\Sigma$-结构对公式的解释)}
	设 $M$ 是范畴 $\mathcal C$ 中的 $\Sigma$-结构.
	我们归纳地将 $\Sigma$ 上的一个带语境的公式 $\vec x.\phi$ (其中 $\vec x= (x_1,\cdots,x_n)$, $x_i\colon A_i$) 解释为一个子对象
	$$
	\interpretation{\vec x.\phi}_M \hookrightarrow MA_1 \times \cdots \times MA_n.
	$$
	\begin{itemize}
		\item (关系) 若 $\phi(\vec x)$ 形如 $R(t_1,\cdots,t_m)$, $R$ 为类型 $B_1\cdots B_m$ 的关系符号, $t_i$ 为公式, 则
		$\interpretation{\vec x.\phi}_M$ 为如下拉回.
		% https://q.uiver.app/#q=WzAsNCxbMCwwLCJbW1xcdmVjIHguXFxwaGldXV9NIl0sWzEsMCwiTVIiXSxbMSwxLCJNQl8xXFx0aW1lc1xcY2RvdHNcXHRpbWVzIE1CX20iXSxbMCwxLCJNQV8xXFx0aW1lc1xcY2RvdHNcXHRpbWVzIE1BX24iXSxbMywyLCIoW1tcXHZlYyB4LnRfMV1dLFxcY2RvdHMsW1tcXHZlYyB4LnRfbV1dKSJdLFswLDMsIiIsMCx7InN0eWxlIjp7InRhaWwiOnsibmFtZSI6Imhvb2siLCJzaWRlIjoidG9wIn19fV0sWzAsMV0sWzEsMiwiIiwyLHsic3R5bGUiOnsidGFpbCI6eyJuYW1lIjoiaG9vayIsInNpZGUiOiJ0b3AifX19XV0=
		\[\begin{tikzcd}[ampersand replacement=\&,column sep=6em]
			{\interpretation{\vec x.\phi}_M} \& MR \\
			{MA_1\times\cdots\times MA_n} \& {MB_1\times\cdots\times MB_m}
			\arrow["{(\interpretation{\vec x.t_1},\cdots,\interpretation{\vec x.t_m})}", from=2-1, to=2-2]
			\arrow[hook, from=1-1, to=2-1]
			\arrow[from=1-1, to=1-2]
			\arrow[hook, from=1-2, to=2-2]
		\end{tikzcd}\]
		\item (等式) 若 $\phi(\vec x)$ 形如 $(s=t)$, $s,t$ 为类型 $B$ 的项, 则 $\interpretation{\vec x.\phi}$ 为两个态射 $\interpretation{\vec x.s},\interpretation{\vec x.t}\colon MA_1\times\cdots\times MA_n \to MB$ 的等化子.
		\item (真) 若 $\phi = \top$, 则 $\interpretation{\vec x.\phi}$ 是 $MA_1\times\cdots\times MA_n$ 作为自身的子对象.
		\item (且) 若 $\phi = \psi \wedge \chi$, 则 $\interpretation{\vec x.\phi}$ 是子对象 $\interpretation{\vec x.\psi}$ 与 $\interpretation{\vec x.\chi}$ 的交 (拉回).
		\item (假) 若 $\phi = \bot$, 则 $\interpretation{\vec x.\phi}$ 是 $MA_1\times\cdots\times MA_n$ 的子对象 $\bot$.
		\item (或) 若 $\phi = \psi \lor \chi$ 且 $\mathcal C$ 是\emph{凝聚范畴} (定义 \ref{coherent-category}),
		则 $\interpretation{\vec x.\phi}$ 是子对象 $\interpretation{\vec x.\psi}$ 与 $\interpretation{\vec x.\chi}$ 的并.
		\item (蕴涵) 若 $\phi = (\psi \Rightarrow \chi)$, 且 $\mathcal C$ 是 \emph{Heyting 范畴}, 则
		$\interpretation{\vec x.\phi}$ 是 $\interpretation{\vec x.\psi} \Rightarrow \interpretation{\vec x.\chi}$.
		\item (存在量词) 若 $\phi = \exists y. \psi$ 且 $\mathcal C$ 是\emph{正则范畴}, 则 $\interpretation{\vec x.\phi} = \exists_{\pi} (\interpretation{(\vec x,y).\psi})$,
		其中 $\pi$ 为 $MA_1\times\cdots\times MA_n \times MB$ 到前 $n$ 分量的投影.
		\item (全称量词) 若 $\phi = \forall y. \psi$ 且 $\mathcal C$ 是 \emph{Heyting 范畴}, 则 $\interpretation{\vec x.\phi} = \forall_{\pi} (\interpretation{(\vec x,y).\psi})$, 其中 $\pi$ 同上.
		\item (无穷析取) 若 $\phi = \bigvee_i \psi_i$ 且 $\mathcal C$ 是\emph{几何范畴}, 则 $\interpretation{\vec x.\phi}$ 是所有 $\interpretation{\vec x.\psi_i}$ 的并.
		\item (无穷合取) 若 $\phi = \bigwedge_i \psi_i$ 且 $\mathcal C$ 具有子对象的任意交, 则 $\interpretation{\vec x.\phi}$ 是所有 $\interpretation{\vec x.\psi_i}$ 的交.
	\end{itemize}
\end{definition}

回忆几类公式的定义 (\ref{kinds-of-formulae}), 我们发现在正则范畴中可以解释正则公式, 在凝聚范畴中可以解释凝聚公式, 在几何范畴中可以解释几何公式, 凡此种种, 不一而足. 任何逻辑公式都有适当的范畴结构可以解释之.

\begin{prop}
	{(公式的代换性质)}
	% Elephant D1.2.7
	设 $\vec y.\phi$ 是符号表 $\Sigma$ 上带语境的公式, 在范畴 $\mathcal C$ 中可解释,
	其中 $\vec y = (y_1,\cdots,y_m), y_i\colon B_i$.
	设 $\vec s = (s_1,\cdots,s_m), s_i\colon B_i$ 是任意的项, $\vec x=(x_1,\cdots,x_n)\,(x_i\colon A_i)$ 是适合于 $\vec s$ 的语境.
	那么, 对 $\mathcal C$ 上任意 $\Sigma$-结构 $M$,
	有如下拉回.
	% https://q.uiver.app/#q=WzAsNCxbMCwwLCJcXGludGVycHJldGF0aW9ue1xcdmVjIHguXFxwaGlbXFx2ZWMgcyAvIFxcdmVjIHldfSJdLFsxLDAsIlxcaW50ZXJwcmV0YXRpb257XFx2ZWMgeS5cXHBoaX0iXSxbMCwxLCJNQV8xXFx0aW1lc1xcY2RvdHNcXHRpbWVzIE1BX24iXSxbMSwxLCJNQl8xXFx0aW1lc1xcY2RvdHNcXHRpbWVzIE1CX20iXSxbMCwyXSxbMiwzLCIoXFxpbnRlcnByZXRhdGlvbntcXHZlYyB4LnNfMX0sXFxjZG90cyxcXGludGVycHJldGF0aW9ue1xcdmVjIHguc19tfSkiXSxbMCwxXSxbMSwzXV0=
	\[\begin{tikzcd}[ampersand replacement=\&,column sep=6em]
		{\interpretation{\vec x.\phi[\vec s / \vec y]}} \& {\interpretation{\vec y.\phi}} \\
		{MA_1\times\cdots\times MA_n} \& {MB_1\times\cdots\times MB_m}
		\arrow[from=1-1, to=2-1]
		\arrow["{(\interpretation{\vec x.s_1},\cdots,\interpretation{\vec x.s_m})}", from=2-1, to=2-2]
		\arrow[from=1-1, to=1-2]
		\arrow[from=1-2, to=2-2]
	\end{tikzcd}\]
\end{prop}
\begin{proof}
	对 $\phi$ 的 ``复杂度'' 归纳, 由定义直接验证.
\end{proof}

\subsection{一阶理论在范畴中的模型}

\begin{definition}
	{($\Sigma$-结构对相继式的满足)}
	设 $M$ 是范畴 $\mathcal C$ 上的 $\Sigma$-结构, $\phi\vdash_{\vec x}\psi$ 是一个相继式.
	称 $M$ \emph{满足} $\phi\vdash_{\vec x}\psi$ (记作 $M\models(\phi\vdash_{\vec x}\psi)$) 是指 $\interpretation{\vec x.\phi}_M \leq \interpretation{\vec x.\psi}_M$.
\end{definition}

\begin{definition}
	{(一阶理论在范畴中的模型)}
	% TST p.37
	设 $\mathbb T$ 是符号表 $\Sigma$ 上的一阶理论, $M$ 是范畴 $\mathcal C$ 上的 $\Sigma$-结构,
	且 $\mathbb T$ 的公理在 $\mathcal C$ 中可解释.
	若 $M$ 满足 $\mathbb T$ 的所有公理, 则称 $M$ 为 $\mathbb T$ 在 $\mathcal C$ 中的一个\emph{模型} (model).
	记 $\mathbb T$ 的模型在 $\Sigma\text{-}\mathsf{Str}(\mathcal C)$ 中构成的满子范畴为 $\mathbb T\text{-Mod}(\mathcal C)$.
\end{definition}

一阶逻辑的推理系统可以证明关于其模型的命题, 这个性质称为一阶逻辑的\emph{可靠性} (soundness).

\begin{prop}
	[label={soundness-first-order-logic}]
	{(可靠性)}
	设 $\mathbb T$ 为符号表 $\Sigma$ 上的 Horn (正则, 凝聚, 几何, $\cdots$) 理论, $M$ 为 $\mathbb T$ 在合适的范畴 $\mathcal C$ 中的模型, 则对任何 $\mathbb T$-可证 (定义 \ref{first-order-provability}) 的相继式 $\sigma$, 都有 $M\models\sigma$.
\end{prop}
\begin{proof}
	我们只需对定义 \ref{deduction-system-first-order-logic} 的每种规则 $\displaystyle\sqc{\Gamma}{\sigma}$ 验证若 $M\models\Gamma$, 则 $M\models\sigma$. 这基本上是平凡的. 例如, 存在量词规则的可靠性正是由于存在量词的解释是拉回的左伴随 (定义 \ref{interpretation-formolae}). 又如, 分配公理的可靠性是由于相应的范畴中子对象的运算 $A\land {-}$ 保持二元并.
\end{proof}

\begin{remark}
	{}
	反之, 只要找到一个模型 $M$ \emph{不满足} $\sigma$, 就说明 $\sigma$ 在理论 $\mathbb T$ 中\emph{不可证}.
\end{remark}

\section{高阶逻辑}

高阶逻辑相比一阶逻辑拥有表达能力更强的语义, 如允许将存在量词和全称量词用于函数与子集. 相应地, 其范畴语义需要范畴上更多的结构.

\subsection{高阶语言的基本要件}

% Elephant 4.1.1

\begin{definition}
	{(符号表)}
	高阶逻辑的一个符号表 $\Sigma$ 由如下内容构成:
	\begin{itemize}
		\item 一族\emph{基本类型} (basic types);
		% 其中任意有限个类型可作乘积得到一个新的类型 (包括 $0$ 个类型的乘积, 记之为 $1$),
		\item 一些\emph{函数符号} (function symbols);
		\item 一些\emph{关系符号} (relation symbols);
	\end{itemize}
	并且我们归纳地定义 $\Sigma$ 上的\emph{类型}:
	\begin{itemize}
		\item 每个基本类型都是一个类型;
		\item (零元积) $1$ 是一个类型;
		\item (二元积) 对任意两个类型 $A,B$, 有一个类型 $A\times B$;
		\item (函数类型) 对任意两个类型 $A,B$, 有一个类型 $[A\to B]$;
		\item (幂) 对任意类型 $A$, 有一个类型 $PA$; 记 $\Omega = P1$;
		\item (列表) 对任意类型 $A$, 有一个类型 $LA$.
	\end{itemize}
	对每个函数符号 $f$ 指定两个类型 $A,B$, 记 $f\colon A\to B$;
	对每个关系符号 $R$ 指定一个类型 $A$, 记 $R\hookrightarrow A$.
	这里的 $A,B$ 可以是基本类型, 也可以是由上述归纳规则定义的类型.
	例如 $A=A_1\times \cdots\times A_n$ 时, $f\colon A\to B$ 是 $n$ 元函数.
\end{definition}

高阶逻辑比一阶逻辑多了函数类型, 幂以及列表的操作, 使得可以谈论 ``对任意子集 $A\subset X$'' ``对任意函数 $f\colon X\to Y$'' 之类.



\begin{definition}
	{(项)}
	设 $\Sigma$ 为高阶逻辑的符号表, 归纳定义 $\Sigma$ 上的项.
	%我们仅列举一阶逻辑的定义 (\ref{definition-terms}) 中没有的条款.
	\begin{itemize}
		\item 一个类型 $A$ 的单独一个变量 $x$ 是一项;
		%\item 若 $x_1,\cdots,x_n$ 分别是类型 $A_1,\cdots,A_n$ 的项, 那么 $(x_1,\cdots,x_n)$ 是类型 $A_1\times\cdots\times A_n$ 的项;
		\item 对于函数符号 $f \colon A \to B$, 若有一项 $x\colon A$, 则有 $f(x)\colon B$.
		\item ($1$, 引入法则) 类型 $1$ 有一个项 $\star$;
		\item (二元对, 引入法则) 对于 $s\colon A$, $t\colon B$, 有 $(s,t)\colon A\times B$;
		\item (二元对, 消去法则) 对于 $t\colon A\times B$, 有 $\mathsf {fst}(t)\colon A$, $\mathsf {snd}(t)\colon B$, 分别代表取二元对的第一项和第二项;
		\item (函数, 引入法则) 对于类型 $A$ 的变量 $x$ 与 $t\colon B$ (其中 $t$ 的自由变量可包含 $x$, 也可不包含 $x$), 有 $\big(\lambda {x} . t\big) \colon [A\to B]$, 其自由变量为 $t$ 的自由变量去掉 $x$;
		\item (函数, 消去法则) 对于 $f\colon [A\to B]$ 以及 $t\colon A$, 有 $\mathsf {app}(f,t)\colon B$, 其自由变量为 $f$ 与 $t$ 的自由变量之并;
		\item (子集, 引入法则) 对公式 $\phi$ 以及 $x\colon A$, 有 $\{x\in A \mid \phi\} \colon PA$, 其自由变量为 $\phi$ 的自由变量去掉 $x$;
		\item (列表, 引入法则) 对于类型 $A$, 有 ``空列表'' $[\ ]\colon LA$;
		对于 $s,t\colon LA$ 有 ``列表的拼接'' $\mathsf {cons}(s,t)\colon LA$; 此外, 列表还有一个引入法则 ``迭代器'' (iterator), 即递推构造列表的法则, 这里略去 (见 \cite{Elephant} D4.1.2).
	\end{itemize}
	其中没有提到子集的消去法则, 是因为子集的消去法则给出的是一个原子公式 (下面的定义 \ref{higher-atomic-formulae}), 而非一个项.
\end{definition}

对于 $\Sigma$ 中的函数符号 $f\colon A\to B$, 取类型 $A$ 的变量 $x$, 由函数的引入法则 $\lambda x.f(x)\colon [A\to B]$ 是类型 $[A\to B]$ 的一个常量 (因为按定义其自由变量是 $f(x)$ 的自由变量去掉 $x$, 也即没有自由变量).

\begin{definition}
	[label={higher-atomic-formulae}]
	{(原子公式)}
	设 $\Sigma$ 为高阶逻辑的符号表, 定义下列公式为 $\Sigma$ 上的原子公式.
	\begin{itemize}
		\item (关系) 对关系符号 $R\hookrightarrow A$ 与 $t\colon A$, 关系 $R(t)$ 为原子公式, 其中的自由变量为 $t$ 的自由变量;
		\item (等式) 对 $s,t\colon  A$, 等式 $s=t$ 为原子公式, 其中的自由变量为 $s,t$ 的自由变量之并;
		\item (成员关系, 即 ``子集的消去法则'') 对 $t\colon A$ 与 $S\colon PA$, 成员关系 $t\in S$ 为原子公式, 其中的自由变量为 $S,t$ 的自由变量之并.
	\end{itemize}
	相比一阶逻辑中的原子公式 (定义 \ref{kinds-of-formulae}), 高阶逻辑多了成员关系.
\end{definition}

\begin{definition}
	{(公式)}
	高阶逻辑中的公式由原子公式以及定义 \ref{formula} 中的法则归纳定义.
\end{definition}

\begin{definition}
	{($\Sigma$-结构)}
	设 $\Sigma$ 为高阶逻辑的一个符号表, $\mathcal C$ 为合适的范畴.
	定义 $\mathcal C$ 上的 $\Sigma$-结构 $M$ 为如下资料:
	\begin{itemize}
		\item 对 $\Sigma$ 中每个基本类型 $A$ 指定 $\mathcal C$ 的一个对象 $MA$ (由此可归纳地对每个类型也指定一个对象, 如 $1$ 对应终对象, $A\times B$ 对应 $MA\times MB$, $[A\to B]$ 对应指数对象 $MB^{MA}$, $PA$ 对应幂对象 $P(MA)$
		%, $LA$ 对应列表对象 $L(MA)$
		);
		\item 对 $\Sigma$ 中每个函数符号 $f\colon A\to B$ 指定一个态射 $Mf\colon MA\to MB$.
		\item 对 $\Sigma$ 中每个关系符号 $R\hookrightarrow A$ 指定一个子对象 $MR \hookrightarrow MA$.
	\end{itemize}
\end{definition}

\begin{definition}
	[label={higher-order-term-interpretation}]
	{(项的解释)}
	设 $\vec x.t$ 是带语境的项, 其中 $\vec x=(x_1,\cdots,x_n), x_i\colon A_i$. 记 $A=A_1\times\cdots\times A_n$.
	\begin{itemize}
		\item (1, 引入法则) 若 $t=\star\colon 1$, 则 $\interpretation{\vec x.t}$ 是唯一的映射 $MA \to 1$.
		\item (二元对, 引入法则) 若 $t= (r,s)\colon B_1\times B_2$, 则 $\interpretation{\vec x.t} = (\interpretation{\vec x.r}, \interpretation{\vec x.s})\colon MA \to MB_1\times MB_2$.
		\item (二元对, 消去法则) 若 $t=\mathsf {fst}(s)$, $s\colon B\times C$, 则 $\interpretation{\vec x.t} = \pi_1\circ \interpretation{\vec x.s}\colon MA\to MB$, 对 $\mathsf {snd}$ 类似定义;
		\item (函数, 引入法则) 若 $t=(\lambda z.s)\colon [B\to C]$, 变量 $z$ 不出现在语境 $\vec x$ 中, 则 $\interpretation{\vec x.t}$ 是 $\interpretation{(\vec x,z).s}\colon MA\times MB\to MC$ 对应的态射 $MA \to MC^{MB}$.
		\item (函数, 消去法则) 若 $t = \mathsf {app}(r,s)$, $r\colon [B\to C]$, 则
		$\interpretation{\vec x.t}$ 是复合 % https://q.uiver.app/#q=WzAsMyxbMCwwLCJNQSJdLFsxLDAsIk1DXntNQn1cXHRpbWVzIE1CIl0sWzIsMCwiTUMiXSxbMCwxLCIoW1tcXHZlYyB4LnJdXSwgW1tcXHZlYyB4LiBzXV0pIl0sWzEsMiwiXFx0ZXh0e2V2fSJdXQ==
		\[\begin{tikzcd}[ampersand replacement=\&]
			MA \&\& {MC^{MB}\times MB} \& MC.
			\arrow["{(\interpretation{\vec x.r}, \interpretation{\vec x. s})}", from=1-1, to=1-3]
			\arrow["{\text{ev}}", from=1-3, to=1-4]
		\end{tikzcd}\]
		\item (子集, 引入法则) 若 $t=\{z\in B\mid \phi\}$, 则 $\interpretation{\vec x.t}$ 是 $\interpretation{(\vec x,z),\phi}\hookrightarrow MA\times MB$ 对应的态射 $MA \to P(MB)$.
		\item (列表, 引入法则) 略 (见 \cite{Elephant} D4.1.5).
	\end{itemize}
\end{definition}
\begin{definition}
	{(公式的解释)}
	原子公式以及一般的公式的解释沿用定义 \ref{interpretation-formolae}, 只是添加如下对成员关系的解释:
	\begin{itemize}
		\item 若 $\phi = (t\in S)$, $t\colon B$, 则 $\interpretation{\vec x.\phi}$ 是如下拉回,
		% https://q.uiver.app/#q=WzAsNCxbMSwxLCJNQlxcdGltZXMgUChNQikiXSxbMSwwLCJcXGluIl0sWzAsMSwiTUEiXSxbMCwwLCJbW1xcdmVjIHguXFxwaGldXSJdLFsxLDAsIiIsMCx7InN0eWxlIjp7InRhaWwiOnsibmFtZSI6Imhvb2siLCJzaWRlIjoidG9wIn19fV0sWzIsMCwiKFtbXFx2ZWMgeC50XV0sW1tcXHZlYyB4LlNdXSkiLDJdLFszLDIsIiIsMix7InN0eWxlIjp7InRhaWwiOnsibmFtZSI6Imhvb2siLCJzaWRlIjoidG9wIn19fV0sWzMsMV1d
		\[\begin{tikzcd}[ampersand replacement=\&,column sep=large]
			{\interpretation{\vec x.\phi}} \& \in \\
			MA \& {MB\times P(MB).}
			\arrow[hook, from=1-2, to=2-2]
			\arrow["{(\interpretation{\vec x.t},\interpretation{\vec x.S})}"', from=2-1, to=2-2]
			\arrow[hook, from=1-1, to=2-1]
			\arrow[from=1-1, to=1-2]
		\end{tikzcd}\]
	\end{itemize}
\end{definition}

\section{类型论}

\label{appendix-type-theory}

\philoquote{A logic is always a logic over a type theory.}{Bart Jacobs, \emph{Categorical Logic and Type Theory}}

类型论是一种做数学的视角; 不同于一阶逻辑中的 ``类型'', 类型论中的一切对象, 包括函数, 命题, 甚至证明, 都有一个确定的类型. 仅仅通过检查一个对象的类型, 就能验证推理的正确性. 关于类型论的更详细介绍见 \cite{Trebor-History}. 本节的目的是介绍类型论与范畴论的联系.
由于 ``类型论'' 有许多变种和风味, 很难给它下一个精确而完整的定义. 我们从例子开始.

\begin{example}
	{(简单类型论)}
	一种\emph{简单类型论} (simple type theory, STT, 又称 \emph{Church 类型论}) 包含如下的陈述:
	\begin{itemize}
		\item $n\colon \mathbb{N}$ (``$n$ 是类型 $\mathbb{N}$ 的项'');
		\item $\mathsf {succ}\colon \mathbb{N}\to\mathbb{N}$ (``后继 $\mathsf {succ}$ 是函数类型 $\mathbb{N}\to\mathbb{N}$ 的项'');
		\item $n\colon \mathbb{N},m\colon \mathbb{N} \vdash (n+m) \colon \mathbb{N}$ (``设有类型 $\mathbb{N}$ 的项 $n,m$, 则可以构造类型 $\mathbb{N}$ 的项 $n+m$''), 这里的符号 $\vdash$ 与一阶逻辑中的符号 $\vdash$ 含义不同, 它的左边 $n\colon \mathbb{N},m\colon \mathbb{N}$ 是一列变量的声明, 称为\emph{语境};
%		\item 函数的复合
%		$$
%		\sqc{x\colon X\vdash f(x)\colon Y\quad y\colon Y\vdash g(y)\colon Z}{x\colon X\vdash g(f(x))\colon Z}.
%		$$
	\end{itemize}
	
	\todo{简单类型论与高阶逻辑的关系}
\end{example}

简单类型论中没有一个类型依赖于某个变量, 而这是\emph{依值类型论}的特点.

\begin{example}
	{(依值类型论)}
	\begin{itemize}
		\item $n\colon \mathbb{N}\vdash n\mathsf {Vect}\colon \mathsf {Type}$ (``$n$ 维向量空间'' 是依赖于自然数 $n$ 的一族类型).
		\item 
	\end{itemize}
	这种风味的类型论叫\emph{依值类型论} (dependent type theory, DTT). 它的核心思想是 ``类型是值'', 也即有类型的类型 $\mathsf {Type}$, 而依值类型不过是一个类型到 $\mathsf {Type}$ 的函数. %另一种类型论允许类型依赖于一个\emph{类型变量}, 这称为\emph{多态类型论} (polymorphic type theory).
\end{example}

\begin{definition}
	{(类型论, 粗略定义)}
	一般地, 一种类型论由如下要件组成:
	\begin{itemize}
		\item 类型的\emph{形成法则} (formulation rules), 即 (从已有类型) 构造新类型的方法;
		\item 项的\emph{引入法则} (introduction rules), 即构造一个类型的项的方法;
		\item 项的\emph{消去法则} (elimination rules), 即使用一个类型的项的方法;
		\item \emph{计算法则} (computation rules), 以\emph{等式}表示将\emph{消去法则}作用于\emph{引入法则}上的结果.
	\end{itemize}
\end{definition}

\begin{example}
	{(Martin-L\"of 类型论)}
	\paragraph{空类型}
	\begin{itemize}
		\item 形成法则 $0\colon \mathsf {Type}$
	\end{itemize}
	
	\paragraph{}
\end{example}

\begin{example}
	{(群的理论)}
	
\end{example}

\subsection{命题是类型: Curry--Howard 同构}

Curry--Howard 同构是\emph{形式逻辑}与\emph{计算}之间的联系. 首先, 它指出\emph{命题是一种类型}. 将命题 $A$ 视为类型, 其元素是 $A$ 成立的\emph{证据}; 由此不难理解如下的对应. 例如, ``$A$ 且 $B$'' 成立的一个证据即是 $A$ 成立的一个证据加上 $B$ 成立的一个证据.

\begin{center}
	\begin{tabular}
		{cc}
		类型 & 命题 \\\hline
		$0$ & 假 \\
		$1$ & 真 \\
		$A\times B$ & $A$ 且 $B$ \\
		$A+B$ & $A$ 或 $B$ \\
		$B^A\text{ (又记为 $A\to B$)}$ & $A$ 蕴涵 $B$ \\
		$\prod_{x : A}P(x)$ & ${\forall x : A}\, P(x)$ \\
		$\sum_{x : A}P(x)$ & ${\exists x : A}\, P(x)$
	\end{tabular}
\end{center}

进一步, \emph{证明是一种计算}. 例如证明 $A \land (A\Rightarrow B)\Rightarrow B$ 对应于取值映射 $\operatorname{ev}\colon A \times B^A \to B$. 更复杂的证明对应更复杂的计算. 要检查证明的正确性, 只需检查每一步得到的对象拥有正确的类型.

\section{模态逻辑}

\label{appendix-modal-logic}

在形式逻辑中, \emph{模态} (modality) 或\emph{模态算子} (modal operator) 是一种将命题变为命题的算子, 通常用 $\square$, $\lozenge$, $\bigcirc$ 等符号表示. 直觉上, 一个模态算子 $\square$ 的含义是
$$\square p = \text{``$p$ 以某种方式成立''}.$$
模态的作用正如自然语言中的\emph{情态动词} (modal verb), 如 ``可能'' (can), ``必须'' (must), ``将要'' (would).

\begin{example}
	{(可能性与必然性)}
	设 $\square$ 表示 ``必然性'', $\lozenge$ 表示 ``可能性''. 如下是两者的一些性质:
	\begin{itemize}
		\item $\square p \Rightarrow p$ (必然成立蕴涵实际上成立), $p\Rightarrow \lozenge p$ (实际上成立蕴涵可能成立);
		\item $\square (p \land q) = \square p \land \square q$, $\lozenge (p\lor q) = \lozenge p \lor \lozenge q$;
		\item $\square\square p = \square p$, $\lozenge\lozenge p =\lozenge p$;
		\item ...
	\end{itemize}
	可能性与必然性模态的一种实现方式是考虑 ``所有可能世界的集合'', 或 ``所有可能观测结果的集合''. 一个命题\emph{可能}成立就是说\emph{存在}一个可能世界使得该命题成立; 一个命题\emph{必然}成立就是说在\emph{任意}可能世界中该命题都成立.
	考虑一个\topos{} $\mathcal C$ 中对象 $X$ 产生的三元伴随 (命题 \ref{over-topos-essential-geometric-morphism}, 注 \ref{quantifier-exists-as-left-adjoint})
	\[\begin{tikzcd}[ampersand replacement=\&,background color=\examplecolor]
		{\mathcal C/X} \&\& {\mathcal C,}
		\arrow[""{name=0, anchor=center, inner sep=0}, "{X^*}"{description, pos=0.3}, from=1-3, to=1-1]
		\arrow[""{name=1, anchor=center, inner sep=0}, "{\Sigma_X}"{description, pos=0.3}, shift left=5, from=1-1, to=1-3]
		\arrow[""{name=2, anchor=center, inner sep=0}, "{\Pi_X}"{description, pos=0.3}, shift right=5, from=1-1, to=1-3]
		\arrow["\dashv"{anchor=center, rotate=-90}, draw=none, from=1, to=0]
		\arrow["\dashv"{anchor=center, rotate=-90}, draw=none, from=0, to=2]
	\end{tikzcd}\]
	想象 $X$ 为 ``所有可能世界的集合'', 而 $\mathcal C/X$ 的对象是在不同可能世界中变化的集合. 令
	$$
	\lozenge = X^*\Sigma_X,\quad
	\square = X^*\Pi_X.
	$$
	在此种实现下, 可能性 $\lozenge$ (存在) 是必然性 $\square$ (任意) 的左伴随.
\end{example}

