\chapter{前言}


每一个意象都是一个数学宇宙. 集合范畴 $\mathsf {Set}$ 是最简单和最重要的意象, 对应着 ``通常数学'' 的宇宙. 尽管一般意象可能有远比集合范畴丰富的结构, 其范畴论性质却与 $\mathsf {Set}$ 几乎相同. %意象的性质为几何学和逻辑学提供了良好的土壤: 几何学上, 每个意象中商空间, 纤维积, 映射空间等等构造有符合期望的性质; 逻辑学上, 每个意象都提供了一种语言以完全在范畴内部进行推理, 仿佛所处理的对象是普通集合一样. 对于熟悉的数学对象, 每一个意象都给我们一个新的视角; 一些关系在特定意象的语言中很简洁, 而在通常数学语言中则不然. 语言的重要性不言而喻.

拓扑空间 $X$ 上的层构成一个意象 $\operatorname{Sh}(X)$. 由此, 一般的意象可视为 ``广义空间'' 上的层范畴, 这种看法淡化范畴中的对象, 而将范畴当作一个整体. 意象还可视为一类直觉主义逻辑的语义. 意象理论的经典文献 \textit{Sketches of an Elephant} \cite{Elephant} 的开头列举了意象更多的解读方式, 正如盲人摸象一样.

本讲义参考了许多文献以及 \nlab 上的无数条目. 讲义内容的编排原则大致是使初学者最易接受, 并且提供启发性的观点, 从而使人能更快入门去看更多的资料. 如果本讲义成功写下去的话, 或许能够改善这一领域缺少中文教材的现状.


\todo{重写前言}

\todo{解释本书不处理集合论问题 \cite{stacks-project}}

\newpage

~\vspace{4em}

\philoquote{
	Je vous souhaite le meilleur succès. Ce serait magnifique que vous puissiez étudier la théorie des topos de Grothendieck et travailler dans ce domaine. Travaillez beaucoup, soyez patient, ayez bon courage et vos efforts seront récompensés.
}{
	Laurent Lafforgue\footnotemark
}
\footnotetext{这是 Lafforgue 教授在一次讲座之后写给作者的话. ``祝愿你获得最大的成功. 你能够学习 Grothendieck 意象理论并在这个领域工作, 是一件美妙的事情. 努力学习, 保持耐心, 勇往直前, 你的努力将会得到回报.'' Lafforgue 教授是 Caramello 教授多年的合作者.}

\philoquote{Once you see at least one example and you do it yourself and you experience the kind of enlightenment it brings, you will be convinced forever.}{Olivia Caramello\footnotemark}
\footnotetext{这是 Caramello 教授在与作者的采访中关于 topos 理论的评论. Caramello 教授是意象理论和逻辑学专家.}

% TODO 详细介绍参考文献