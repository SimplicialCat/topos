\chapter{前言}

\todo{重写}

每一个\topos{} (topos) 都是一个数学宇宙. 集合范畴 $\mathsf {Set}$ 是最简单和最重要的\topos{}, 对应着 ``通常数学'' 的宇宙. 一般\topos{}可能有远比集合范畴丰富的结构, 其范畴论性质却与 $\mathsf {Set}$ 几乎相同. 逻辑学上, 每个\topos{}都提供了一种语言以完全在范畴内部进行推理, 仿佛所处理的对象是普通集合一样. 对于熟悉的数学对象, 每一个\topos{}都给我们一个新的视角; 一些关系在特定\topos{}的语言中很简洁, 而在通常数学语言中则不然. 拓扑空间 $X$ 上的层构成一个\topos{} $\operatorname{Sh}(X)$; 由此, 一般的\topos{}可视为 ``广义空间'' 上的层范畴, 这种看法淡化范畴中的对象, 而将\topos{}当作一个整体. \topos{}理论的经典文献 \textit{Sketches of an Elephant} \cite{Elephant} 的开头列举了意象更多的解读方式, 正如盲人摸象一样.

本书对数学没有原创性的贡献, 其中几乎所有结果在 20 世纪已经为该领域的专家所熟知. 但是大部分内容几乎找不到成体系的中文资料 (除了李文威老师的\emph{代数学方法} \cite{lww2}), 这是我编写本书的动机之一. 内容的编排原则大致是使初学者最易接受, 并且提供启发性的观点, 从而使人能更快入门去看更多的资料. 本书收录的论证都是十分简单而直观的; 阅读本书虽不能让人成为本领域的专家, 但能让人发现一些事情并非想象的那样困难, 在将玄妙的 topos 祛魅的过程中产生信心和乐趣. 书中许多内容的含入仅仅是由于个人的品味. 一些证明出自本人的思考, 因此错漏是难免的. 限于水平, 书中许多细节无法深究, 包括有关集合论的大基数的问题, 以及其它涉及到数学基础的问题. 许多术语没有通行的中文版本, 姑且使用了本人的翻译.

在编写本书的过程中, 我得到了 Olivia Caramello 教授, 以及杨家同, 陈潇扬等友人的帮助和鼓励. 在本书参考的文献中, 最重要的是一个名叫 \nlab 的网站, 而其中最主要的贡献者是 Urs Schreiber 教授. 向他们表达诚挚的感谢.

\newpage

~\vspace{4em}

\philoquote{
	Je vous souhaite le meilleur succès. Ce serait magnifique que vous puissiez étudier la théorie des topos de Grothendieck et travailler dans ce domaine. Travaillez beaucoup, soyez patient, ayez bon courage et vos efforts seront récompensés.
}{
	Laurent Lafforgue\footnotemark
}
\footnotetext{这是 Lafforgue 教授在一次讲座之后写给作者的话. ``祝愿你获得最大的成功. 你能够学习 Grothendieck 意象理论并在这个领域工作, 是一件美妙的事情. 努力学习, 保持耐心, 勇往直前, 你的努力将会得到回报.'' Lafforgue 教授是 Caramello 教授多年的合作者.}

\philoquote{Once you see at least one example and you do it yourself and you experience the kind of enlightenment it brings, you will be convinced forever.}{Olivia Caramello\footnotemark}
\footnotetext{这是 Caramello 教授在与作者的采访中关于 topos 理论的评论. Caramello 教授是意象理论和逻辑学专家.}

% TODO 详细介绍参考文献