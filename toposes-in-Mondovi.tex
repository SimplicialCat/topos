\documentclass{article}
\usepackage[UTF8]{ctex}

\usepackage{amsthm}
\usepackage{amsmath}
\usepackage{amssymb}
\usepackage{mathrsfs}
\usepackage{hyperref}
\usepackage{stmaryrd}

\usepackage{tikz}
\usetikzlibrary{cd}
\usetikzlibrary{decorations.pathmorphing}

\usepackage{enumerate}

\newcommand{\philoquote}[2]{
	~\\
	\begin{center}
		\begin{minipage}{0.7\linewidth}
			{\quad\sffamily #1}\\
			\begin{flushright}
				\textsf{#2}
			\end{flushright}
		\end{minipage}
	\end{center}
	~\\
}

\usepackage{geometry}
\geometry{
	b5paper,
	left=15mm,
	right=15mm,
	top=20mm,
	bottom=20mm
}



% 术语翻译

\newcommand{\topos}{意象}
\newcommand{\regular}{常理}
\newcommand{\coherent}{贯理}
\newcommand{\cohesive}{凝集}
\newcommand{\cohesion}{凝集}

% 专有名词

\newcommand{\nlab}{$n$Lab}

% 常用记号

\newcommand{\op}{\text{op}} % 对偶范畴
\newcommand{\yo}{\!\text{{\mincho よ}}} % 米田嵌入
\newcommand{\Top}{\mathcal T\hspace{-3pt}opos}
\newcommand{\interpretation}[1]{{[\![#1]\!]}}
\newcommand{\upward}[1]{\uparrow{\!}{#1}}

% sequent calculus
\newcommand{\sqc}[2]{\dfrac{\quad #1 \quad}{\quad #2 \quad}}
\newcommand{\sqqc}[2]{
	\begin{array}
		{c}
		#1 \\ \hline \hline #2
	\end{array}
}

\newcommand{\internalprop}[1]{{\ulcorner {#1} \urcorner}}

\newcommand{\todo}[1]{{\color{red} [\textbf{未完成: #1}]}}




% 彩色方框, 参数的使用有点意思
% 这是打印版的参数 (用于省墨)

\usepackage{tcolorbox}

\tcbuselibrary{breakable}

\newcommand{\framecolor}{gray!50!white}

\newtcolorbox[
auto counter,
number within=section,
]{remark}[2][]{
	colback=white,
	colbacktitle=white,
	colframe=\framecolor,
	coltitle=black,
	breakable,
	title={\textsf{注~\thetcbcounter} #2},
	#1
}

\def\examplecolor{white}
\newtcolorbox[
use counter from=remark,
%number within=chapter,
]{example}[2][]{
	colback=white,
	colbacktitle=white,
	colframe=\framecolor,
	coltitle=black,
	breakable,
	title={\textsf{例~\thetcbcounter} #2},
	#1
}
\newtcolorbox[
use counter from=remark,
%number within=chapter,
]{definition}[2][]{
	colback=white,
	colbacktitle=white,
	colframe=\framecolor,
	coltitle=black,
	breakable,
	title={\textsf{定义~\thetcbcounter} #2},
	#1
}

\def\propcolor{white}
\newtcolorbox[
use counter from=remark,
]{prop}[2][]{
	colback=white,
	colbacktitle=white,
	colframe=\framecolor,
	coltitle=black,
	breakable,
	title={\textsf{命题~\thetcbcounter} #2},
	#1
}
\newtcolorbox[
use counter from=remark,
]{propdef}[2][]{
	colback=white,
	colbacktitle=white,
	colframe=\framecolor,
	coltitle=black,
	breakable,
	title={\textsf{命题-定义~\thetcbcounter} #2},
	#1
}

\newtcolorbox[
auto counter,
number within=chapter
]{exercise}[2][]{
	colback=white,
	colbacktitle=white,
	colframe=\framecolor,
	coltitle=black,
	title={\textsf{习题~\alph{\thetcbcounter}} #2},
	#1
}
\newtcolorbox[
use counter from=remark,
]{axiom}[2][]{
	colback=white,
	colbacktitle=white,
	colframe=\framecolor,
	coltitle=black,
	title={\textsf{公理~\thetcbcounter} #2},
	#1
}


\begin{document}
	
	\section{Claudio Fontanari: 模空间}
	
	\section{Joseph Bernstein: 什么是群表示}
	% 9/9 15:20
	\begin{definition}
		{(商群胚)}
		设群 $G$ 作用于集合 $X$, 定义\emph{商群胚} (quotient groupoid) $X/G$ 为
	\end{definition}
	
	\begin{prop}
		{}
		$$
		\operatorname{Sh}(X/G) \simeq \operatorname{Sh}_G (X).
		$$
	\end{prop}
	
	\section{Pivet: $2$-范畴上的层}
	% 9/10 13:45
	
	
	\section{Peter Haine: 由平展\topos{}重构概形}
	
	
	
	
	\section{景的态射与余态射}
	
	\section{Matthias Ritter (Hutzler): 综合代数几何}
	% 9/11 11:40
	
	我们使用的综合代数几何的语言是用同伦类型论表述的 Zariski ($\infty$-){\topos{}}的内语言.
	
	\begin{definition}
		{(射影空间)}
		$$
		\mathbb P^n := \sum_{L \subset R^{n+1}\,\text{子模}} \|L\simeq R^1 \|_{\text{prop}}.
		$$
	\end{definition}
	
	\begin{definition}
		{(抽象直线的空间, 线丛, Picard 群)}
		定义\emph{抽象直线的空间} (space of abstract lines)
		$$
		BR^\times := \sum_{L: R\mathsf{Mod}} \| L\simeq R^1\|_{\text{prop}}.
		$$
		
		由于 $R$-模的张量积满足 $R^1\otimes R^1\simeq R^1$, 有运算
		$$
		\otimes\colon BR^\times \times BR^\times \to BR^\times,
		$$
		且 $\otimes$ 构成 $BR^\times$ 上的 (高阶) 群结构, 单位为 $R^1$, 逆为 $L\mapsto L^\vee=\operatorname{Hom}(L,R^1)$.
		
		定义空间 (类型) $X$ 上的\emph{线丛}为映射 $X\to BR^\times$.
		定义 $X$ 的 \emph{Picard 群}为
		$$
		\operatorname{Pic}(X) := \|X\to BR^\times\|_{\text{set}}.
		$$
	\end{definition}
	
	\begin{example}
		{(重言线丛, $\mathcal O(d)$)}
		射影空间 $\mathbb P^n$ 上的\emph{重言线丛} $\mathcal O(-1)$ 定义为
		$$
		\mathcal O(-1) \colon \mathbb P^n \to BR^\times,\, L\mapsto L.
		$$
		定义 $\mathcal O(d)$ 为 $\mathcal O(-1)$ 的 $(-d)$ 次张量积.
	\end{example}
	
	\begin{prop}
		{}
		$$
		\operatorname{Pic}(\mathbb P^n) \simeq\mathbb Z.
		$$
	\end{prop}
	
	\section{Michael Shulman: \topos{}图表的内语言}
	
	\begin{definition}
		{(\topos{}的图表)}
		定义一个\emph{\topos{}的图表}是一个 $2$-函子 $\mathcal M \to \Top$, 其中 $\mathcal M$ 是任意 $2$-范畴, $\Top$ 是\topos{}的 $2$-范畴.
	\end{definition}
	
	\begin{example}
		{}
		以下结构均为\topos{}的图表的特例.
		\begin{itemize}
			\item $\mathcal S$-\topos{}, 也即几何态射 $f\colon \mathcal E \to \mathcal S$;
			\item 局部 $\mathcal S$-\topos{}, 也即几何态射 $f\colon \mathcal E \to \mathcal S$ 及其 ...%左伴随 $c\colon \mathcal S\to\mathcal E$, 满足单位 $cf\to 1$ 为同构;
			\item 完全连通 $\mathcal S$-\topos{}.
		\end{itemize}
	\end{example}
\end{document}